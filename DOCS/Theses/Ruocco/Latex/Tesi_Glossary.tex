%------------------------------------------------------------------------------------------
% Meta-commands for the TeXworks editor
%
% !TeX root = ./Tesi.tex
% !TEX encoding = UTF-8
% !TEX program = pdflatex
%

%--------------------------------------------------------------------------------
%                                                G L O S S A R Y    E N T R I E S

%%% ---------------------------------------------------------------- Main glossary

\newdualentry{ACRF} % label
  {ACRF}            % abbreviation
  {Aicraft Construction Reference Frame}  % long form
  {The reference frame which has its origin in the fuselage forwardmost point, the x-axis pointing from the nose to the tail, the y-axis from fuselage plane of symmetry to the right wing (from the pilot's point of view) and the z-axis from pilot's feet to pilot's head} % description

%\newglossaryentry{ACRF}{%
%   name=Aicraft Construction Reference Frame (ACRF),
%   description={%
%      the reference frame which has its origin in the fuselage forwardmost point, the x-axis pointing from the nose to the tail, the y-axis from fuselage plane of symmetry to the right wing (from the pilot's point of view) and the z-axis from pilot's feet to pilot's head}
%}


\newdualentry{GUI}%
  {GUI}            % abbreviation
  {Graphical User Interface}  % long form
  {In computing, a Graphical User Interface is a type of interface that allows users to interact with electronic devices through graphical icons and visual indicators such as secondary notation, as opposed to text-based interfaces, typed command labels or text navigation}


\newdualentry{KBE}% label
  {KBE}            % abbreviation
  {Knowledge-Based Engineering}  % long form
  {KBE is essentially engineering on the basis of knowledge models. A knowledge model uses knowledge representation to represent the artifacts of the design process (as well as the process itself) rather than, or in addition to, conventional programming and database techniques. KBE can have a wide scope that covers the full range of activities related to Product Lifecycle Management and Multidisciplinary Design Optimization. KBE's scope includes design, analysis, manufacturing, and support} % description

\newdualentry{IDE}% label
  {IDE}            % abbreviation
  {Integrated Development Environment}  % long form
  {An integrated development environment (IDE) is a software application that provides comprehensive facilities to computer programmers for software development. An IDE normally consists of a source code editor, build automation tools and a debugger. Most modern IDEs have an intelligent code completion. Some IDEs contain a compiler, interpreter, or both, such as NetBeans and Eclipse; others do not, such as SharpDevelop and Lazarus. The boundary between an integrated development environment and other parts of the broader software development environment is not well-defined. Sometimes a version control system, or various tools to simplify the construction of a Graphical User Interface (GUI), are integrated. Many modern IDEs also have a class browser, an object browser, and a class hierarchy diagram, for use in object-oriented software development} % description

\newdualentry{MDO}% label
  {MDO}            % abbreviation
  {Multi-disciplinary Design Optimization}  % long form
  {Multi-disciplinary design optimization (MDO) is a field of engineering that uses optimization methods to solve design problems incorporating a number of disciplines.MDO allows designers to incorporate all relevant disciplines simultaneously. The optimum of the simultaneous problem is superior to the design found by optimizing each discipline sequentially, since it can exploit the interactions between the disciplines. However, including all disciplines simultaneously significantly increases the complexity of the problem} % description


\newglossaryentry{List}{%
	name=List,
	description={The java.util.List interface is a subtype of the java.util.Collection interface. It represents an ordered list of objects, meaning you can access the elements of a List in a specific order, and by an index too. You can also add the same element more than once to a List}
}

\newglossaryentry{Enum}{%
	name=Enumeration,
	description={The java.util.Enumeration interface represents a special data type that enables for a variable to be a set of predefined constants. The variable must be equal to one of the values that have been predefined for it. Because they are constants, the names of an enum type's fields are in uppercase letters.}
}

\newglossaryentry{Static}{%
	name=static method,
	description={The term static means that the method is available at the Class level, and so does not require that an object is instantiated before it's called}
}

\newglossaryentry{Map}{%
	name=Map,
	description={The java.util.Map interface represents a mapping between a key and a value. The Map interface is not a subtype of the Collection interface. Therefore it behaves a bit different from the rest of the collection types}
}

\newglossaryentry{Interface}{%
	name=Interface,
	description={In the Java programming language, an interface is a reference type, similar to a class, that can contain only constants, method signatures, default methods, static methods, and nested types. Method bodies exist only for default methods and static methods. Interfaces cannot be instantiated—they can only be implemented by classes or extended by other interfaces.}
}

\newglossaryentry{DATCOM}{%
	name=DATCOM,
	description={Digital Datcom is a computer program which calculates static stability, high lift and control, and dynamic derivative characteristics using the methods contained in the USAF Stability and Control Datcom (Data Compendium). Configuration geometry, attitude, and Mach range capabilities are consistent with those accommodated by the Datcom. The program contains a trim option that computes control deflections and aerodynamic increments for vehicle trim at subsonic Mach numbers}
}


\newglossaryentry{User:Developer}{%
   name=user developer,
   description={%
      The term refers to the developer which will use a method without being interested in how the method performs the required action. This is the case of a utility method: the developer is the one who writes the method, while the user developer is who uses that method to accomplish some action which requires the functionality provided by the utility method. It has to be noticed that the user developer and the developer can be the same person}
}
%
%%%% --------------------------------------------------------------- List of symbols
%%%% see _local_macros.tex
%
%
\newglossaryentry{mass}{%
   type=symbols,
   name={\ensuremath{m}},
   sort=m,
   description={mass, in kg or lb}
}

\newglossaryentry{weight}{%
   type=symbols,
   name={\ensuremath{W}},
   sort=m,
   description={weight, in Newtons}
}


\newglossaryentry{mass:Wing}{%
   type=symbols,
   name={\ensuremath{m_\mathrm{W}}},
   sort=mW,
   description={wing mass}
}

\newglossaryentry{diam}{%
   type=symbols,
   name={\ensuremath{d}},
   sort=m,
   description={diameter}
}

\newglossaryentry{length}{%
   type=symbols,
   name={\ensuremath{l}},
   sort=m,
   description={length}
}

\newglossaryentry{span}{%
   type=symbols,
   name={\ensuremath{b}},
   sort=m,
   description={span}
}

\newglossaryentry{velocity}{%
   type=symbols,
   name={\ensuremath{V}},
   sort=m,
   description={scalar velocity}
}

\newglossaryentry{surface}{%
   type=symbols,
   name={\ensuremath{S}},
   sort=m,
   description={surface}
}

\newglossaryentry{ar}{%
   type=symbols,
   name={\ensuremath{\AR}},
   sort=m,
   description={aspect ratio}
}

\newglossaryentry{taperRatio}{%
   type=symbols,
   name={\ensuremath{\lambda}},
   sort=m,
   description={taper ratio}
}

\newglossaryentry{sweep}{%
   type=symbols,
   name={\ensuremath{\Lambda}},
   sort=m,
   description={sweep}
}

\newglossaryentry{thickness}{%
   type=symbols,
   name={\ensuremath{t}},
   sort=m,
   description={thickness}
}

\newglossaryentry{chord}{%
   type=symbols,
   name={\ensuremath{c}},
   sort=m,
   description={chord}
}

\newglossaryentry{rho}{%
   type=symbols,
   name={\ensuremath{\rho}},
   sort=zz:rho,
   description={air density}
}

\newglossaryentry{alpha:Wing}{%
   type=symbols,
   name={\ensuremath{\alpha_\Wing}},
   sort=zz:alpha:W,
   description={angolo d'attacco riferito alla corda di radice dell'ala}
}

\newglossaryentry{beta}{%
   type=symbols,
   name={\ensuremath{\beta}},
   sort=zz:beta,
   description={sideslip angle}
}

\newglossaryentry{Drag}{%
   type=symbols,
   name={\ensuremath{D}},
   sort=D,
   description={aerodynamic drag}
}

\newglossaryentry{Lift}{%
   type=symbols,
   name={\ensuremath{L}},
   sort=L,
   description={aerodynamic lift}
}

\newglossaryentry{Thrust}{%
   type=symbols,
   name={\ensuremath{T}},
   sort=L,
   description={Thrust}
}

\newglossaryentry{q:bar}{%
   type=symbols,
   name={\ensuremath{q}},
   sort=q,
   description={dynamic pressure}
}

\newglossaryentry{Mach}{%
   type=symbols,
   name={\ensuremath{\Mach}},
   sort=M,
   description={Mach number}
}

\newglossaryentry{Reynolds}{%
   type=symbols,
   name={\ensuremath{\Reynolds}},
   sort=Re,
   description={Reynolds number (evaluated with respect to $\bar{c}$)}
}

\newglossaryentry{CG}{%
   type=symbols,
   name={CG},
   sort=G,
   description={Center of Gravity}
}

\newglossaryentry{i:Wing}{%
   type=symbols,
   name={\ensuremath{i_\Wing}},
   sort=iW,
   description={the angle between the wing root chord and the ACRF x-axis}
}

\newglossaryentry{i:Htail}{%
   type=symbols,
   name={\ensuremath{i_\Htail}},
   sort=iH,
   description={the angle between the horizontal tail root chord and the ACRF x-axis}
}

\newglossaryentry{sub:WF}{%
   type=symbols,
   name={\ensuremath{(\;)_\text{WF}}},
   sort=aaWB,
   description={quantity related to the wing-fuselage configuration}
}

\newglossaryentry{sub:F}{%
   type=symbols,
   name={\ensuremath{(\;)_\text{F}}},
   sort=aaWB,
   description={quantity related to the fuselage}
}

\newglossaryentry{sub:H}{%
   type=symbols,
   name={\ensuremath{(\;)_{\Htail}}},
   sort=aaH,
   description={quantity related to the horizontal tail}
}

\newglossaryentry{sub:V}{%
   type=symbols,
   name={\ensuremath{(\;)_{\Vtail}}},
   sort=aaV,
   description={quantity related to the vertical tail}
}

\newglossaryentry{sub:N}{%
   type=symbols,
   name={\ensuremath{(\;)_\text{N}}},
   sort=aaV,
   description={quantity related to the nacelle}
}

\newglossaryentry{sub:LG}{%
   type=symbols,
   name={\ensuremath{(\;)_\text{LG}}},
   sort=aaV,
   description={quantity related to the landing gear}
}

\newglossaryentry{sub:S}{%
   type=symbols,
   name={\ensuremath{(\;)_\text{S}}},
   sort=aaV,
   description={quantity related to systems}
}

%% TO DO:
%\item[$\bar{q} =$] pressione dinamica di volo, anche detta $q_{\infty}$ in Aerodinamica;
%per la (\ref{eq:Vequivalente})
%\item[$=$]$\dfrac{1}{2} \rho (u^2+v^2+w^2) = \dfrac{1}{2}\,\gamma\,p\,\Mach^2$;
%\item[$\rho =$] densità dell'aria alla quota effettiva di volo;
%\item[$p =$] pressione  statica alla quota effettiva di volo;
%\item[$\Mach =$] numero di Mach di volo;
%\item[$\gamma =$] rapporto dei calori specifici dell'aria $(=\SI{1.4}{})$;
%\item[$S =$] superficie di riferimento, tipicamente la superficie della forma in pianta dell'ala;
%\item[$b =$] apertura alare di riferimento;
%\item[$c =$] corda alare di riferimento, tipicamente la corda media aerodinamica dell'ala, detta anche $\bar{c}$.


%%% -------------------------------------------------------------------- Acronyms

%\newacronym[long={Aicraft Construction Reference Frame}]{acr:ACRF}{ACRF}{%
%  Aicraft Construction Reference Frame.\glspar
%      A reference frame which has its origin in the fuselage forwardmost point, the x-axis pointing from the nose to the tail, the y-axis from fuselage plane of symmetry to the right wing (from the pilot's point of view) and the z-axis from pilot's feet to pilot's head%
%}%

\newacronym{acr:ide}{IDE}{%
  Integrated Development Environment
}

\newacronym{acr:Jpad}{JPAD}{%
 Java Program toolchain for Aircraft Design
}


\newacronym{acr:Adopt}{ADOpT}{%
  Aircraft Design and Optimization Tool 
}

\newacronym{acr:gui}{GUI}{%
  Graphical User Interface
}

\newacronym{acr:cad}{CAD}{%
 Computer-Aided Design 
}

\newacronym{acr:brf}{BRF}{%
  Body Reference Frame
}


\newacronym{acr:lrf}{LRF}{%
  Local Reference Frame
}

\newacronym{acr:cg}{GC}{%
  Gravity Center
}

\newacronym{acr:Mean:Sea:Level}{MSL}{%
   Mean Sea Level
}

\newacronym{acr:MAC}{MAC}{%
   Mean Aerodynamic Chord
}

\newacronym{acr:MAPE}{MAPE}{%
   Mean Absolute Percentage Error
}

\newacronym{acr:MLW}{MLW}{%
   Maximum Landing Weight
}

\newacronym{acr:MZFW}{MZFW}{%
   Maximum Zero Fuel Weight
}

\newacronym{acr:MTOW}{MTOW}{%
   Maximum Take Off Weight
}

\newacronym{acr:AIAA}{AIAA}{American Institute of Aeronautics and Astronautics}
\newacronym{acr:FAA}{FAA}{Federal Aviation Administration}
\newacronym{acr:GNC}{GNC}{Guidance Navigation and Control}


