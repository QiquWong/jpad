\chapter*{Conclusions}
An aircraft design and optimization desktop application written in Java, and its functionality, has been introduced. The adoption of established software engineering practices, the use of advanced development tools, and concurrent development enabled the developer team to build a feature-rich application in a relatively short period of time. As of its design, the application is easily maintainable and extensible. The software is still growing and the choice of Java language was really helpful. Infact being Java a pure object oriented programming language, it greatly encourages and simplifies modularization. Each module (package) can be programmed quite independently so that it is relatively easy to divide the work among several programmers working simultaneously or one after the other.\\
The application, moreover, can be easily integrated into a comprehensive aircraft optimization cycle. As all analysis modules inside the JPADCore package will be completed and tested, the final purpose of the code will be to allow users to define a certain numbers of macroscopical geometrical parameters, along with a given objective function, and to receive as output the best set of the previous parameters which suits the wanted objective. These future targets will make the software able to carry out an analysis of an aircraft during its preliminary design phase in a fast and flexible way.