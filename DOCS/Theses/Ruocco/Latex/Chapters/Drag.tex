\chapter{Wing Drag Characteristics}

\label{ch:workobject}
\markboth{Wing Drag Characteristics}{}

\begin{flushright}
	{\smaller
		\textit{Citazione\\ citazione}\\
		-- Autore}
\end{flushright}


As mentioned in the previous chapter, the drag is the force component acting in the opposite direction to the airspeed vector.\\
There isn't a single classification of the drag but, dependent on the purpose of the work, the drag may be broken down in different way. Following will be explained the two main classification.

\begin{itemize}
\item The drag is subdivided using a causal breakdown. In this way the drag contributes are in accordance with the physical mecchanism such as the viscosity of the flow.
\item The drag is subdivided using a component breakdown. Every component of aircraft added an own drag contribute.
\end{itemize}

% foto componenti dal Roskam

According to the casual breakdown it's possible to make a preliminary division considering normal and tangential stress. The tangential forces produce the {\itshape friction drag}. While it's possible to divide the drag due of the normal component in viscous, that generates {\itshape form drag}, and inviscid. A further division can be made for the last one, in {\itshape induced drag}  and {\itshape wawe drag}.\\

% grafico componente causale  p376 Sforza pdf

% breve spiegazione vari contributi  p 377 Sforza pdf

% divisione per componente
% grafico componente -->> fai come per causale

\section{Theoretical background}
% affrontiamo il calcolo della drag dell ala
\subsection{Integral estimation}
% spiegazione di come fatto e come funziona
\subsection{Sforza}
% p 394 pdf

\section{Java Class Architecture}

% tabella con varie classi
% schema in yed

\section{Case Study}
% output


