\selectlanguage{english}
\begin{abstract}
The purpose of this Thesis work is to introduce the features and the potenciality of ADOpT ({\itshape Aircraft Design and Optimization Tool}), a java-based framework concieved as a fast and efficient tool useful as support in the preliminary design phases of an aircraft, and during its optimizaton process.\\
The ADOpT development originates in the Departement of Industrial Engineering of University of Naples ``Federico II'', where is still in development. At present this tool is capable to perform a multi-disciplinary analysis of an aircraft whose data can be entered by the user, with an XML, or loaded into memory. The ultimate goal of ADOpT is to carry out an optimization process where the analysis are cyclically repeated in order to optimize some parameters while keeping others in fixed limits.
Currently the software is able to esimate the aircraft weight breackdown, the center of gravity location, calculate some aerodynamic parameters and estimate the performance. All these types of estimates can be usually performed using several interchangeable analysis methods, comparable and interchangeable. 
It is also provided a static longitudinal stability analysis, take-off and landing performances and the generation of Payload Range chart.\\
ADOpT can be used from the command line or with a dedicated graphical user interface (GUI). The GUI allows the user to have an immediate feedback about the aircraft features when changing the input parameters, to manage multiple aircraft simultaneously and compare them side by side, and to view a 3D CAD model of the aircraft. \\
The ADOpT potentiality, in the world of research or in industry, are remarkable and the software strengths are a considerable computing speed and flexibility, with an user-friendly GUI.\\ \\
The structure of this thesis work has as ultimate goal to provide a comprehensive overview about ADOpT and, at the same time, it is intended to be a developer's manual. The first chapters provide a complete software overview paying particular attention at actual features and future goals.
Following chapters introduce  some case of study and the results achieved. At the beginning of each chapter is exposed the theoretical background, afterwards there is a description of the Java architecture and, at the end is reported the Test class used for the analysis and its results.

\end{abstract}

\selectlanguage{italian}
\begin{abstract}

Lo scopo che il presente lavoro di Tesi auspica raggiungere è quello di presentare le capacità possedute e le potenzialità future di ADOpT, un software scritto in Java che si configura come uno strumento veloce ed efficiente per il supporto nella fase di progetto preliminare di un velivolo e per la sua ottimizzazione. \\
Lo sviluppo di ADOpT ({\itshape Aircraft Design and Optimization Tool}) nasce all' interno del Dipartimento di Ingegneria Industriale dell' Università degli Studi di Napoli Federico II, ove è tutt'ora in fase di progresso. Il software è attualmente in grado di svolgere una parziale analisi multi-disciplinare di un velivolo i cui dati sono immessi dall' utente tramite XML o caricati in memoria. La linea guida dello sviluppo del software porta verso l'implementazione di un processo di ottimizzazione nel quale le analisi sono ciclicamente ripetute al fine di ottimizzare alcuni parametri mantenendone altri all' interno di limiti imposti.\\
Attualmente ADOpT è in grado di effettuare una completa stima dei pesi, valutare la posizione del baricentro, calcolare un notevole numero di parametri aerodinamici e caratteristiche di performance. La stima di ciascun parametro può essere effettuata tramite diversi metodi implementati, tra di loro confrontabili ed intercambiabili. Inoltre è prevista un' analisi di stabilità statica, prestazioni di decollo ed atterraggio e la generazione del diagramma  {\itshape Payload Range}.\\
ADOpT può essere utilizzato sia in modalità {\itshape batch}, ossia da riga di comando, che tramite interfaccia grafica. Tale duplice scelta consente di ottenere le migliori prestazioni sia nei processi di analisi, ove la GUI consente di avere un immediato riscontro grafico, sia nei processi di ottimizzazione.\\
Dunque le potenzialità di ADOpT nel mondo della ricerca od anche in quello industriale sono senza dubbio notevoli e il software gioca i suoi punti di forza in un' elevata flessibilità e una notevole rapidità di calcolo, senza dimenticare una {\itshape user-friendly} interfaccia grafica. \\ \\

L' organizzazione di questo lavoro di Tesi è stata studiata per cercare di fornire una completezza di esposizione, ma allo stesso tempo risultare un utile manuale per lo sviluppatore. I primi capitoli forniscono una visione globale del software con particolare attenzione alle funzionalità presenti e alle scelte effettuate.
I capitoli che seguono presentano una panoramica su alcune funzionalità di ADOpT con i relativi casi di studio e i risultati ottenuti dalle analisi. Per ogni capitolo viene preliminarmente fornita una visione globale circa la teoria alla base dei metodi implementati, seguita dalla descrizione delle classi e dei metodi  relativi in Java ed è, infine, riportato il codice della {\itshape Test Class} implementata per lo svolgimento delle analisi con i relativi risultati.





%La tesi descrive lo sviluppo diadopt,, un'applicazione scritta in linguaggio Java concepita per essere uno strumento veloce, affidabile e di semplice utilizzo per le fasi di sviluppo concettuale e preliminare di un velivolo da trasporto. Lo scopo finale del programma è effettuare un'analisi multidisciplinare di una configurazione definita dall'utente e successivamente alterarla in modo da ottenerne una ottimizzata. Il dominio di ricerca di tale configurazione è definito dall'utente tramite un apposito insieme di parametri.
%
%Allo stato attuale il programma effettua la stima dei pesi dei principali componenti di un velivolo, valuta la posizione del baricentro, i principali parametri aerodinamici e alcune derivate di stabilità. La stima di ciascun parametro può essere generalmente effettuata tramite diversi metodi tra loro intercambiabili. Il carico aerodinamico sull'ala, in particolare, è stato stimato tramite un metodo numerico tratto danasa blackwell. Molta attenzione è stata in generale dedicata alla validazione dei risultati forniti dall'applicazione.
%
%ADOpT può essere usato sia da riga di comando sia tramite un'apposita interfaccia grafica (GUI). Quest'ultima fornisce un riscontro immediato riguardo il cambiamento delle prestazioni del velivolo nel momento in cui l'utente modifica uno o più parametri di input; inoltre permette di gestire più velivoli (o più configurazioni dello stesso velivolo) simultaneamente, di confrontarne le caratteristiche e di visualizzarne il modello CAD. Questo è generato tramite la libreria opencascade e può essere salvato su file in modo da poterlo usare in altri applicativi.
%
%Per quanto riguarda i files di input e di output, l'applicazione accetta files in formato XML (eXtensible Markup Language) contenenti la configurazione del velivolo e può esportare i risulati sia in formato XML sia in formato XLS. I file XML di uscita possono esssere successivamente re-importati e quindi modificati tramite l'interfaccia grafica. L'applicazione è stata sviluppata facendo largo uso delle ultime caratteristiche del linguaggio Java introdotte da Oracle nel 2014 con la versione 8; questa include, tra l'altro, la piattaforma JavaFX che è stata usata per costruire il visualizzatore del modello CAD.
\end{abstract}
\selectlanguage{english}