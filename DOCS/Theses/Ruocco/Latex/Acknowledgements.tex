\chapter{Acknowledgements}

Ho avuto la fortuna di apprendere in questi mesi di tirocinio e tesi molto più di quanto avrei mai immaginato.  Per questo desidero ringraziare chi mi ha trasmesso conoscenze e passione con una dedizione che va ben oltre il Loro lavoro. \\\\\\
Ringrazio il Professor Fabrizio Nicolosi per la sua immensa professionalità e disponibilità. La figura del Professore, per me, non potrebbe avere altro nome.\\\\
Ringrazio, altresì il professore Agostino De Marco per tutto ciò che mi ha insegnato. Ho ricevuto notifiche del suo lavoro ad ogni ora del giorno e della notte, la sua curiosità per ciò che è nuovo e la passione nel renderlo conosciuto, non ha eguali.\\\\
Ringrazio di cuore, per la loro disponibilità, Pierluigi, Salvatore, Danilo e Vincenzo. Grazie per averci fatto spazio quando spazio non c’era,  grazie per averci aiutato ogni volta che ne avevamo bisogno e grazie anche per le finestre chiuse quando avevo sempre freddo.\\\\
Ho avuto la fortuna di lavorare per 6 mesi a questo splendido progetto. Tuttavia io non credo esista la fortuna. La fortuna è l’attimo in cui il talento incontra l’occasione. Io l’occasione l'ho avuta grandissima, spero di aver dimostrato a Voi un talento che ne sia stato almeno all’altezza.\\\\\\
Desidero ringraziare, infine, tutti quei Professori che durante questo percorso mi hanno trasmesso qualcosa. In fondo, credo sia facile trovare del buono quando lo si cerca. E del buono nell’università ce n’è davvero ancora.
\\\\\\\\\\
Vorrei dedicare questo lavoro di Tesi a tutti quelli che mi sono stati vicini, quando starmi vicino non è mai facile.\\\\
A Francesco, il mio futuro. Un futuro così non ci sarebbe mai stato senza di te. Mi hai aiutato a diventare la donna che nemmeno sapevo di essere, ma che tu vedevi da sempre. E tu sei diventato sempre più l'uomo della mia vita. Hai capito anche quando non c’era da capire e hai cercato di pensarla come me, dove nemmeno io sapevo cosa cercavo. Perché l'amore è compromesso, ma noi siamo un incastro perfetto.\\\\
A mia Madre, per tutte le volte che io piangevo e lei con me. Per quando ero felice e lei gioiva. A lei che ogni mia richiesta diventava lo scopo della sua giornata solo per avere un sorriso la sera. Gli abbracci che ti do ogni giorno, non sono mai abbastanza. A mia madre che é lo specchio della mia anima.\\\\
A mio fratello, che in questi due anni ha più vissuto all'estero che a casa. Ho dovuto prendere la difficile decisione di se era più fastidiosa la sua televisione accesa ogni giorno o l'assordante silenzio della sua assenza. Danilo, non sai che rumore faceva la tua stanza quando andavo da te dimenticandomi che eri via.\\\\
A mia zia Iaia. Perché tu sei la sicurezza di una casa e la dolcezza di una famiglia. Grazie per essere sempre presente anche quando non devi. Una vita senza di te non si può nemmeno immaginare. A mio zio Gianfranco, grazie anche a te.\\\\
A Giusi perché difficilmente incontrerò una persona dallo spirito puro come il tuo e difficilmente io sarei mai potuta legarmi a qualcuno come a te. Perché spero di non smettere mai di imparare da te come si fa ad essere puri. So che tu sei la parte più buona di me, per sempre.\\\\
Ad Antonetta e Giuseppe. Mi avete accolto da sempre come una figlia e non vi ringrazierò mai abbastanza per questo. Voi siete la mia seconda famiglia.\\\\
A Federica, la mia amica ritrovata. In fondo gli amici possono fare dei giri un po’ più lunghi, ma ritornano sempre a casa.\\\\
A Marina e Martina. Ci siamo conosciute per caso, ma continueremo a stare vicine per scelta.\\\\
A Marta perché mi ha sempre perdonato per tutte le volte che non so essere una buona amica.\\\\
Ai miei compagni di questo viaggio: Giovanni, Nicola. Mal che vada, ci vediamo al ristorante.\\\\
E a me stessa, per tutte le notti passate sui libri, per tutti i denti stretti troppo e il coraggio di osare che non mi ha mai abbandonato. A me che quella notte prima dell'ultimo esame continuavo a ripetermi "ma chi me lo fa fare?" e che trovavo la motivazione di nuovo, ogni ora. A me, che sono quello che sono grazie a voi.



