\pagestyle{pippo}
\chapter{Java Reference}
\label{ch:jr}
\markboth{Java Reference}{}

\section{Object and Class in Java}
Object oriented programming scales very well, from the most trivial of problems to the most complex tasks. It provides a form of abstraction that resonates with techniques people use to solve problems in their everyday life. And for most of the dominant ob ject-oriented languages there are an increasingly large number of
libraries that assist in the development of applications for many domains.\cite{javareference}\\

An entity that has state and behavior is known as an object. It can be physical or logical (tengible and intengible). The example of integible object is banking system.

An object has three characteristics:

\begin{itemize}
\item {\bfseries State}: represents data (value) of an object.
\item {\bfseries Behavior}: represents the behavior (functionality) of an object such as deposit, withdraw etc.
\item {\bfseries Identity}: Object identity is typically implemented via a unique ID. The value of the ID is not visible to the external user. But,it is used internally by the JVM to identify each object uniquely.
\end{itemize}

Object is an instance of a class. Class is a template or blueprint from which objects are created. So object is the instance (result) of a class.

\section{Packages}\label{sec:package}
A Java package is a mechanism for organizing Java classes into namespaces, providing modular programming. Java packages can be stored in compressed files called JAR files, allowing classes to be downloaded faster as groups rather than individually. Programmers also typically use packages to organize classes belonging to the same category or providing similar functionality.

A package provides a unique namespace for the types it contains.
In general, a namespace is a container for a set of identifiers (also known as symbols, names). Namespaces provide a level of direction to specific identifiers, thus making it possible to distinguish between identifiers with the same exact name. For example, a surname could be thought of as a namespace that makes it possible to distinguish people who have the same given name. In computer programming, namespaces are typically employed for the purpose of grouping symbols and identifiers around a particular functionality.
\section{Documentation}
The doc comments, which are a particular type of comments provided by Java, are automatically recognized by the most popular IDE programs; this helps the developer to quickly get an idea of the actions performed by each method, the parameter which has to be passed to it and the what it returns to the user.


