\begin{abstract}
This thesis project involves developing some modules of the java library named JPAD, which stands for Java toolchain of Programs for Aircraft Design. These modules refer to conventional airplanes flying at subsonic speeds and have been set up to compute aerodynamic coefficients, with regard to lateral-directional motion. The calculation formulas are mostly based on the book by Napolitano~\cite{book:Napolitano} and USAF DATCOM~\cite{book:USAF_DATCOM}.

Firstly in this work it is introduced the structure of JPAD library. Then the empirical methods and the formulas used in the processing functions are presented. Next a numerical test made on a regional turboprop is shown. And finally, in an appendix, it is presented how charts digitization has been made.
\end{abstract}

\newpage\thispagestyle{empty}
\selectlanguage{italian}
\begin{abstract}
Questo progetto di tesi prevede lo sviluppo di alcuni moduli della libreria java denominata JPAD (Java toolchain di Programs for Aircraft Design). Questi moduli si riferiscono ad aeroplani convenzionali che volano a velocità subsoniche e sono stati creati per calcolare i coefficienti aerodinamici relativi al moto latero-direzionale. Le formule di calcolo sono per lo più basate sul libro di Napolitano~\cite{book:Napolitano} e USAF DATCOM~\cite{book:USAF_DATCOM}.

In primo luogo in questo scritto sarà introdotta la struttura della libreria JPAD. Quindi saranno presentati i metodi empirici e le formule utilizzate nelle funzioni di calcolo. Successivamente sarà effettuato un test numerico su un turboelica regionale. E infine, in un'appendice, viene presentato il modo con cui è stata realizzata la digitalizzazione dei grafici.
\end{abstract}
\selectlanguage{english}