% !TeX program = PdfLaTeX

\documentclass[a4paper,12pt,oneside,openright]{book}


%--------------------------------------------------------------------------------------------------------------------------------------------
%PACKAGES
%--------------------------------------------------------------------------------------------------------------------------------------------

%Document settings
\usepackage[utf8]{inputenc}
\usepackage[T1]{fontenc}
\usepackage[full]{textcomp}
\usepackage[italian, english]{babel}
%Questi comandi servono per settare i linguaggi
\providecommand{\csletcs}[2]{%
\expandafter\let\csname#1\expandafter\endcsname\csname#2\endcsname}
\newcommand{\equivUC}[3][\def]{%
\expandafter#1\csname#2\expandafter\endcsname\expandafter{%
\csname#3\endcsname}}
\newcommand{\UPCASElang}[1]{
\uppercase{\def\temp{#1}}
\csletcs{l@\temp}{l@#1}
\equivUC{date\temp}{date#1}
\equivUC{captions\temp}{captions#1}
\equivUC{extras\temp}{extras#1}
\equivUC{noextras\temp}{noextras#1}
\equivUC[\edef]{\temp hyphenmins}{#1hyphenmins}
}
\UPCASElang{english}
\UPCASElang{italian}
\usepackage{csquotes}
\usepackage{indentfirst}    
\usepackage{geometry}
\geometry{a4paper, top=20mm, bottom=20mm, left=35mm, right=20mm } %MARGINI 
\raggedbottom  %Per non riempire tutta la pagina stirando il testo. puo lasciare spazi vuoti alla fine di una pagina
\linespread{1} 

%Other useful packages
\usepackage[TM]{ar} %aspect ratio symbol. To use it --> \AR
\usepackage[usenames,dvipsnames]{xcolor}
\usepackage{pgfplots}
\pgfplotsset{compat=1.13}
\usepackage{caption}
\usepackage{graphicx}
\usepackage{tabularx}
\usepackage{longtable}
\usepackage{multicol}
\usepackage{multirow}
\usepackage{setspace}
\usepackage{booktabs}
\usepackage[intlimits]{mathtools}
\usepackage{siunitx}
\usepackage{subfig}
\usepackage{rotating}
\usepackage{float}
\usepackage{fancyhdr}
\usepackage{floatflt}
\usepackage{xcolor}
\usepackage{colortbl}
\usepackage{wrapfig}
\usepackage{lipsum}
\usepackage[nouppercase, swapnames]{frontespizio}
\usepackage{adjustbox}
\usepackage{booktabs}
\usepackage{amsmath}
\usepackage{cancel}
\usepackage{listings}
\usepackage{lipsum}
\usepackage{courier}
\usepackage{xcolor}
\usepackage{relsize}
\usepackage{titlesec} 
\usepackage{appendix}
\usepackage{xspace}
\usepackage{floatflt}
\usepackage{varioref}
\usepackage{tabulary}
\usepackage[backend=biber]{biblatex}
\addbibresource{bibliography.bib}
\definecolor{grigio_chiaro}{gray}{0.85}
%\usepackage[autostyle,italian=guillemets]{csquotes}
%\usepackage[backend=biber]{biblatex}

%---------------------------------------------------------------
% Listing settings
%----------------------------------------------------------------

\DeclareCaptionFormat{listing}{{\textwidth+17pt\relax\centering}\par\vskip1pt#1#2#3}
\captionsetup[lstlisting]{format=listing,singlelinecheck=false, justification=centering, margin=0pt, font={rm},labelsep=space,labelfont=bf}


\definecolor{light-gray}{gray}{0.97}
\lstset{frameround=fttt}
\lstset{language=Java}
\lstset{%
backgroundcolor=\color{light-gray}}
\lstset{basicstyle=\scriptsize\ttfamily,
keywordstyle=\color{blue}\bfseries,
commentstyle=\color{OliveGreen},
stringstyle=\color{blue},
showstringspaces=true}


\lstset{framextopmargin=50pt,frame=bottomline}

\usepackage { fancyhdr }
\newcommand {\fncyblank }{\fancyhf {}}
\newenvironment { abstract }%
{\cleardoublepage \fncyblank \null \vfill \begin { center }%
\bfseries \abstractname \end { center }}%
{\vfill \null }

\newenvironment{abstract}%
{\cleardoublepage%
  \thispagestyle{empty}%
  \null \vfill\begin{center}%
  \bfseries \huge\scshape  \abstractname \end{center}}%
{\vfill\null}

\newcommand{\myChapterHeadingColorMix}{%
  %blue!65!black
  gray!0!black%
}

\newcommand{\myChapterHeadingColor}{%
  %blue!65!black
  blue!0!black%
}


\usepackage{fancyhdr}


\renewcommand{\chaptermark}[1]{\markboth{#1}{}}
\lhead {{\bfseries \chaptername \ \thechapter} \; \leftmark}
\chead{}
\rhead{ \thepage }
\lfoot{}
\cfoot{}
\rfoot{\scriptsize Manuela Ruocco - Development of a Java-Based Framework for Aircraft Preliminary Design}
\renewcommand{\headrulewidth}{0.4pt}
\renewcommand{\footrulewidth}{0.4pt}


\fancypagestyle{pippo}{%
\pagestyle{fancy}
\lhead{}
\chead{}
\rhead{ \thepage }
\lfoot{}
\cfoot{}
\rfoot{\scriptsize Manuela Ruocco - Development of a Java Application for Aircraft Preliminary Design}
\renewcommand{\headrulewidth}{0.4pt}
\renewcommand{\footrulewidth}{0.4pt}

}


\titleformat{\chapter}[display]
  {  \bfseries \Large}
  {\filleft \huge { \bfseries{\chaptertitlename}}   \lapbox[6pt]{\width} 
 \ \colorbox{\myChapterHeadingColorMix}{\color{white}\Huge \strut {\thechapter}}}
  {3ex}
  {\titlerule\vspace{2ex}\filright}
  [\vspace{2ex}{\titlerule[2pt]}]



%------------------------------------------------------------------------------------------
% One of the last package to be loaded must be hyperref

\usepackage[%
            %dvipdfmx,%dvips,%
            %pdfborder = 0 0 1,
            baseurl= http://,
            colorlinks=true,%
            linkcolor=black,% black
            citecolor=black% black
            ]{hyperref}
\usepackage{cleveref}
%\usepackage{createspace}


%\addbibresource%
%%   [datatype=bibtex]%
%   {Tesi.bib}% extension required

%*************************************************************************************
% BIBLATEX SETTINGS POST-HYPERREF

\bibliography{Tesi_Bibliography}% with biblatex
\defbibheading{myBibliography}[\bibname]{\chapter*{\centering#1}
   %\markboth{#1}{#1}
   \markboth{}{}
}

\DeclareCiteCommand{\citetitle}{}{\printfield{title}}{;}{}


% http://tex.stackexchange.com/questions/83440/inputenc-error-unicode-char-u8-not-set-up-for-use-with-latex
%\DeclareUnicodeCharacter{00A0}{ }
%\DeclareUnicodeCharacter{00A0}{~}

%*************************************************************************************


 % queste sono alcune impostazioni gia settate

% nuovi comandi
\newcommand\nbvspace[1][3]{\vspace*{\stretch{#1}}}
\newcommand\nbstretchyspace{\spaceskip0.5em plus 0.25em minus 0.25em}

%--------------------------------------------------------------------------------------------------------------------------------------------
%BEGIN DOCUMENT
%--------------------------------------------------------------------------------------------------------------------------------------------
\begin{document}

%--------------------------------------------------------------------------------------------------------------------------------------------
%COPERTINA/FRONTESPIZIO
%--------------------------------------------------------------------------------------------------------------------------------------------

%FRONTESPIZIO
\begin{frontespizio}
\Universita {Napoli Federico II}
\Divisione {Scuola Politecnica e delle Scienze di Base}
\Corso [Laurea Triennale]{Ingegneria Aerospaziale}
\Logo [3cm]{immagini/Logo_UNINA.pdf}
\Titoletto {Tesi di Laurea Triennale \\ in \\ Ingegneria Aerospaziale}
\Titolo {Assessment of minimum unstick speed \\ of a complete Aircraft using JPAD \\ (Java toolchain Program  for Aircraft Design)}
%\Sottotitolo {Wing Aerodynamic Analysis Module,\\ Aircraft Longitudinal Static Stability Module }
\Candidato {Bruno Spoti \\ Matricola N35/2018}
%\Relatore {Prof. Ing. Fabrizio Nicolosi }
\Relatore {Prof. Ing. Agostino De Marco}
\Correlatore{Ing. Manuela Ruocco}
\Annoaccademico {2017/2018}
\end{frontespizio}

% -----------------------------------------------------------------------------------------
%                                                                    INSERIRE PAGINA BIANCA
% -----------------------------------------------------------------------------------------

\newpage\null\thispagestyle{empty}\newpage\thispagestyle{empty}

% -----------------------------------------------------------------------------------------
%                                                                    D E D I C A  (non in indice)
% -----------------------------------------------------------------------------------------

\null\vspace{\stretch{1}}
\begin{flushright}
\textit{Sono sempre i sogni a dare forma al mondo,\\ sono sempre i sogni a fare la realtà.}
\end{flushright}
\vspace{\stretch{2}}\null
\newpage\null\thispagestyle{empty}

% -----------------------------------------------------------------------------------------
%                                                                   PREFAZIONE
% -----------------------------------------------------------------------------------------

\selectlanguage{english}
\begin{abstract}
The purpose of this Thesis work is to introduce the features and the potenciality of ADOpT ({\itshape Aircraft Design and Optimization Tool}), a java-based framework concieved as a fast and efficient tool useful as support in the preliminary design phases of an aircraft, and during its optimizaton process.\\
The ADOpT development originates in the Departement of Industrial Engineering of University of Naples ``Federico II'', where is still in development. At present this tool is capable to perform a multi-disciplinary analysis of an aircraft whose data can be entered by the user, with an XML, or loaded into memory. The ultimate goal of ADOpT is to carry out an optimization process where the analysis are cyclically repeated in order to optimize some parameters while keeping others in fixed limits.
Currently the software is able to esimate the aircraft weight breackdown, the center of gravity location, calculate some aerodynamic parameters and estimate the performance. All these types of estimates can be usually performed using several interchangeable analysis methods, comparable and interchangeable. 
It is also provided a static longitudinal stability analysis, take-off and landing performances and the generation of Payload Range chart.\\
ADOpT can be used from the command line or with a dedicated graphical user interface (GUI). The GUI allows the user to have an immediate feedback about the aircraft features when changing the input parameters, to manage multiple aircraft simultaneously and compare them side by side, and to view a 3D CAD model of the aircraft. \\
The ADOpT potentiality, in the world of research or in industry, are remarkable and the software strengths are a considerable computing speed and flexibility, with an user-friendly GUI.\\ \\
The structure of this thesis work has as ultimate goal to provide a comprehensive overview about ADOpT and, at the same time, it is intended to be a developer's manual. The first chapters provide a complete software overview paying particular attention at actual features and future goals.
Following chapters introduce  some case of study and the results achieved. At the beginning of each chapter is exposed the theoretical background, afterwards there is a description of the Java architecture and, at the end is reported the Test class used for the analysis and its results.

\end{abstract}

\selectlanguage{italian}
\begin{abstract}

Lo scopo che il presente lavoro di Tesi auspica raggiungere è quello di presentare le capacità possedute e le potenzialità future di ADOpT, un software scritto in Java che si configura come uno strumento veloce ed efficiente per il supporto nella fase di progetto preliminare di un velivolo e per la sua ottimizzazione. \\
Lo sviluppo di ADOpT ({\itshape Aircraft Design and Optimization Tool}) nasce all' interno del Dipartimento di Ingegneria Industriale dell' Università degli Studi di Napoli Federico II, ove è tutt'ora in fase di progresso. Il software è attualmente in grado di svolgere una parziale analisi multi-disciplinare di un velivolo i cui dati sono immessi dall' utente tramite XML o caricati in memoria. La linea guida dello sviluppo del software porta verso l'implementazione di un processo di ottimizzazione nel quale le analisi sono ciclicamente ripetute al fine di ottimizzare alcuni parametri mantenendone altri all' interno di limiti imposti.\\
Attualmente ADOpT è in grado di effettuare una completa stima dei pesi, valutare la posizione del baricentro, calcolare un notevole numero di parametri aerodinamici e caratteristiche di performance. La stima di ciascun parametro può essere effettuata tramite diversi metodi implementati, tra di loro confrontabili ed intercambiabili. Inoltre è prevista un' analisi di stabilità statica, prestazioni di decollo ed atterraggio e la generazione del diagramma  {\itshape Payload Range}.\\
ADOpT può essere utilizzato sia in modalità {\itshape batch}, ossia da riga di comando, che tramite interfaccia grafica. Tale duplice scelta consente di ottenere le migliori prestazioni sia nei processi di analisi, ove la GUI consente di avere un immediato riscontro grafico, sia nei processi di ottimizzazione.\\
Dunque le potenzialità di ADOpT nel mondo della ricerca od anche in quello industriale sono senza dubbio notevoli e il software gioca i suoi punti di forza in un' elevata flessibilità e una notevole rapidità di calcolo, senza dimenticare una {\itshape user-friendly} interfaccia grafica. \\ \\

L' organizzazione di questo lavoro di Tesi è stata studiata per cercare di fornire una completezza di esposizione, ma allo stesso tempo risultare un utile manuale per lo sviluppatore. I primi capitoli forniscono una visione globale del software con particolare attenzione alle funzionalità presenti e alle scelte effettuate.
I capitoli che seguono presentano una panoramica su alcune funzionalità di ADOpT con i relativi casi di studio e i risultati ottenuti dalle analisi. Per ogni capitolo viene preliminarmente fornita una visione globale circa la teoria alla base dei metodi implementati, seguita dalla descrizione delle classi e dei metodi  relativi in Java ed è, infine, riportato il codice della {\itshape Test Class} implementata per lo svolgimento delle analisi con i relativi risultati.





%La tesi descrive lo sviluppo diadopt,, un'applicazione scritta in linguaggio Java concepita per essere uno strumento veloce, affidabile e di semplice utilizzo per le fasi di sviluppo concettuale e preliminare di un velivolo da trasporto. Lo scopo finale del programma è effettuare un'analisi multidisciplinare di una configurazione definita dall'utente e successivamente alterarla in modo da ottenerne una ottimizzata. Il dominio di ricerca di tale configurazione è definito dall'utente tramite un apposito insieme di parametri.
%
%Allo stato attuale il programma effettua la stima dei pesi dei principali componenti di un velivolo, valuta la posizione del baricentro, i principali parametri aerodinamici e alcune derivate di stabilità. La stima di ciascun parametro può essere generalmente effettuata tramite diversi metodi tra loro intercambiabili. Il carico aerodinamico sull'ala, in particolare, è stato stimato tramite un metodo numerico tratto danasa blackwell. Molta attenzione è stata in generale dedicata alla validazione dei risultati forniti dall'applicazione.
%
%ADOpT può essere usato sia da riga di comando sia tramite un'apposita interfaccia grafica (GUI). Quest'ultima fornisce un riscontro immediato riguardo il cambiamento delle prestazioni del velivolo nel momento in cui l'utente modifica uno o più parametri di input; inoltre permette di gestire più velivoli (o più configurazioni dello stesso velivolo) simultaneamente, di confrontarne le caratteristiche e di visualizzarne il modello CAD. Questo è generato tramite la libreria opencascade e può essere salvato su file in modo da poterlo usare in altri applicativi.
%
%Per quanto riguarda i files di input e di output, l'applicazione accetta files in formato XML (eXtensible Markup Language) contenenti la configurazione del velivolo e può esportare i risulati sia in formato XML sia in formato XLS. I file XML di uscita possono esssere successivamente re-importati e quindi modificati tramite l'interfaccia grafica. L'applicazione è stata sviluppata facendo largo uso delle ultime caratteristiche del linguaggio Java introdotte da Oracle nel 2014 con la versione 8; questa include, tra l'altro, la piattaforma JavaFX che è stata usata per costruire il visualizzatore del modello CAD.
\end{abstract}
\selectlanguage{english} 
\newpage
% !TeX program = PdfLaTeX
% !TeX root = ../Main.tex

\renewcommand{\abstractname}{Sommario}
\begin{abstract}
Lo scopo di questo lavoro di tesi è implementare una specifica routine Java per valutare la
VMU di un aereo completo. Questo è stato fatto implementando nuove ca\-rat\-te\-ri\-sti\-che in JPAD
(Java toolchain Program for Aircraft Design), un framework scritto in Java concepito come un veloce ed
efficiente strumento a supporto nelle fasi preliminari di progettazione di un aereo e durante il suo
processo di ottimizzazione. \\
Lo sviluppo di JPAD ha origine nel Dipartimento di Ingegneria Industriale dell'Università di Napoli "Federico II", dove è ancora in corso.
JPAD è un software dotato di un'interfaccia grafica (GUI) e, al momento, è in grado di eseguire un'analisi multidisciplinare
di un aeromobile i cui dati possono essere inseriti dall'utente, con un file XML, o caricati in memoria. \\

La struttura di questo lavoro di tesi è articolata in 3 capitoli e 1 appendice. \\
Prima di tutto, viene presentata la struttura di JPAD. \\
Il secondo capitolo riguarda un background teorico il quale, partendo da una breve introduzione della velocità di decollo, si sposterà sulla spiegazione dell'effetto suolo e sulla sua implementazione. Questo capitolo termina con la presentazione della VMU. \\
Il terzo capitolo riguarda un caso studio eseguito su un turboelica regionale di 130 passeggeri, riguardante le caratteristiche aerodinamiche, i dispositivi di ipersostentazione e l'effetto suolo sia sulla curva di portanza dell'ala che sull'angolo di downwash. \\
Lo sviluppo delle nuove funzionalità Java, in JPAD, è stato fatto prendendo in con\-si\-de\-ra\-zio\-ne la possibilità di ulteriori sviluppi per consentire l'introduzione di nuovi metodi. \\
Alla fine, nell'appendice A, viene mostrato come digitalizzare un grafico, creare un set di dati in formato .h5 e impostare la classe che legge di database in JPAD.
\end{abstract}

% -----------------------------------------------------------------------------------------
%                                                                   INDICE  (non in indice)
% -----------------------------------------------------------------------------------------

\tableofcontents 
\listoffigures
% -----------------------------------------------------------------------------------------
%                                                                 C A P I T O L I
% -----------------------------------------------------------------------------------------

\chapter{Introduction to JPAD}
\label{ch1}

Nowadays the preliminary design phase of an aircraft is becoming very challenging due to the need for more demanding requirements which deals with different fields of applications. In this perspective, there is a certain need for simple design tools both in aircraft industries and academic research groups which can perform fast and reliable multi-disciplinary analyses and optimizations.

This chapter provides a comprehensive overview of JPAD (Java toolchain of Programs for Aircraft Design) \cite{paper:JPAD}, a Java-based open-source library conceived as a fast and efficient tool useful as support in the preliminary design phases of an aircraft, and during its optimization process. The library has been completely realized at the Department of Industrial Engineering of the University of Naples “Federico II” where is still in development.

One of the main features of JPAD lies in the smart management of both the aircraft parametric model, which is conceived as a set of interconnected and parameterized components, and the available analyses. The library has been developed with the purpose of simplify the composition of the input file for the user and doing fast analysis with a satisfying grade of accuracy. The following section will focus on the description of the software structure and its main advantages. Another key point is the possibility to easily interface JPAD with other external tools in order to achieve a higher level of accuracy.

The JPAD library is an alternative to a plethora of similar software tools, both freeware and commercial \cite{AAA} \cite{Piano5} \cite{RDS}. Most of these tools have an important history, and many of them have been in use for decades. Some of them were conceived with poor software design criteria, have a rigid textual input and come with no visualization features. This is the main reason why JPAD has been developed paying a lot of attention to simplicity and flexibility. Moreover, it has been conceived as an open-source tool.

% --------------------------------------------------------------------------------------------------------------------------------------------
% SEZIONE 1
% --------------------------------------------------------------------------------------------------------------------------------------------
\section{Software structure}
\label{sec1.1}

To achieve a clear input file organization a considerable study has been done. The result is an input structure composed by different XML files with the purpose to allow users to easily manage all data needed to execute the desired analyses. In figure~\vref{JPADSchematicFlowChart} the entire structure of the software is schematized. It is possible to clearly note that there are two main blocks: input and core.

\begin{figure}[htbp] 
\centering
\includegraphics[height=0.45\textheight]{Immagini/Capitolo1/1_1-JPADSchematicFlowChart}
\caption[JPAD schematic flow-chart] {JPAD schematic flow-chart}
\label{JPADSchematicFlowChart}
\end{figure}

\begin{figure}[htbp] 
\centering
\includegraphics[height=0.45\textheight]{Immagini/Capitolo1/1_2-AnExampleOfTheAnalysisXmlFile}
\caption[Example of analysis.xml file] {An example of the analysis.xml file}
\label{AnExampleOfTheAnalysisXmlFile}
\end{figure}

\begin{figure}[htbp] 
\centering
\includegraphics[width=\textwidth]{Immagini/Capitolo1/1_3-AnExtractFromAGeneralAircraftXmlInputFile}
\caption[Extract from general Aircraft.xml input file] {An extract from a general Aircraft.xml input file}
\label{AnExtractFromAGeneralAircraftXmlInputFile}
\end{figure}

The input block is defined by two main parts: aircraft and analyses definitions. The first one defines the aircraft model in parametric way using a main file (Aircraft.xml, see figure~\ref{AnExtractFromAGeneralAircraftXmlInputFile}) which collects all the components, linking them to their related XML file (i.e. fuselage.xml, vtail.xml, and so on) which contains all geometrical data.

The second one defines all necessary data for each analysis presents into core module (see figure~\ref{AnExampleOfTheAnalysisXmlFile}). Since the aircraft model contains only geometrical data, it is necessary to define several further data referred to each analysis. \\

The structure described above allows to generate different aircrafts, or different configurations of the same model, combining different components. The possibility to generate a series of different aircrafts in a simple and fast way, allows to easily perform comparisons between these latter. For example, assuming different wings and engines, it is possible to estimate the effects that some design parameters have on a specific output. Figure~\vref{FAR-25} shows how the FAR-25 take-off field length behaves with different values of the wing surface and the engine static thrust at fixed aircraft maximum take-off weight. This feature plays also a key role in the optimization process described in figure~\ref{JPADSchematicFlowChart}.

\begin{figure}[htbp] 
\centering
\includegraphics[width=\textwidth]{Immagini/Capitolo1/1_4-FAR-25TakeOffFieldLengthAtDifferentWingLoadingsAndThrustWeightRatios}
\caption[FAR-25 take-off field length at different $W/S_\text W$ and $T/W$] {FAR-25 take-off field length at different wing loadings and thrust-weight ratios}
\label{FAR-25}
\end{figure}

In the same way, it is possible to perform a complete analysis (those present into core block in figure~\ref{JPADSchematicFlowChart}), or a specific one, combining different analyses files. This allows an easier evaluation of generic cost function during optimization tasks, resulting in reduced amount of computational costs required for this kind of operations.

Besides the input, the second main block is the core, which manages all the available analyses. This contains several independent modules, as shown in the figure~\ref{JPADSchematicFlowChart}, that deals with following application fields.
\begin{itemize}
\item Weights: estimates the aircraft weight breakdown starting from a first guess maximum take-off weight and some mission requirements. In particular, it evaluates each aircraft component mass using well-known semi-empirical equations.
\item Balance: estimates the centre of gravity position related to each weight condition and draws the balance diagram.
\item Aerodynamics and Stability: the aerodynamics module estimates all the aerodynamic characteristics concerning lift, drag and moments coefficients at different operating conditions for each aircraft component (wing, tails, fuselage and nacelles), whereas the stability module gives useful data about static stability of the whole aircraft.
\item Performance: evaluates most important aircraft performance such as Payload-Range diagram, mission profile, cruise flight envelope, ground performance, climb performance and the cruise grid chart.
\item Costs: estimates the DOC (Direct Operating Costs) breakdown.
\end{itemize}

Finally, JPAD allows to obtain different kind of output: charts and data in XLS format.

% --------------------------------------------------------------------------------------------------------------------------------------------
% SEZIONE 2
% --------------------------------------------------------------------------------------------------------------------------------------------
\section{The Java Language}
\label{sec1.2}

Java was developed by Sun Microsystems, a company that was incorporated in Oracle from a few years. This programming language is a general-purpose, concurrent, class-based, object-oriented language. It is designed to be simple enough that many programmers can achieve fluency in the language \cite{javaoracle}.

One design goal of Java is portability, which means that programs written for the Java platform must run similarly on any combination of hardware and operating system with adequate runtime support. This is achieved by compiling the Java language code to an intermediate representation called Java bytecode, instead of directly to architecture-specific machine code. Java bytecode instructions are analogous to machine code, but they are intended to be executed by a virtual machine (VM) written specifically for the host hardware. End users commonly use a Java Runtime Environment (JRE) installed on their own machine for standalone Java applications, or in a web browser for Java applets \cite{wiki:java}. \\

There were five primary goals in the creation of the Java language \cite{java}:
\begin{itemize}
\item it must be ``simple, object-oriented, and familiar'';
\item it must be ``robust and secure'';
\item it must be ``architecture-neutral and portable'';
\item it must execute with ``high performance'';
\item it must be ``interpreted, threaded, and dynamic''.
\end{itemize}

Actually Java is the most used programming language according to TIOBE (see figure~\vref{TIOBE}). The TIOBE Programming Community index is an indicator of the popularity of programming languages. The ratings are based on the number of skilled engineers world-wide, courses and third party vendors.

\begin{figure}[htbp]
\centering
\includegraphics[width=\textwidth]{Immagini/Capitolo1/1_5-TrendTIOBE} 
\caption{TIOBE Programming Community index (\href{www.tiobe.com}{www.tiobe.com})}
\label{TIOBE}
\end{figure}

% --------------------------------------------------------------------------------------------------------------------------------------------
% SEZIONE 3
% --------------------------------------------------------------------------------------------------------------------------------------------
\section{Java choice}
\label{sec1.3}

The choice of Java as the programming language was driven by several considerations. These include the following:
\begin{itemize}
\item the language should be widely supported; this to avoid the case of many valid aircraft design applications and libraries that became obsolete due to the aging of the programming language used to build them;
\item the language is object oriented;
\item the language should promote the use of open source libraries, especially for I/O tasks and for complex mathematical operations;
\item the language and the companion Integrated Development Environment (IDE) should provide a widely supported Graphical User Interface (GUI) framework and a GUI visual builder;
\item the language should support and promote modularity.
\end{itemize}

The Java programming language meets all these requirements; moreover it is backed by Oracle and by a huge community of developers so it is continuously updated. Also, advanced and free IDEs (such as Eclipse) allow programmers to streamline and simplify the development process; in particular, the Eclipse IDE has been chosen to develop JPAD.

Being Java a pure object oriented programming language, it greatly encourages and simplifies modularization. Each module (package) can be programmed quite independently so that it is relatively easy to divide the work among several programmers. This is essential since the amount of classes and calculations needed to abstract, manage and analyse the entire aircraft is very large (presently the whole project counts more than 10 millions lines of code). For such a reason the establishment of common practices and the adherence to fundamental principle of software development (\emph{Don’t Repeat Yourself}, \emph{Separation of Concerns}, \emph{Agile software development}) are equally important.
\chapter{Calculation formulas}
\label{ch2}

The purpose of this chapter is to provide the calculation formulas implemented in JPAD software and used to evaluate the aerodynamic coefficients with regard to the aircraft as a whole. The calculation formulas are mostly based on the well known USAF DATCOM~\cite{book:USAF_DATCOM}. In particular, both for the specific formulations and for the symbols, we have used a single reference, namely the textbook by Napolitano~\cite{book:Napolitano}.

In the following we will recall some of the most important formulas used to model the lateral force coefficient and the rolling and yawing moment coefficients. These assume the validity of the superposition principle and express all quantities referred to the aircraft as a sum of a number of contributions. Each aerodynamic coefficient is computed first evaluating the contribution due to each part taken singularly; then the extra contribution due to aerodynamic interferences are estimated; finally, all contributions are summed up.

From now on, all the angular gradients will be evaluated in \si{rad^{-1}}.

% --------------------------------------------------------------------------------------------------------------------------------------------
% SEZIONE 1
% --------------------------------------------------------------------------------------------------------------------------------------------
\section{Steady-state lateral force coefficient}
\label{sec2.1}

The steady-state lateral force can be evaluated by the relationship:
\begin{equation}
\label{eq:LateralForce}
Y = \CY \overline{q} \SW
\end{equation}
where the lateral force coefficient can be expressed by:
\begin{equation}
\CY = f(\beta,\deltaA,\deltaR)
\end{equation}
The first order approximation for the Taylor expansion gives the following expression for \CY:
\begin{equation}
\CY = \CYzero + \CYbeta \beta + \CYdeltaa \deltaA + \CYdeltar \deltaR
\end{equation}
in which the term \CYzero is zero if the aircraft is symmetric with respect to the \emph{XZ} plane.

% --------------------------------------------------------------------------------------------------------------------------------------------
% SOTTOSEZIONE 1.1
% --------------------------------------------------------------------------------------------------------------------------------------------
\subsection{Sideslip angle effect}
\label{subsec2.1.1}

The aerodynamic coefficient $\CYbeta$ can be expressed through its dependencies as shown in the following formula:
\begin{equation}
\label{eq:sideslipeffect1}
\CYbeta =  \CYbetaWB + \CYbetaH + \CYbetaV
\end{equation}
In this equation there are the contributions of the wing-body configuration, of the horizontal tail and of the vertical tail. For a detailed modelling of \CYbeta, it is useful to separate the contribution of the wing and of the fuselage:
\begin{equation}
\label{eq:sideslipeffect2}
\CYbeta =  \CYbetaW + \CYbetaB + \CYbetaH + \CYbetaV
\end{equation}

A relationship for the wing contribution is given by:
\begin{equation}
\label{eq:sideslipW}
\CYbetaW = -0.00573 \abs \GammaW
\end{equation}

A relationship for the fuselage contribution is instead given by:
\begin{equation}
\label{eq:sideslipB}
\CYbetaB = -2 K_{\mathrm{int}} \frac{S_{\mathrm P \rightarrow \mathrm V}}{\SW}
\end{equation}
where $k_{\mathrm{int}}$ is an interference factor calculated from figure~\vref{Kint}, whereas $S_{\mathrm P \rightarrow \mathrm V}$ is the cross section at the location of the fuselage where the flow ceases to be potential.

\begin{figure}[htbp] 
\centering
\includegraphics[width=0.75\textwidth]{Immagini/Capitolo2/4_8-Kint}
\caption[Wing-body interference factor] {Wing-body interference factor}
\label{Kint}
\end{figure}

The horizontal tail contribution is given by:
\begin{equation}
\label{eq:sideslipH}
\CYbetaH = -0.00573 \abs \GammaH \etaH \biggl( 1 - \frac{\mathrm{d}\sigma}{\mathrm{d}\beta} \biggr) \frac{\SH}{\SW}
\end{equation}

The vertical tail contribution is expressed in the following formula:
\begin{equation}
\label{eq:sideslipV}
\CYbetaV = - k_{Y_\mathrm V} \big|\CLalphaV \big| \etaV \biggl( 1 - \frac{\mathrm{d}\sigma}{\mathrm{d}\beta} \biggr) \frac{\SV}{\SW}
\end{equation}
where $k_{Y_\mathrm V}$ is an empirical factor shown in figure~\vref{KYV}.

\begin{figure}[htbp] 
\centering
\includegraphics[width=0.75\textwidth]{Immagini/Capitolo2/4_13-KYV}
\caption[Empirical factor lateral force vertical tail due to $\beta$] {Empirical factor for the lateral force at the vertical tail due to $\beta$}
\label{KYV}
\end{figure}

% --------------------------------------------------------------------------------------------------------------------------------------------
% SOTTOSEZIONE 1.2
% --------------------------------------------------------------------------------------------------------------------------------------------
\subsection{Ailerons deflection effect}
\label{subsec2.1.2}

The ailerons are an asymmetric control surface and the forces associated with their deflections act along the vertical and horizontal directions; therefore, their components along the lateral direction is negligible. Thus, we have:
\begin{equation}
\label{eq:aileronlateralforce}
\CYdeltaa \approx 0
\end{equation}

% --------------------------------------------------------------------------------------------------------------------------------------------
% SOTTOSEZIONE 1.3
% --------------------------------------------------------------------------------------------------------------------------------------------
\subsection{Rudder deflection effect}
\label{subsec2.1.3}

The rudder is a control surface. A mathematical relationship for \CYdeltar is given by:
\begin{equation}
\label{eq:rudderlateralforce}
\CYdeltar = \big|\CLalphaV \big| \etaV \frac{\SV}{\SW} \textrm \textDelta (K_{\textrm r}) \taurud
\end{equation}
In this relationship the term \CLalphaV is the lift-curve slope for the vertical tail, $\textrm \textDelta (K_{\textrm r})$ is a correction factor associated with the span of the rudder within the span of the vertical tail, evaluated graphically from figure~\vref{KR} using $\eta$ of the inner and outer rudder station, and \taurud is a control surface effectiveness factor, calculated from figure~\vref{taurudder}.

\begin{figure}[htbp] 
\centering
\includegraphics[width=0.75\textwidth]{Immagini/Capitolo2/4_27-KR}
\caption[Span factor rudder vertical tail] {Span factor between rudder and vertical tail}
\label{KR}
\end{figure}

\begin{figure}[htbp] 
\centering
\includegraphics[width=0.75\textwidth]{Immagini/Capitolo2/4_26-Effectiveness_Rudder}
\caption[Effectiveness of the rudder] {Effectiveness of the rudder \taurud as function of $\overline c_\text r/\overline c_\text V$}
\label{taurudder}
\end{figure}

% --------------------------------------------------------------------------------------------------------------------------------------------
% SEZIONE 2
% --------------------------------------------------------------------------------------------------------------------------------------------
\newpage
\section{Steady-state rolling moment coefficient}
\label{sec2.2}

The steady-state rolling moment can be evaluated by the relationship:
\begin{equation}
\label{eq:RollMoment}
\mathcal{L} = \CL \overline{q} \SW \bW
\end{equation}
where the rolling moment coefficient can be expressed by:
\begin{equation}
\CL = f(\beta,\deltaA,\deltaR)
\end{equation}
The first order approximation for the Taylor expansion gives the following expression for \CL:
\begin{equation}
\CL = \CLzeroRoll + \CLbeta \beta + \CLdeltaa \deltaA + \CLdeltar \deltaR
\end{equation}
in which the term \CLzeroRoll is zero if the aircraft is symmetric with respect to the \emph{XZ} plane.

% --------------------------------------------------------------------------------------------------------------------------------------------
% SOTTOSEZIONE 2.1
% --------------------------------------------------------------------------------------------------------------------------------------------
\subsection{Dihedral effect}
\label{subsec2.2.1}

The aerodynamic coefficient $\CLbeta$ is known as dihedral effect and it can be expressed through its dependencies as shown in the following formula:
\begin{equation}
\label{eq:dihedraleffect}
\CLbeta =  \CLbetaWB + \CLbetaH + \CLbetaV
\end{equation}
In this equation there are the contributions of the wing-body, of the horizontal tail and of the vertical tail.

The contribution of the wing-body consists of three terms: the first one is due to the wing geometric dihedral angle; the second one is due to an aerodynamic phenomenon associated with the location of the fuselage with respect to the wing (high wing or low wing); the third one is due to the geometric wing sweep angle. A closed-form expression for the modelling for $\CLbetaWB$  is given by:
\begin{multline}
\label{eq:dihedralWB}
\CLbetaWB = 57.3 \CLift \Biggl[ \Biggl(\frac{\CLbeta} {\CLift} \Biggr)_{\Lambda_{c/2}} K_{M_\Lambda}  K_{f} + \Biggl(\frac{\CLbeta} {\CLift} \Biggr)_{AR} \Biggr] + \\ + 57.3 \Biggl\{ \GammaW\Biggl[\frac{\CLbeta}{\GammaW}  K_{M_{\Gamma}} + \frac{\textrm \textDelta \CLbeta}{\GammaW}\Biggr] + \Bigl( \textrm \textDelta \CLbeta \Bigr)_{\ZW}+ \epsW \tan\Lambda_{c/4} \Biggl( \frac{\textrm \textDelta \CLbeta}{\epsW \tan \Lambda_{c/4}} \Biggr) \Biggr\}
\end{multline}
In the formula \ref{eq:dihedralWB} there are some semi-empirical coefficients:
\begin{enumerate}
\item \label{RollLiftLamb} $\bigl({\CLbeta}/{\CLift} \bigr)_{\Lambda_{c/2}}$ is the contribution associated with the wing sweep angle;
\item \label{KappaMLamb} $K_{M_\Lambda}$ is a correction factor associated with the Mach number and the wing sweep angle;
\item \label{Kappaf} $K_{f}$ is a correction factor associated with the length of the forward portion of the fuselage;
\item \label{RollLiftAR} $\bigl({\CLbeta}/{\CLift} \bigr)_{AR}$ is the contribution associated with the wing aspect ratio;
\item \label{RollGamma} ${\CLbeta}/{\GammaW}$ is the contribution associated with the wing dihedral angle;
\item \label{KappaMGamma} $K_{M_{\Gamma}}$ is a correction factor associated with the Mach number and the wing dihedral angle;
\item \label{DeltaRollGamma} ${\textrm \textDelta \CLbeta}/{{\GammaW}}$ is a correction factor associated with the size of the fuselage;
\item \label{DeltaRollZ} $\bigl( \textrm \textDelta \CLbeta \bigr)_{\ZW}$ is a correction factor associated with the location of the fuselage with respect to the wing;
\item \label{DeltaRollEps} ${\textrm \textDelta \CLbeta}/{(\epsW \tan \Lambda_{c/4})}$ is a correction factor associated with the twist angle \epsW between the zero-lift lines of the wing sections at the tip and at the root stations.
\end{enumerate}
The terms (\ref{RollLiftLamb}), (\ref{KappaMLamb}), (\ref{Kappaf}), (\ref{RollLiftAR}), (\ref{RollGamma}), (\ref{KappaMGamma}) and (\ref{DeltaRollEps}) are are given by figure~\vref{sweepanglecontribution} to figure~\vref{twistanglecontribution}. The term (\ref{DeltaRollGamma}) is modelled using the relationship:
\begin{equation}
\frac{\textrm \textDelta \CLbeta}{{\GammaW}} = -0.0005 \ARW \biggl( \frac{\dB}{\bW} \biggr)^2
\end{equation}
The factor (\ref{DeltaRollZ}) is modelled using the relationship:
\begin{equation}
\bigl( \textrm \textDelta \CLbeta \bigr)_{\ZW} = 1.2 \sqrt{\ARW} \frac{\ZW}{\bW} \biggl( \frac{2\dB}{\bW} \biggr)
\end{equation}

\begin{figure}[htbp]
\centering
\subfloat[]
	{\includegraphics[width=0.55\textwidth]{Immagini/Capitolo2/4_39-K_Roll_Lam_C2_lam0}} \\
\subfloat[]
	{\includegraphics[width=0.55\textwidth]{Immagini/Capitolo2/4_39-K_Roll_Lam_C2_lam05}} \\
\subfloat[]
	{\includegraphics[width=0.55\textwidth]{Immagini/Capitolo2/4_39-K_Roll_Lam_C2_lam1}}
\caption[Contribution to \CLbetaWB due to $\Lambda_{c/2}$] {Contribution to \CLbetaWB due to wing sweep angle}
\label{sweepanglecontribution}
\end{figure}

\begin{figure}[htbp]
\centering
\includegraphics[width=0.55\textwidth]{Immagini/Capitolo2/4_40-K_M_Lam}
\caption[Compressibility correction factor for \CLbetaWB due to $\Lambda_{c/2}$] {Compressibility correction factor for \CLbetaWB due to wing sweep angle}
\label{compressibilitycorrection}
\end{figure}

\begin{figure}[htbp]
\centering
\includegraphics[width=0.55\textwidth]{Immagini/Capitolo2/4_41-K_Roll_f}
\caption[Fuselage correction factor for \CLbetaWB due to $\Lambda_{c/2}$] {Fuselage correction factor for \CLbetaWB due to wing sweep angle}
\label{fuselagecorrection}
\end{figure}

\begin{figure}[htbp]
\centering
\includegraphics[width=0.55\textwidth]{Immagini/Capitolo2/4_42-K_Roll_AR}
\caption[Correction for \CLbetaWB due to \ARW] {Contribution to \CLbetaWB due to wing aspect ratio}
\label{wingaspectratiocorrection}
\end{figure}

\begin{figure}[htbp]
\centering
\subfloat[]
	{\includegraphics[width=0.55\textwidth]{Immagini/Capitolo2/4_43-K_Roll_Gam_lam0}} 
\\
\subfloat[]
	{\includegraphics[width=0.55\textwidth]{Immagini/Capitolo2/4_43-K_Roll_Gam_lam05}} 
\\
\subfloat[]
	{\includegraphics[width=0.55\textwidth]{Immagini/Capitolo2/4_43-K_Roll_Gam_lam1}}
\caption[Contribution to \CLbetaWB due to \GammaW] {Contribution to \CLbetaWB due to wing dihedral angle}
\label{dihedralanglecontribution}
\end{figure}

\begin{figure}[htbp]
\centering
\includegraphics[width=0.75\textwidth]{Immagini/Capitolo2/4_44-K_Roll_M_Gam}
\caption[Compressibility correction factor for \CLbetaWB due to \GammaW] {Compressibility correction factor for \CLbetaWB due to wing dihedral angle}
\label{compressibilitycorrectiondihedral}
\end{figure}

\begin{figure}[htbp]
\centering
\includegraphics[width=0.75\textwidth]{Immagini/Capitolo2/4_46-K_Roll_Cl_Lam_twist}
\caption[Contribution to \CLbetaWB due to \epsW] {Contribution to \CLbetaWB due to wing twist angle}
\label{twistanglecontribution}
\end{figure}

The horizontal tail can be considered as a wing, operating at a lower dynamic pressure, with a smaller surface and a smaller span. Therefore, a relationship for \CLbetaH is given by:
\begin{equation}
\label{eq:dihedralH}
\CLbetaH = \CLbetaWB \Big|_{\mathrm H} \etaH \frac{\SH}{\SW} \frac{\bH}{\bW}
\end{equation}
where the $\CLbetaWB \big|_{\mathrm H}$ is the previously introduced \CLbetaWB evaluated with the geometric parameters of the horizontal tail. 

The vertical tail contribution to the dihedral effect is expressed by the following formula:
\begin{equation}
\label{eq:dihedralV}
\begin{split}
\CLbetaV & = \CYbetaV \frac{\ZV \cos\alphaB - \XV\sin\alphaB}{\bW} = \\
& = - k_{Y_\mathrm V} \big|\CLalphaV \big| \etaV \biggl( 1 - \frac{\mathrm{d}\sigma}{\mathrm{d}\beta} \biggr) \frac{\SV}{\SW} \frac{\ZV \cos\alphaB - \XV\sin\alphaB}{\bW}
\end{split}
\end{equation}

% --------------------------------------------------------------------------------------------------------------------------------------------
% SOTTOSEZIONE 2.2
% --------------------------------------------------------------------------------------------------------------------------------------------
\subsection{Ailerons deflection effect}
\label{subsec2.2.2}

Ailerons are an asymmetric control surface. According to European conventions, a positive deflection of the ailerons implies a trailing edge down deflection of the right aileron and a trailing edge up deflection of the left aileron. The combined result of these deflections is a negative rolling moment. The mathematical expression of $\CLdeltaa$ is:
\begin{equation}
\label{eq:ailRoll}
\CLdeltaa = \CLdeltaa' \tauail
\end{equation}
where $\CLdeltaa'$ is given by the following relationship:
\begin{equation}
\CLdeltaa' = - \textrm \textDelta (R\hspace{-0.1em} M \hspace{-0.2em} E) \frac{k}{\sqrt{1 - M^2}}
\end{equation}
in which $R \hspace{-0.1em} M \hspace{-0.2em} E$ is given by charts of figures~\vref{rme0} to~\vref{rme1}. The term \tauail is a control surface effectiveness factor, calculated from figure~\vref{tauaileron}. Finally, the term $k$ is equal to: %; in this parameter it is made the assumption that the entire wing section between the aileron inner and outer stations is deflected
\begin{equation}
k = \frac{\CLalphaW \sqrt{1 - M^2}}{2\pi}
\end{equation}

\begin{figure}[H]
\centering
\subfloat[]
	{\includegraphics[width=0.5\textwidth]{Immagini/Capitolo2/4_51-RME_lam0_2}}\\
\subfloat[]
	{\includegraphics[width=0.5\textwidth]{Immagini/Capitolo2/4_51-RME_lam0_4}}\\
\subfloat[]
	{\includegraphics[width=0.5\textwidth]{Immagini/Capitolo2/4_51-RME_lam0_8}}
\caption[Rolling Moment Effectiveness for $\lambdaW = 0$] {Rolling Moment Effectiveness for different geometries of the wing ($\lambdaW = 0$)}
\label{rme0}
\end{figure}

\begin{figure}[H]
\centering
\subfloat[]
	{\includegraphics[width=0.5\textwidth]{Immagini/Capitolo2/4_52-RME_lam05_2}} \\
\subfloat[]
	{\includegraphics[width=0.5\textwidth]{Immagini/Capitolo2/4_52-RME_lam05_4}} \\
\subfloat[]
	{\includegraphics[width=0.5\textwidth]{Immagini/Capitolo2/4_52-RME_lam05_8}}
\caption[Rolling Moment Effectiveness for $\lambdaW = 0.5$] {Rolling Moment Effectiveness for different geometries of the wing ($\lambdaW = 0.5$)}
\label{rme05}
\end{figure}

\begin{figure}[H]
\centering
\subfloat[]
	{\includegraphics[width=0.5\textwidth]{Immagini/Capitolo2/4_53-RME_lam1_2}} \\
\subfloat[]
	{\includegraphics[width=0.5\textwidth]{Immagini/Capitolo2/4_53-RME_lam1_4}} \\
\subfloat[]
	{\includegraphics[width=0.5\textwidth]{Immagini/Capitolo2/4_53-RME_lam1_8}}
\caption[Rolling Moment Effectiveness for $\lambdaW = 1$] {Rolling Moment Effectiveness for different geometries of the wing ($\lambdaW = 1$)}
\label{rme1}
\end{figure}

\begin{figure}[H]
\centering
\includegraphics[width=0.75\textwidth]{Immagini/Capitolo2/4_55-Effectiveness_Aileron}
\caption[Effectiveness of the aileron] {Effectiveness of the aileron \tauail as function of $\overline c_\text a/\overline c_{\text {W (at aileron)}}$}
\label{tauaileron}
\end{figure}

% --------------------------------------------------------------------------------------------------------------------------------------------
% SOTTOSEZIONE 2.3
% --------------------------------------------------------------------------------------------------------------------------------------------
\subsection{Rudder deflection effect}
\label{subsec2.2.3}

The rudder is the control surface, hinged at the tip of the vertical tail, which provides yaw control. Its contribution to the rolling moment originates from the lateral force associated with its deflection through its moment arm with respect to the aircraft centre of gravity. A mathematical relationship for this moment coefficient is:
\begin{equation}
\label{eq:rollRudder}
\CLdeltar = \CYdeltar \frac{\ZR \cos\alphaB - \XR \sin\alphaB}{\bW} = \big|\CLalphaV \big| \etaV \frac{\SV}{\SW} \textrm \textDelta (K_{\textrm r}) \taurud \frac{\ZR \cos\alphaB - \XR \sin\alphaB}{\bW}
\end{equation}

% --------------------------------------------------------------------------------------------------------------------------------------------
% SEZIONE 3
% --------------------------------------------------------------------------------------------------------------------------------------------
\section{Steady-state yawing moment coefficient}
\label{sec2.3}

The steady-state yawing moment can be evaluated by the relationship:
\begin{equation}
\label{eq:YawMoment}
\mathcal{N} = \CN \overline{q} \SW \bW
\end{equation}
where the rolling moment coefficient can be expressed by:
\begin{equation}
\CN = f(\beta,\deltaA,\deltaR)
\end{equation}
The first order approximation for the Taylor expansion gives the following expression for \CN :
\begin{equation}
\CN = \CNzeroYaw + \CNbeta \beta + \CNdeltaa \deltaA + \CNdeltar \deltaR
\end{equation}
in which the term \CNzeroYaw is zero if the aircraft is symmetric with respect to the \emph{XZ} plane.

% --------------------------------------------------------------------------------------------------------------------------------------------
% SOTTOSEZIONE 3.1
% --------------------------------------------------------------------------------------------------------------------------------------------
\subsection{Weathercock effect}
\label{subsec2.3.1}

The aerodynamic coefficient \CNbeta is known as weathercock effect and it can be expressed through its dependencies as shown in the following formula:
\begin{equation}
\label{eq:weathercockeffect}
\CNbeta = \CNbetaW + \CNbetaBody + \CNbetaH + \CNbetaV
\end{equation}

The contribution of the wing and the horizontal tail are negligible for all configurations. 

The fuselage contribution is evaluated using the relationship:
\begin{equation}
\label{eq:weathercockBody}
\CNbetaBody = -57.3 K_{\textrm N} K_{\textrm{Re}_\textrm B} \frac{\SBside}{\SW} \frac{\lB}{\bW}
\end{equation}
where the coefficient $K_{\textrm N}$ is an empirical factor, given by figure~\vref{wingbodyinterface} and related to the geometric coefficients of the axial cross section of the fuselage, whereas the coefficient $K_{\textrm{Re}_\textrm B}$, given by figure~\vref{reynoldscontribution}, is related to the fuselage Reynolds number.

The most significant contribution to \CNbeta is provided by the vertical tail. This contribution is evaluated by:
\begin{equation}
\label{eq:weathercockVertical}
\begin{split}
\CNbetaV & = -\CYbetaV \frac{\ZV \sin\alphaB + \XV \cos\alphaB}{\bW} = \\
& = k_{Y_\textrm V} \big|\CLalphaV \big| \etaV \biggl( 1 - \frac{\mathrm{d}\sigma}{\mathrm{d}\beta} \biggr) \frac{\SV}{\SW} \frac{\ZV \sin\alphaB + \XV \cos\alphaB}{\bW}
\end{split}
\end{equation}

\begin{figure}[htbp]
\centering
\includegraphics[width=\textwidth]{Immagini/Capitolo2/4_68-KN}
\caption[Empirical factor $K_\text N$ for wing-body interface] {Empirical factor $K_\text N$ for wing-body interface}
\label{wingbodyinterface}
\end{figure}

\begin{figure}[htbp]
\centering
\includegraphics[width=0.75\textwidth]{Immagini/Capitolo2/4_69-KReB}
\caption[Effect of Reynolds number on wing-body interface] {Effect of Reynolds number on wing-body interface}
\label{reynoldscontribution}
\end{figure}

% --------------------------------------------------------------------------------------------------------------------------------------------
% SOTTOSEZIONE 3.2
% --------------------------------------------------------------------------------------------------------------------------------------------
\subsection{Ailerons deflection effect}
\label{subsec2.3.2}

The asymmetric deflection of the left and right ailerons also generate small but not negligible drag force leading to a small positive yawing moment. A relationship for modelling \CNdeltaa is given by:
\begin{equation}
\label{eq:yawailerons}
\CNdeltaa = - \textrm \textDelta \bigl(K_{\mathcal{N}_\textrm a} \bigr) \CLift \CLdeltaa
\end{equation}
where \CLift is the aircraft lift coefficient, the term \CLdeltaa is the rolling moment coefficient due to ailerons deflection and $K_{\mathcal{N}_\textrm a}$ is evaluated from graphs of figure~\vref{aileroncontribution}.

\begin{figure}[htbp]
\subfloat[]
	{\includegraphics[width=.5\textwidth]{Immagini/Capitolo2/4_72-KNA_vs_eta_lam025}}
\subfloat[]
	{\includegraphics[width=.5\textwidth]{Immagini/Capitolo2/4_72-KNA_vs_eta_lam05}} \\
\subfloat[]
	{\includegraphics[width=.5\textwidth]{Immagini/Capitolo2/4_72-KNA_vs_eta_lam075}}
\subfloat[]
	{\includegraphics[width=.5\textwidth]{Immagini/Capitolo2/4_72-KNA_vs_eta_lam1}}
\caption[Correlation coefficient for $\mathcal N$ due to $\delta_\text a$] {Correlation coefficient for yawing moment due to deflection of ailerons}
\label{aileroncontribution}
\end{figure}

% --------------------------------------------------------------------------------------------------------------------------------------------
% SOTTOSEZIONE 3.3
% --------------------------------------------------------------------------------------------------------------------------------------------
\subsection{Rudder deflection effect}
\label{subsec2.3.3}

This coefficient is related to the contribution given by the rudder deflection. The mathematical expression is: 
\begin{equation}
\label{eq:yawrudder}
\begin{split}
\CNdeltar & = -\CYdeltar \frac{\ZR\sin\alphaB + \XR \cos\alphaB}{\bW} = \\
& = - \big|\CLalphaV \big| \etaV \frac{\SV}{\SW} \textrm \textDelta (K_{\textrm r}) \taurud \frac{\ZR\sin\alphaB + \XR \cos\alphaB}{\bW}
\end{split}
\end{equation}

% --------------------------------------------------------------------------------------------------------------------------------------------
% SEZIONE 4
% --------------------------------------------------------------------------------------------------------------------------------------------
\newpage
\section{Unsteady-state lateral force coefficient}
\label{sec2.4}

% --------------------------------------------------------------------------------------------------------------------------------------------
% SOTTOSEZIONE 4.1
% --------------------------------------------------------------------------------------------------------------------------------------------
\subsection{Roll rate effect}
\label{subsec2.4.1}

The coefficient \CYprate models the contribution of the lateral force coefficient due to roll rate. The vertical tail is the only component of the aircraft contributing to \CYprate. The mathematical relationship is given by:
\begin{equation}
\label{eq:sideforcerollrate}
\begin{split}
\CYprate \approx \CYprateV & = 2 \CYbetaV \frac{\ZV \cos\alphaB - \XV\sin\alphaB}{\bW} = \\
& = -2 k_{Y_\textrm V} \big|\CLalphaV \big| \etaV \biggl( 1 - \frac{\mathrm{d}\sigma}{\mathrm{d}\beta} \biggr) \frac{\SV}{\SW} \frac{\ZV \cos\alphaB - \XV\sin\alphaB}{\bW}
\end{split}
\end{equation}

% --------------------------------------------------------------------------------------------------------------------------------------------
% SOTTOSEZIONE 4.2
% --------------------------------------------------------------------------------------------------------------------------------------------
\subsection{Yaw rate effect}
\label{subsec2.4.2}

The coefficient \CYrrate models the contribution of the lateral force coefficient due to yaw rate. As for \CYprate, the vertical tail is the only component of the aircraft contributing to \CYrrate. The mathematical relationship is given by:
\begin{equation}
\label{eq:sideforceyawrate}
\begin{split}
\CYrrate \approx \CYrrateV & = - 2 \CYbetaV \frac{\ZV\sin\alphaB + \XV \cos\alphaB}{\bW} = \\
& = 2 k_{Y_\textrm V} \big|\CLalphaV \big| \etaV \biggl( 1 - \frac{\mathrm{d}\sigma}{\mathrm{d}\beta} \biggr) \frac{\SV}{\SW} \frac{\ZV\sin\alphaB + \XV \cos\alphaB}{\bW}
\end{split}
\end{equation}

% --------------------------------------------------------------------------------------------------------------------------------------------
% SEZIONE 5
% --------------------------------------------------------------------------------------------------------------------------------------------
\section{Unsteady-state rolling moment coefficient}
\label{sec2.5}

% --------------------------------------------------------------------------------------------------------------------------------------------
% SOTTOSEZIONE 5.1
% --------------------------------------------------------------------------------------------------------------------------------------------
\subsection{Roll rate effect}
\label{subsec2.5.1}

The coefficient \CLprate models the contribution of the rolling moment coefficient due to roll rate. The mathematical relationship is given by:
\begin{equation}
\label{eq:rollrollrate}
\CLprate = \CLprateWB + \CLprateH + \CLprateV
\end{equation}

The contribution of the fuselage is negligible, so $\CLprateWB \approx \CLprateW$, and the relationship for this coefficient is given by:
\begin{equation}
\label{eq:rollrollrateW}
\CLprateW = RDP \frac{k}{\sqrt{1 - M^2}}
\end{equation}
where \emph{RDP} is the rolling damping parameter, evaluated from graphs of figure~\vref{rollingdampingparameters}.

\begin{figure}[H] 
\centering
\subfloat[]
	{\includegraphics[width=.5\textwidth]{Immagini/Capitolo2/4_80-RDP_lam0}}
\subfloat[]
	{\includegraphics[width=.5\textwidth]{Immagini/Capitolo2/4_80-RDP_lam025}} \\
\subfloat[]
	{\includegraphics[width=.5\textwidth]{Immagini/Capitolo2/4_81-RDP_lam05}}
\subfloat[]
	{\includegraphics[width=.5\textwidth]{Immagini/Capitolo2/4_81-RDP_lam1}}
\caption[Rolling Damping Parameters] {Rolling Damping Parameters for different wing geometry}
\label{rollingdampingparameters}
\end{figure}

A relationship for \CLprateH is given by:
\begin{equation}
\label{eq:rollrollrateH}
\CLprateH = \frac{1}{2} \CLprateW \Big|_{\textrm H} \frac{\SH}{\SW} \biggl( \frac{\bH}{\bW} \biggr)^2
\end{equation}
where the $\CLprateW \big|_{\textrm  H}$ is the previously introduced \CLprateW evaluated with the geometric parameters of the horizontal tail. This coefficient is often negligible due to the low numerical value of the product of $\SH / \SW \cdot (\bH / \bW)^2$.

The contribution of the vertical tail is:
\begin{equation}
\label{eq:rollrollrateV}
\CLprateV = 2 \CYbetaV \biggl( \frac{\ZV}{\bW} \biggr)^2 = - 2  k_{Y_\textrm V}  \big|\CLalphaV \big| \etaV \biggl( 1 - \frac{\mathrm{d}\sigma}{\mathrm{d}\beta} \biggr) \frac{\SV}{\SW} \biggl( \frac{\ZV}{\bW} \biggr)^2
\end{equation}

% --------------------------------------------------------------------------------------------------------------------------------------------
% SOTTOSEZIONE 5.2
% --------------------------------------------------------------------------------------------------------------------------------------------
\subsection{Yaw rate effect}
\label{subsec2.5.2}

The contribution to the rolling moment due to the yaw rate is contained in the coefficient \CLrrate. The modelling for this coefficient starts from the following relationship:
\begin{equation}
\label{eq:rollyawrate}
\CLrrate = \CLrrateWB + \CLrrateH + \CLrrateV
\end{equation}
The fuselage and the horizontal tail do not significantly contribute to this coefficient, so we have $\CLrrate \approx \CLrrateW + \CLrrateV$.

The wing contribution is given by:
\begin{equation}
\label{eq:rollyawrateW}
\CLrrateW = \Biggl( \frac{\CLrrate}{\CLift} \Biggr) \Bigg|_{\CLift = 0} \CLift + \Biggl( \frac{\textrm \textDelta\CLrrate}{\GammaW} \Biggr) \GammaW + \Biggl( \frac{\textrm \textDelta\CLrrate}{\epsW} \Biggr) \epsW
\end{equation}
The coefficient $\Bigl( {\CLrrate}/{\CLift} \Bigr) \Big|_{\CLift = 0}$ is given by:
\begin{equation}
\Biggl( \frac{\CLrrate}{\CLift} \Biggr) \Bigg|_{\CLift = 0} = D \Biggl( \frac{\CLrrate}{\CLift} \Biggr) \Bigg|_{M = 0, \CLift = 0}
\end{equation}
where:
\begin{equation}
D = \frac{1 + \dfrac{\ARW \bigl (1 - B^2 \bigr )}{2B \bigl [\ARW B + 2\cos\LambdaQC \bigr ]} + \dfrac{\ARW B + 2\cos\LambdaQC}{\ARW B + 4\cos\LambdaQC} \dfrac{\tan^2\LambdaQC}{8}}{1 + \dfrac{\ARW + 2\cos\LambdaQC}{\ARW + 4\cos\LambdaQC}\dfrac{\tan^2\LambdaQC}{8}}
\end{equation}
in which:
\begin{equation}
B = \sqrt{1 - M^2 \cos^2\LambdaQC}
\end{equation}
and $\Bigl( {\CLrrate}/{\CLift} \Bigr) \Big|_{M = 0, \CLift = 0}$ is given by figure~\vref{CLR}. In the equation \ref{eq:rollyawrateW} the term ${\textrm \textDelta\CLrrate}/{\GammaW}$ is a factor due to the wing dihedral angle, whereas the term ${\textrm \textDelta\CLrrate}/{\epsW}$ is a factor due to the wing twist angle. The first factor is modelled using the relationship:
\begin{equation}
\frac{\textrm \textDelta\CLrrate}{\GammaW} = \frac{1}{12} \frac{\pi \ARW \sin\LambdaQC}{\ARW + 4\cos\LambdaQC}
\end{equation}
the second one is evaluated using figure~\vref{effectwingtwistonlateralforce}.

The vertical tail contribution into the equation~\ref{eq:rollyawrate} is given by:
\begin{equation}
\label{eq:rollyawrateV}
\begin{split}
\CLrrateV & = - 2 \CYbetaV \frac{\ZV \sin\alphaB + \XV\cos\alphaB}{\bW} \frac{\ZV \cos\alphaB - \XV\sin\alphaB}{\bW} = \\
& = 2 k_{Y_\textrm V} \big|\CLalphaV \big| \etaV \biggl( 1 - \frac{\mathrm{d}\sigma}{\mathrm{d}\beta} \biggr) \frac{\SV}{\SW} \frac{\ZV \sin\alphaB + \XV\cos\alphaB}{\bW} \frac{\ZV \cos\alphaB - \XV\sin\alphaB}{\bW}
\end{split}
\end{equation}

\begin{figure}[p] 
\centering
\includegraphics[width=\textwidth]{Immagini/Capitolo2/4_85-CLR}
\caption[Evaluation of $(C_{\mathcal L_{\mathrm r}}/C_L)_{M=0,C_L=0}$ ] {Evaluation of $\left(\dfrac{C_{\mathcal L_r}}{C_L}\right)_{M=0, C_L=0}$}
\label{CLR}
\end{figure}

\begin{figure}[p] 
\centering
\includegraphics[width=.75\textwidth]{Immagini/Capitolo2/4_87-TwistEffect}
\caption[Effect of $\epsilon_\text W$ on $C_{\mathcal L_\mathrm r}$] {Effect of wing twist on $C_{\mathcal L_r}$.}
\label{effectwingtwistonlateralforce}
\end{figure}

% --------------------------------------------------------------------------------------------------------------------------------------------
% SEZIONE 6
% --------------------------------------------------------------------------------------------------------------------------------------------
\section{Unsteady-state yawing moment coefficient}
\label{sec2.6}

% --------------------------------------------------------------------------------------------------------------------------------------------
% SOTTOSEZIONE 6.1
% --------------------------------------------------------------------------------------------------------------------------------------------
\subsection{Roll rate effect}
\label{subsec2.6.1}

The coefficient \CNprate models the contribution to the yawing moment due to the roll rate. The relationship for this coefficient is:
\begin{equation}
\label{eq:yawrollrate}
\CNprate = \CNprateWB + \CNprateH + \CNprateV
\end{equation}
Because the fuselage and the horizontal tail do not significantly contribute to this coefficient, we have that $\CNprate \approx \CNprateW + \CNprateV$.

A relationship for the wing contribution is given by:
\begin{equation}
\label{eq:yawrollrateW}
\CNprateW = -\CLprateW \tan \alphaB + \CLprate \tan \alphaB + \Biggl( \frac{\CNprate}{\CLift} \Biggr) \Bigg|_{\CLift = 0} \CLift + \Biggl( \frac{\textrm \textDelta\CNprate}{\epsW} \Biggr) \epsW
\end{equation}
where \CLprateW and \CLprate are described in \ref{eq:rollrollrateW} and \ref{eq:rollrollrate}, respectively. The coefficient $\Bigl( {\CNprate}/{\CLift} \Bigr) \Big|_{\CLift = 0}$ is given by: 
\begin{equation}
\Biggl( \frac{\CNprate}{\CLift} \Biggr) \Bigg|_{\CLift = 0} = C \Biggl( \frac{\CNprate}{\CLift} \Biggr) \Bigg|_{M = 0, \CLift = 0}
\end{equation}
where:
\begin{equation}
C = \frac{\ARW + 4\cos\LambdaQC}{\ARW B + 4\cos\LambdaQC} \frac{\ARW B + \frac{1}{2} \bigl [\ARW B + 4\cos\LambdaQC \bigr ] \tan^2\LambdaQC}{\ARW + \frac{1}{2} \bigl [\ARW + 4\cos\LambdaQC \bigr ] \tan^2\LambdaQC}
\end{equation}
in which:
\begin{equation}
B = \sqrt{1 - M^2 \cos^2\LambdaQC}
\end{equation}
and $\Bigl ( \CNprate / \CLift \Bigr ) \Big|_{M = 0, \CLift = 0}$ is modelled using the relationship:
\begin{equation}
\Biggl( \frac{\CNprate}{\CLift} \Biggr) \Bigg|_{M = 0, \CLift = 0} = - \frac{1}{6} \frac{\ARW + 6 \bigl (\ARW + \cos\LambdaQC \bigr ) \Bigl[ (\xcgadim - \xacadim)\frac{\tan\LambdaQC}{\ARW} + \frac{\tan^2\LambdaQC}{12} \Bigr]}{\ARW + \cos\LambdaQC}
\end{equation}
The term ${\textrm \textDelta\CNprate}/{\epsW}$ is associated with the wing twist angle and is taken from figure~\vref{effectwingtwist}.

The vertical tail contribution into the equation \ref{eq:yawrollrate} is given by:
\begin{equation}
\label{eq:rollyawrateV}
\begin{split}
\CLrrateV & = -2 \CYbetaV \frac{\ZV\sin\alphaB + \XV \cos\alphaB}{\bW} \frac{\ZV \cos\alphaB - \XV\sin\alphaB - \ZV}{\bW} = \\
& = 2 k_{Y_\textrm V} \big|\CLalphaV \big| \etaV \biggl( 1 - \frac{\mathrm{d}\sigma}{\mathrm{d}\beta} \biggr) \frac{\SV}{\SW} \frac{\ZV\sin\alphaB + \XV \cos\alphaB}{\bW} \frac{\ZV \cos\alphaB - \XV\sin\alphaB - \ZV}{\bW}
\end{split}
\end{equation}

\begin{figure}[htbp]
\centering
\includegraphics[width=.75\textwidth]{Immagini/Capitolo2/4_83-WingTwist_Unsteady}
\caption[Effect of wing twist on $C_{\mathcal N_\mathrm p}$] {Effect of wing twist on $C_{\mathcal N_p}$}
\label{effectwingtwist}
\end{figure}

% --------------------------------------------------------------------------------------------------------------------------------------------
% SOTTOSEZIONE 6.2
% --------------------------------------------------------------------------------------------------------------------------------------------
\subsection{Yaw rate effect}
\label{subsec2.6.2}

The coefficient \CNrrate models the contribution to the yawing moment due to the yaw rate. A relationship for this coefficient is:
\begin{equation}
\label{eq:yawyawrate}
\CNrrate = \CNrrateWB + \CNrrateH + \CNrrateV
\end{equation}
The fuselage and the horizontal tail do not significantly contribute to this coefficient; therefore, we have that $\CNrrate \approx \CNrrateW + \CNrrateV$.

The wing contribution is evaluated by the relationship:
\begin{equation}
\label{eq:yawyawrateW}
\CNrrateW = \Biggl( \frac{\CNrrate}{\CLift^2} \Biggr) \CLift^2 + \Biggl( \frac{\CNrrate}{\CDo} \Biggr) \CDo
\end{equation}
in which the terms ${\CNrrate}/{\CLift^2}$ and ${\CNrrate}/{\CDo}$ are evaluated from figure~\vref{CNR} and figure~\vref{CND} respectively.

The vertical tail contribution is given by:
\begin{equation}
\label{eq:yawyawrateV}
\begin{split}
\CNrrateV & = 2 \CYbetaV \biggl ( \frac{\ZV \sin\alphaB + \XV\cos\alphaB}{\bW} \biggr )^2 = \\
& = -2  k_{Y_\textrm V} \big|\CLalphaV \big| \etaV \biggl( 1 - \frac{\mathrm{d}\sigma}{\mathrm{d}\beta} \biggr) \frac{\SV}{\SW} \biggl ( \frac{\ZV \sin\alphaB + \XV\cos\alphaB}{\bW} \biggr )^2
\end{split}
\end{equation}

\begin{figure}[htbp]
\subfloat[]
	{\includegraphics[width=\textwidth]{Immagini/Capitolo2/4_89-CNL_0}} \\
\subfloat[]
	{\includegraphics[width=\textwidth]{Immagini/Capitolo2/4_89-CNL_02}} \\
\subfloat[]
	{\includegraphics[width=\textwidth]{Immagini/Capitolo2/4_89-CNL_04}} \\
\caption[Evaluation of $C_{\mathcal N_\mathrm r}/C^2_L$ ] {Effect of lift on $C_{\mathcal N_r}$}
\label{CNR}
\end{figure}

\begin{figure}[htbp]
\centering
\subfloat[]
	{\includegraphics[width=.55\textwidth]{Immagini/Capitolo2/4_90-CND_0}} \\
\subfloat[]
	{\includegraphics[width=.55\textwidth]{Immagini/Capitolo2/4_90-CND_02}} \\
\subfloat[]
	{\includegraphics[width=.55\textwidth]{Immagini/Capitolo2/4_90-CND_04}}
\caption[Effect of $C_{D_0}$ on $C_{\mathcal N_\mathrm r}$] {Effect of parasite drag on $C_{\mathcal N_r}$}
\label{CND}
\end{figure}
\chapter{Example of application on a regional turboprop}
\label{ch3}

\begin{figure}[htbp]
\centering
\includegraphics[width=\textwidth]{Immagini/Capitolo3/IRON}
\caption{Three views of the aircraft}
\end{figure}

\newpage

\begin{table}[H]
\centering
\caption{Aircraft data}
\begin{tabular}{ccc}
\toprule
Variable & SI units & USCS units \\
\midrule
\CDo & $\SI{0.03035}{}$ & ---\\
\xcgadim & $\SI{0.1}{}$ & ---\\
SSM & $\SI{-0.080}{}$ & ---\\
\bottomrule
\end{tabular}
\end{table}

\begin{table}[H]
\centering
\caption{Flight conditions}
\begin{tabular}{ccc}
\toprule
Variable & SI units & USCS units \\
\midrule
\alphaB & $\SI{0}{rad}$ & $\SI{0}{deg}$ \\
\CLift & $\SI{0.650}{}$ & --- \\
$M$ & $\SI{0.640}{}$ & ---\\
\bottomrule
\end{tabular}
\end{table}

\begin{table}[H]
\centering
\caption{Wing related parameters}
\begin{tabular}{ccc}
\toprule
Variable & SI units & USCS units \\
\midrule
\SW & $\SI{98.60}{m^2}$ & $\SI{1061.32}{ft^2}$ \\
\bW & $\SI{34.34}{m}$ & $\SI{112.66}{ft}$ \\
\ARW & $\SI{11.961}{}$ & --- \\
\lambdaW & $\SI{0.383}{}$ & --- \\
$\Lambda_{c/4, \text W}$ & $\SI{0.10159}{rad}$ & $\SI{5.821}{deg}$ \\
$\Lambda_{c/2, \text W}$ & $\SI{0.02755}{rad}$ & $\SI{1.578}{deg}$ \\
\GammaW & $\SI{0.09599}{rad}$ & $\SI{5.5}{deg}$ \\
\epsW & $\SI{-0.03491}{rad}$ & $\SI{-2}{deg}$ \\
\CLalphaW & $\SI{6.342}{rad^{-1}}$ & $\SI{0.11068}{deg^{-1}}$ \\
\ZW & $\SI{1.33}{m}$ & $\SI{4.36}{ft}$ \\
\etaailin & $\SI{0.78}{}$ & --- \\
\etaailout & $\SI{0.95}{}$ & --- \\
$\overline c_\text{a}/\overline c_{\text {W (at aileron)}}$ & $\SI{0.32}{}$ & --- \\
\tauail & $\SI{0.530}{}$ & --- \\
\bottomrule
\end{tabular}
\end{table}

\begin{table}[H]
\centering
\caption{Fuselage related parameters}
\begin{tabular}{ccc}
\toprule
Variable & SI units & USCS units \\
\midrule
$S_{\mathrm P \rightarrow \mathrm V}$ & $\SI{6.77}{m^2}$ & $\SI{72.85}{ft^2}$ \\
\SBside & $\SI{108.46}{m^2}$ & $\SI{1167.45}{ft^2}$ \\
$Re_\textrm B$ & 2.35E8 & --- \\
$d_{\text W,\text B}$ & $\SI{3.54}{m}$ & $\SI{11.60}{ft}$ \\
$d_{\text W,\text H}$ & $\SI{2.46}{m}$ & $\SI{8.07}{ft}$ \\
$d$ & $\SI{3.56}{m}$ & $\SI{11.69}{ft}$ \\
$l_\text B$ & $\SI{38.04}{m}$ & $\SI{124.80}{ft}$ \\
$w_\text{max}$ & $\SI{3.51}{m}$ & $\SI{11.52}{ft}$ \\
$Z_\text{max}$ & $\SI{3.56}{m}$ & $\SI{11.69}{ft}$ \\
$Z_1$ & $\SI{3.56}{m}$ & $\SI{11.69}{ft}$ \\
$Z_2$ & $\SI{3.46}{m}$ & $\SI{11.34}{ft}$ \\
$r_1$ & $\SI{2.60}{m}$ & $\SI{8.53}{ft}$ \\
\bottomrule
\end{tabular}
\end{table}

\begin{table}[H]
\centering
\caption{Horizontal tail related parameters}
\begin{tabular}{ccc}
\toprule
Variable & SI units & USCS units \\
\midrule
\SH & $\SI{39.91}{m^2}$ & $\SI{429.59}{ft^2}$ \\
\bH & $\SI{13.04}{m}$ & $\SI{42.78}{ft}$ \\
\ARH & $\SI{4.261}{}$ & --- \\
\lamH & $\SI{0.624}{}$ & --- \\
$\Lambda_{c/4, \text H}$ & $\SI{0.07485}{rad}$ & $\SI{4.286}{deg}$ \\
$\Lambda_{c/2, \text H}$ & $\SI{-0.02633}{rad}$ & $\SI{-1.509}{deg}$ \\
\GammaH & $\SI{0.26180}{rad}$ & $\SI{15}{deg}$ \\
\epsH & $\SI{0}{rad}$ & $\SI{0}{deg}$ \\
\CLalphaH & $\SI{4.261}{rad^{-1}}$ & $\SI{0.07437}{deg^{-1}}$ \\
\etaH & $\SI{0.8}{}$ & ---\\
\bottomrule
\end{tabular}
\end{table}

\begin{table}[H]
\centering
\caption{Vertical tail related parameters}
\begin{tabular}{ccc}
\toprule
Variable & SI units & USCS units \\
\midrule
\SV & $\SI{24.45}{m^2}$ & $\SI{263.18}{ft^2}$ \\
\bV & $\SI{5.78}{m}$ & $\SI{18.96}{ft}$ \\
\lamV & $\SI{0.640}{}$ & --- \\
\XV & $\SI{13.17}{m}$ & $\SI{43.21}{ft}$ \\
\ZV & $\SI{1.52}{m}$ & $\SI{4.99}{ft}$ \\
\CLalphaV & $\SI{2.079}{rad^{-1}}$ & $\SI{0.03629}{deg^{-1}}$ \\
\etaV & $\SI{0.9}{}$ & ---\\
\etarudin & $\SI{0}{}$ & --- \\
\etarudout & $\SI{0.9}{}$ & --- \\
$\overline c_\text{r}/\overline c_{\text {V}}$ & $\SI{0.35}{}$ & --- \\
\taurud & $\SI{0.562}{}$ & --- \\
\XR & $\SI{15.21}{m}$ & $\SI{49.90}{ft}$ \\
\ZR & $\SI{4.12}{m}$ & $\SI{13.52}{ft}$ \\
\bottomrule
\end{tabular}
\end{table}

\begin{table}[H]
\centering
\caption{Outcome parameters from lateral-directional calculations -- Steady coefficients}
\begin{tabular}{ccc}
\toprule
Variable & SI units & USCS units \\
\midrule
\CYzero & $\SI{0}{}$ & ---\\
\CYbetaW & $\SI{-0.032}{rad^{-1}}$ & $\SI{-0.00056}{deg^{-1}}$\\
\CYbetaB & $\SI{-0.154}{rad^{-1}}$ & $\SI{-0.00269}{deg^{-1}}$\\
\CYbetaH & $\SI{-0.056}{rad^{-1}}$ & $\SI{-0.00098}{deg^{-1}}$\\
\CYbetaV & $\SI{-0.533}{rad^{-1}}$ & $\SI{-0.00930}{deg^{-1}}$\\
\CYbeta & $\SI{-0.774}{rad^{-1}}$ & $\SI{-0.01351}{deg^{-1}}$\\
\CYdeltaa & $\SI{0}{rad^{-1}}$ & $\SI{0}{deg^{-1}}$ \\
\CYdeltar & $\SI{0.251}{rad^{-1}}$ & $\SI{0.00438}{deg^{-1}}$\\
\CLzeroRoll & $\SI{0}{}$ & ---\\
\CLbetaWB & $\SI{-0.056}{rad^{-1}}$ & $\SI{-0.00098}{deg^{-1}}$\\
\CLbetaH & $\SI{-0.024}{rad^{-1}}$ & $\SI{-0.00042}{deg^{-1}}$\\
\CLbetaV & $\SI{-0.024}{rad^{-1}}$ & $\SI{-0.00042}{deg^{-1}}$\\
\CLbeta & $\SI{-0.104}{rad^{-1}}$ & $\SI{-0.00182}{deg^{-1}}$\\
\CLdeltaa & $\SI{-0.052}{rad^{-1}}$ & $\SI{-0.00091}{deg^{-1}}$\\
\CLdeltar & $\SI{0.030}{rad^{-1}}$ & $\SI{0.00052}{deg^{-1}}$\\
\CNzeroYaw & $\SI{0}{}$ & ---\\
\CNbetaBody & $\SI{-0.167}{rad^{-1}}$ & $\SI{-0.00292}{deg^{-1}}$\\
\CNbetaV & $\SI{0.205}{rad^{-1}}$ & $\SI{0.00357}{deg^{-1}}$\\
\CNbeta & $\SI{0.038}{rad^{-1}}$ & $\SI{0.00065}{deg^{-1}}$\\
\CNdeltaa & $\SI{0}{rad^{-1}}$ & $\SI{0}{deg^{-1}}$\\
\CNdeltar & $\SI{-0.111}{rad^{-1}}$ & $\SI{-0.00194}{deg^{-1}}$\\
\bottomrule
\end{tabular}
\end{table}

\begin{table}[H]
\centering
\caption{Outcome parameters from lateral-directional calculations -- Unsteady coefficients}
\begin{tabular}{ccc}
\toprule
Variable & SI units & USCS units \\
\midrule
\CYprate & $\SI{-0.047}{rad^{-1}}$ & $\SI{-0.00082}{deg^{-1}}$\\
\CYrrate & $\SI{0.409}{rad^{-1}}$ & $\SI{0.00714}{deg^{-1}}$\\
\CLprateWB & $\SI{-0.522}{rad^{-1}}$ & $\SI{-0.00911}{deg^{-1}}$\\
\CLprateH & $\SI{-0.009}{rad^{-1}}$ & $\SI{-0.00016}{deg^{-1}}$\\
\CLprateV & $\SI{-0.002}{rad^{-1}}$ & $\SI{-0.00005}{deg^{-1}}$\\
\CLprate & $\SI{-0.533}{rad^{-1}}$ & $\SI{-0.00932}{deg^{-1}}$\\
\CLrrateW & $\SI{0.175}{rad^{-1}}$ & $\SI{0.00305}{deg^{-1}}$\\
\CLrrateV & $\SI{0.018}{rad^{-1}}$ & $\SI{0.00031}{deg^{-1}}$\\
\CLrrate & $\SI{0.193}{rad^{-1}}$ & $\SI{0.00336}{deg^{-1}}$\\
\CNprateW & $\SI{-0.095}{rad^{-1}}$ & $\SI{-0.00166}{deg^{-1}}$\\
\CNprateV & $\SI{0}{rad^{-1}}$ & $\SI{0}{deg^{-1}}$\\
\CNprate & $\SI{-0.095}{rad^{-1}}$ & $\SI{-0.00166}{deg^{-1}}$\\
\CNrrateW & $\SI{-0.017}{rad^{-1}}$ & $\SI{-0.00029}{deg^{-1}}$\\
\CNrrateV & $\SI{0.084}{rad^{-1}}$ & $\SI{0.00146}{deg^{-1}}$\\
\CNrrate & $\SI{0.067}{rad^{-1}}$ & $\SI{0.00117}{deg^{-1}}$\\
\bottomrule
\end{tabular}
\end{table} 

% -----------------------------------------------------------------------------------------
%                                                                 APPENDICI
% -----------------------------------------------------------------------------------------

\appendix

\chapter{HDF dataset and database reader creation}
\label{ch:hdigitalizer}
\markboth{HDF dataset and database reader creation}{}
In a tool for preliminary design phase of an aircraft, it's very important to have aviable database. It's possible to create database starting from graphics using external software. In this appendix will be explained the step required in order to digitalize the graphics, create an HDF dataset and set up the database-reader class in Jpad.

\section{Chart Digitization}
The first step required for create a dataset is to digitalize a chart. Often data is found presented in reports and references as functional X-Y type scatter or line plots. In order to use this data, it must somehow be digitized. This is made with an external software, such as {\itshape Plot Digitizer}. Plot Digitizer is a Java program used to digitize scanned plots of functional data. This program will allow you to take a scanned image of a plot (in GIF, JPEG, or PNG format) and quickly digitize values off the plot just by clicking the mouse on each data point.\cite{plotdigitizer}

\begin{figure}[H]
\centering
{\includegraphics[height=7.9cm]{Immagini/digitize.png}} 
\caption{Chart digitization using Plot Digitizer.}
\label{angles}
\end{figure} 


In order to digitize a chart, first of all it's necessary to calibrate the axis. Plot Digitizer works with both linear and logarithmic axis scales. After it's possible to digitize a curve simply click o it. The values obtained can then be saved to a tex file or .csv file.


\section{Creation of an HDF file with Matlab}

Obtained the .csv file from digitization is necessary to create the HDF file. First of all it's necessary to import the file with couple of coordinates as matrix. After saving the imported files as .mat file, Matlab code comes in play to manage these data and to generate the digitalized curves and the HDF dataset.  The code interpolates curves points with cubic splines in order to have more points to plot for each curve.

\bigskip
\lstset{language=Matlab}
\begin{lstlisting}[frame=rbl,caption={{\footnotesize MATLAB script for creating the HDF Database}},label= [style=\bfseries]{Listing}]
clc; close all; clear all;

%% Import data
DeltaAlphaCLmax_vs_LambdaLE_dy1p2 = importdata('DeltaAlphaCLmax_vs_LambdaLE_dy1p2.mat');
DeltaAlphaCLmax_vs_LambdaLE_dy2p0 = importdata('DeltaAlphaCLmax_vs_LambdaLE_dy2p0.mat');
DeltaAlphaCLmax_vs_LambdaLE_dy3p0 = importdata('DeltaAlphaCLmax_vs_LambdaLE_dy3p0.mat');
DeltaAlphaCLmax_vs_LambdaLE_dy4p0 = importdata('DeltaAlphaCLmax_vs_LambdaLE_dy4p0.mat');

nPoints = 30;
lambdaLEVector_deg = transpose(linspace(0, 40, nPoints));

%% dy/c = 1.2
smoothingParameter = 0.999999;
DAlphaVsLambdaLESplineStatic_Dy1p2 = csaps( ...
    DeltaAlphaCLmax_vs_LambdaLE_dy1p2(:,1), ...
    DeltaAlphaCLmax_vs_LambdaLE_dy1p2(:,2), ...
    smoothingParameter ...
    );

DAlphaVsLambdaLEStatic_Dy1p2 = ppval( ...
    DAlphaVsLambdaLESplineStatic_Dy1p2, ...
    lambdaLEVector_deg ...
    );

%% dy/c = 2.0

smoothingParameter = 0.999999; 
DAlphaVsLambdaLESplineStatic_Dy2p0 = csaps( ...
    DeltaAlphaCLmax_vs_LambdaLE_dy2p0(:,1), ...
    DeltaAlphaCLmax_vs_LambdaLE_dy2p0(:,2), ...
    smoothingParameter ...
    );

DAlphaVsLambdaLEStatic_Dy2p0 = ppval( ...
    DAlphaVsLambdaLESplineStatic_Dy2p0, ...
    lambdaLEVector_deg ...
    );

%% dy/c = 3.0

smoothingParameter =0.999999;
DAlphaVsLambdaLESplineStatic_Dy3p0 = csaps( ...
    DeltaAlphaCLmax_vs_LambdaLE_dy3p0(:,1), ...
    DeltaAlphaCLmax_vs_LambdaLE_dy3p0(:,2), ...
    smoothingParameter ...
    );

DAlphaVsLambdaLEStatic_Dy3p0 = ppval( ...
    DAlphaVsLambdaLESplineStatic_Dy3p0, ...
    lambdaLEVector_deg ...
    );

%% dy/c = 4.0

smoothingParameter = 0.999999; 
DAlphaVsLambdaLESplineStatic_Dy4p0 = csaps( ...
    DeltaAlphaCLmax_vs_LambdaLE_dy4p0(:,1), ...
    DeltaAlphaCLmax_vs_LambdaLE_dy4p0(:,2), ...
    smoothingParameter ...
    );

DAlphaVsLambdaLEStatic_Dy4p0 = ppval( ...
    DAlphaVsLambdaLESplineStatic_Dy4p0, ...
    lambdaLEVector_deg ...
    );


%% Plots
figure(1)
plot ( ...
    lambdaLEVector_deg, DAlphaVsLambdaLEStatic_Dy1p2, '-*b' ... , ...
 );
 hold on
 
 plot ( ...
    lambdaLEVector_deg, DAlphaVsLambdaLEStatic_Dy2p0, '-b' ... , ...
 );

hold on

plot ( ...
    lambdaLEVector_deg, DAlphaVsLambdaLEStatic_Dy3p0, '*b' ... , ...
 );

hold on

plot ( ...
    lambdaLEVector_deg, DAlphaVsLambdaLEStatic_Dy4p0, 'b' ... , ...
 );

 xlabel('\Lambda_{le} (deg)'); ylabel('\Delta\alpha_{C_{L,max}}');
 title('Angle of attack increment for wing maximum lift in subsonic flight');
  legend('\Delta y/c = 1.2', '\Delta y/c = 2.0', '\Delta y/c = 3.0','\Delta y/c = 4.0');
 axis([0 50 0 9]);
 grid on;
 

 
%% preparing output to HDF

% dy/c
dyVector = [ ...
    1.2;2.0;3.0;4.0 ...
    ];

%columns --> curves
myData = [ ...
    DAlphaVsLambdaLEStatic_Dy1p2,...
        DAlphaVsLambdaLEStatic_Dy2p0, ... % -> 2
        DAlphaVsLambdaLEStatic_Dy3p0, ... % -> 3
        DAlphaVsLambdaLEStatic_Dy4p0];    % -> 4

hdfFileName = 'DAlphaVsLambdaLEVsDy.h5';

if ( exist(hdfFileName, 'file') )
    fprintf('file %s exists, deleting and creating a new one\n', hdfFileName);
    delete(hdfFileName)
else
    fprintf('Creating new file %s\n', hdfFileName);
end

% Dataset: data
h5create(hdfFileName, '/DAlphaVsLambdaLEVsDy/data', size(myData'));
h5write(hdfFileName, '/DAlphaVsLambdaLEVsDy/data', myData');

% Dataset: var_0
h5create(hdfFileName, '/DAlphaVsLambdaLEVsDy/var_0', size(dyVector'));
h5write(hdfFileName, '/DAlphaVsLambdaLEVsDy/var_0', dyVector');

% Dataset: var_1
h5create(hdfFileName, '/DAlphaVsLambdaLEVsDy/var_1', size(lambdaLEVector_deg'));
h5write(hdfFileName, '/DAlphaVsLambdaLEVsDy/var_1', lambdaLEVector_deg');
\end{lstlisting}

\noindent \\ \\ 
This script plot the graph after digitization. In this way it's possible to compare the initial graph and the digitized one.


 \begin{figure}[H]
\centering
{\includegraphics[height=6cm]{Immagini/digitize2.png}} 
\label{angles}
\end{figure} 





%-------------------------------------------------------------------------------------------

\thispagestyle{empty} % per non inserire una pagina nella numerazione

% -----------------------------------------------------------------------------------------
%                                                               B I B L I O G R A F I A
% -----------------------------------------------------------------------------------------
\thispagestyle{empty}
\printbibliography[heading=bibintoc]

\end{document}
