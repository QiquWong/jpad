% !TeX program = PdfLaTeX
% !TeX root = ../Main.tex

\renewcommand{\abstractname}{Abstract}
\begin{abstract}
The purpose of this Thesis work is to implement a specific Java routine in order to assess the 
minimum unstick speed of a complete aircraft. This has been done implementing new features in JPAD 
(Java toolchain Program for Aircraft Design), a Java-based framework conceived as a fast and
efficient tool as a support in the preliminary design phases of an aircraft and during its
optimization process.\\
The JPAD development originates in the Departement of Industrial Engineering of University of Naples “Federico II”, where is still in progress.
JPAD is a framework equiped with a graphic user interface (GUI) that, at present, is capable to perform a multidisciplinary
analysis of an aircraft whose data can be entered by the user, with an XML, or loaded into memory. \\

The structure of this thesis work is articulated in 3 chapter and 1 appendix.  \\
First of all, the JPAD framework is presented.\\
The second chapter concerns a theoretical background and, starting from a brief introduction of take off speeds, it will move on the explanation of the ground effect and its implementation. This chapter ends with the presentation of Minimun Unstick Speed (VMU). \\
The third chapter regards a case study performed on a regional turboprop of 130 pax, concerning aerodynamic characteristics, high lift devices and ground effect both on wing lift curve and on downwash angle.\\
The development of Java functionalities, in JPAD, has made considering further developments thus to allow the possibility to introduce new methods. \\
At the end, in appendix A, it has been shown how to digitalizes a chart, create an HDF dataset and set up the database-reader class in JPAD.
\end{abstract}