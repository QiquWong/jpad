% !TeX program = PdfLaTeX
% !TeX root = ../Main.tex

\renewcommand{\abstractname}{Sommario}
\begin{abstract}
Lo scopo di questo lavoro di tesi è implementare una specifica routine Java per valutare la
VMU di un aereo completo. Questo è stato fatto implementando nuove ca\-rat\-te\-ri\-sti\-che in JPAD
(Java toolchain Program for Aircraft Design), un framework scritto in Java concepito come un veloce ed
efficiente strumento a supporto nelle fasi preliminari di progettazione di un aereo e durante il suo
processo di ottimizzazione. \\
Lo sviluppo di JPAD ha origine nel Dipartimento di Ingegneria Industriale dell'Università di Napoli "Federico II", dove è ancora in corso.
JPAD è un software dotato di un'interfaccia grafica (GUI) e, al momento, è in grado di eseguire un'analisi multidisciplinare
di un aeromobile i cui dati possono essere inseriti dall'utente, con un file XML, o caricati in memoria. \\

La struttura di questo lavoro di tesi è articolata in 3 capitoli e 1 appendice. \\
Prima di tutto, viene presentata la struttura di JPAD. \\
Il secondo capitolo riguarda un background teorico il quale, partendo da una breve introduzione della velocità di decollo, si sposterà sulla spiegazione dell'effetto suolo e sulla sua implementazione. Questo capitolo termina con la presentazione della VMU. \\
Il terzo capitolo riguarda un caso studio eseguito su un turboelica regionale di 130 passeggeri, riguardante le caratteristiche aerodinamiche, i dispositivi di ipersostentazione e l'effetto suolo sia sulla curva di portanza dell'ala che sull'angolo di downwash. \\
Lo sviluppo delle nuove funzionalità Java, in JPAD, è stato fatto prendendo in con\-si\-de\-ra\-zio\-ne la possibilità di ulteriori sviluppi per consentire l'introduzione di nuovi metodi. \\
Alla fine, nell'appendice A, viene mostrato come digitalizzare un grafico, creare un set di dati in formato .h5 e impostare la classe che legge di database in JPAD.
\end{abstract}