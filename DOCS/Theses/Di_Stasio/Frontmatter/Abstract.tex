\begin{abstract}
The main goal of this thesis work is to give an overview on the development of a Java-based library, to be integrated in JPAD (a software dedicated to preliminary aircraft design), whose purpose is dealing with the automatic production of complete or partial aircraft CAD models. So generated 3D models can be imported and visualized in a JavaFX window, to be implemented in the JPAD GUI, or imported into a CFD tool, in order to perform high-fidelity numerical analyses.

A brief introduction on the JPAD software architecture is followed by a detailed description of the JPADCAD package, which contains all the classes and the utility functions designed to generate CAD models. Afterwards, the methodologies and the functions through which the aircraft components (fuselage, wing, horizontal and vertical tail, canard, fairings, control surfaces) are translated from XML data file to 3D models are described. The last chapter deals with the classes and utilities designed and used in order to import JPAD generated CAD files into CFD analysis software CD-adapco STARCCM+, in order to perform workflows involving high-fidelity numerical solutions.
\end{abstract}

\newpage\thispagestyle{empty}
\selectlanguage{italian}
\begin{abstract}
Lo scopo del presente lavoro di tesi è di fornire una panoramica circa lo sviluppo, in linguaggio Java, di una libreria, da integrare al software di preliminary design JPAD, per la generazione automatica di modelli CAD di velivoli completi o parti di essi. I modelli così generati possono essere visualizzati in una finestra JavaFX da integrare nella GUI di JPAD, oppure importati in un tool di CFD, per analisi numeriche ad alto livello di accuratezza.

Ad una breve introduzione circa JPAD e la sua architettura, segue una descrizione accurata del pacchetto JPADCAD, in cui attualmente risiedono le classi e le utility atte alla generazione di modelli CAD. Successivamente vengono descritte le metodologie e le funzioni mediante cui le componenti dell'aereo (fusoliera, ala, piano orizzontale e verticale di coda, canard, fairing, superfici di controllo), a partire da file di input di tipo XML, vengono tradotte in modelli 3D. Conclude il lavoro un excursus sulle classi e le utility create per l'importazione dei file CAD generati in JPAD nel software di analisi CFD CD-adapco STARCCM+, al fine di creare dei cicli automatizzati di lavoro.
\end{abstract}
\selectlanguage{english}