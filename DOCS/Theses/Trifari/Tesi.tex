%******************************************************************************************
%
% AUTHOR:           Agostino De Marco
% DESCRIPTION:      This is "Tesi.tex", the master source file of a thesis manuscript
%                   When compiled with pdflatex creates a PDF document "Tesi.pdf".
% USAGE:            Use the following command line:
%                   $ pfdlatex Tesi
%
%******************************************************************************************

%------------------------------------------------------------------------------------------
% Meta-commands for the TeXworks editor
%
% !TEX encoding = UTF-8
% !TEX program = pdflatex
%------------------------------------------------------------------------------------------

%------------------------------------------------------------------------------------------
% INITIAL DEFINITIONS
%------------------------------------------------------------------------------------------

%******************************************************************************************
%
% AUTHOR:           Agostino De Marco
% DESCRIPTION:      This is "_init.tex", containing plain TeX commands.
%                   Goes in document before \documentclass
%
%******************************************************************************************
%
%------------------------------------------------------------------------------------------
% Meta-commands for the TeXworks editor
%
% !TeX root = ./Tesi.tex
% !TEX encoding = UTF-8
% !TEX program = pdflatex
%------------------------------------------------------------------------------------------

%%% We need a non standard MakeIndex style, because many entries in
%%% the index have names containing `@'.  Compilation of the index
%%% requires the call `makeindex Tesi'

\begin{filecontents*}{Tesi.ist}
actual '='
\end{filecontents*}

%%% Some constant strings

\def\myAuthorsList{Mario~Di~Stasio}
\def\myAuthorListAbbreviated{M.~Di~Stasio}

\def\myDocName{Aircraft Solid Modeling with JPAD and Automatic Workflows for High-Fidelity Numerical Simulations}
\def\myDocSubTitle{A Master Thesis in Aerospace Engineering}

\def\myFooterLeft{\myAuthorsList\  -- \myDocName}
\def\myFooterRight{\myDocSubTitle}

\def\myDocVersion{ver.~2018.a}% <================== %?[ADJUST-HERE] EDIT THIS STRING AS NEEDED
\def\myDocDate{August~2018}% not used

%% EOF
% plain TeX initial definitions in ./_init.tex

%------------------------------------------------------------------------------------------
% LaTeX STARTING DECLARATION
%------------------------------------------------------------------------------------------

\documentclass[12pt,twoside]{book}

%------------------------------------------------------------------------------------------
% PREAMBLE
%------------------------------------------------------------------------------------------

%******************************************************************************************
%
% AUTHOR:           Agostino De Marco
% DESCRIPTION:      This is "_preamble.tex" file, included by the master source file
%                   "Tesi.tex".
%
%******************************************************************************************

%------------------------------------------------------------------------------------------
% Meta-commands for the TeXworks editor
%
% !TeX root = ./Tesi.tex
% !TEX encoding = UTF-8
% !TEX program = pdflatex

%------------------------------------------------------------------------------------------
% page layout with geometry (see layout_geometry_test.tex)
\usepackage[
    % driver=none,% needed with the crop package
%    papersize = a4paper,
	paperwidth = 210mm,
	paperheight = 297mm,
    % papersize={7in,10in},
    %papersize={8in,10in},% CreateSpace industry standard
    %papersize={8.5in,11in},% CreateSpace industry standard
    % papersize={258mm,201mm}, % Metric Large Crown Quarto, see: http://www.prepressure.com/library/paper-sizes
    hmargin={1.75cm,0.90cm},vmargin={1cm,1.8cm},marginparwidth=2.6cm,
    includehead,
%    includefoot,
%    footskip=1.6cm,
    includemp
    ]{geometry}
    
\usepackage{needspace}
\usepackage{pdflscape}

% \usepackage[cam,a4,center]{crop}

%------------------------------------------------------------------------------------------
% Language and encoding

\usepackage[utf8]{inputenc}
\usepackage[english,italian]{babel}
\usepackage{csquotes}% Recommended with biblatex combined with italian (as main language)

%------------------------------------------------------------------------------------------
% LaTeX3 stuff

\usepackage{expl3}

%------------------------------------------------------------------------------------------
% manage date and time

\usepackage[short,nodayofweek,12hr]{datetime}

%-------------------------------------------------------------------------------------
% Mathematics and related stuff

\usepackage{amsmath}
%\usepackage%
%   %[intlimits]
%   {amsmath}
%\usepackage{amsfonts}
%\usepackage{amsbsy}
%\usepackage{fixmath}
\usepackage{mathtools}
\usepackage{cancel}

%-------------------------------------------------------------------------------------
% Fonts and related stuff

\usepackage[T1]{fontenc}
\usepackage[final]{microtype}

\usepackage{relsize}% font relative sizing commands

%\usepackage{newtxtext}% replaces the txfonts package;
%\usepackage[varg]{newtxmath}

% MathTime Professional 2

%\renewcommand{\rmdefault}{ptm}      % set Times as the default text font
%\usepackage{libertine}% \copyright symbol unavailable
\usepackage{newtxtext}

% The following loads mtpro and defines some common MTPro options [2, 4]
\usepackage[subscriptcorrection,slantedGreek,nofontinfo,%
            mtpcal,%
            mtphbi %% mtpbbi
            ]{mtpro2}

% Options for blackboard bold fonts [2.9]:
%   mtphrb - holey roman bold        mtpbb - blackboard bold
%   mtphrd - holey roman bold dark   mtpbbd - blackboard bold dark
%   mtphbi - holey bold italic       mtpbbi - blackboard bold italic

% Options for alternate character sets [2.6, 2.7]:
%   mtpcal - assigns Math Script to the math alphabets \mathcal and \mathbcal,
%   overwriting the default math calligraphic typeface
%   mtpccal - assigns Math Curly to the math alphabets \mathcal and \mathbcal,
%   overwriting the default math calligraphic typeface
%   mtpscr - assigns Math Script to the new math alphabets \mathscr and \mathbscr,
%       leaving \mathcal unchanged
%   mtpfrak - assigns Math Fraktur to a new math alphabet \mathfrak

% Options for AMS symbols
%   amssymbols - makes the mtpro2 AMS symbols available

% Optionally load the following package to use heavy symbols in place of bold symbols
%\usepackage{bm}

% font for Aspect Ratio

\usepackage%
   [TM]% NOTE: needs version 2012 _and_ fonts/tfm/public/aspectratio/*.tfm
   {ar}% original package by Claudio Beccari (based on Computer Modern Roman)
%
%\usepackage{my-ar-tm}% AR ligature based on Times font design
%\newcommand{\AR}{{\ARtm}}
%\newcommand{\ARb}{{\ARtmb}}

\usepackage{textcomp}% additional glyphs
\usepackage{pifont} % for dingbats
\usepackage{marvosym} % for \Keyboard, etc

% Define text sans-serif font
% --> package txfonts uses Helvetica
\usepackage%
   [scaled=0.90]{helvet}% scale Helvetica as appropriate
   %[scaled=0.95]{berasans}% alternative to Helvetica

% Define text mono-spaced font
% --> package txfonts uses TX mono (monospace typewriter font)
\usepackage%
   [scaled=0.825]% 0.865 , 0.84
   {beramono}% set bera as mono-spaced font family

%\usepackage{enumitem}
%\setlist{noitemsep} % or \setlist{nosep} to leave space around whole list
\renewcommand{\labelitemi}{\small$\bullet$}
\newcommand{\smallspacing}{0.1em}
\newcommand{\medspacing}{0.15em}
	
%------------------------------------------------------------------------------------------
% FRONTESPIZIO by Enrico Gregorio
\usepackage{frontespizio}

%------------------------------------------------------------------------------------------
% Theorem-like stuff

%
\usepackage[amsmath,hyperref]{ntheorem}% NB: deve essere caricato DOPO babel

% definisce uno stile di teorema ad header vuoto
% richiede ntheorem.sty
\makeatletter
  \newtheoremstyle{mytheoremstyle}
    {\item[]}%
    {\item[]}
\makeatother
\theoremstyle{mytheoremstyle}% <------  seleziona lo stile appena definito
%\theoremstyle{margin}       %          ... altri stili predefiniti in ntheorem
%\theoremstyle{nonumberplain}

% nuovo theorem-like environment
\newtheorem{myExampleT}{}[chapter]

% myExample label & format
\newcommand\myExampleLabel{Esempio}
\newcommand\myExampleLabelFormat{%
    \textbf{\upshape\hspace{3pt}\myExampleLabel\ \themyExampleT}%
}
\newcommand\myExampleMarkPencilKeyboardMouse{%
    \bf\ding{46}\ \raisebox{-2pt}[0pt][0pt]{\relsize{4}\Keyboard\hspace{2pt}\ComputerMouse}%
}
\newcommand\myExampleMarkKeyboardMouse{%
    \raisebox{-2pt}[0pt][0pt]{\relsize{4}\Keyboard\hspace{2pt}\ComputerMouse}%
}
\newcommand\myExampleEndMark{%
    {\ding{118}}% \ding{111}
}

% environment per gli Esempi
\newenvironment{myExample}[1][\ding{46}]{%
    %###########################################################
    % eseguito all'inizio dell'environment
    %###########################################################
    \begin{myExampleT}%
    \adjustbox{set height=1.1\baselineskip,set depth=0.5\baselineskip,valign=m,%
        center=\linewidth,bgcolor=mylightblue!50}{%
        \adjustbox{left=0.6\linewidth}{%
            \myExampleLabelFormat%
        }%
        \adjustbox{right=0.4\linewidth}{%
            #1\ %
        }%
    }%
    \medskip
    \par\upshape% testo normale
}{%
    %###########################################################
    % eseguito alla fine dell'environment
    %###########################################################
    \end{myExampleT}%
    \smallskip
    \adjustbox{set height=0.55\baselineskip,set depth=0.11\baselineskip,valign=m,%
        center=\linewidth,bgcolor=mylightblue!50}{\relsize{-2}\myExampleEndMark}
    \medskip
}%

% nuovo theorem-like environment
%\newtheorem{myExampleTX}{}[chapter]

% environment per gli Esempi esteso (X)
%\makeatletter
\newenvironment{myExampleX}[2]{% accetta due argomenti: #1 titolo, #2 simbolo nell'header
    %###########################################################
    % ciò che viene eseguito all'inizio dell'environment
    %###########################################################
    \begin{myExampleT}
    %\stepcounter{myexamplecounter}% incrementa il contatore
    \par\vskip 8pt% a capo e skip verticale
    \noindent
    % striscia colorata
    \makebox[0em][l]{%
        {\color{mylightblue!50}\rule[-8pt]{\textwidth}{1.8em}}%
    }%
    % scritta
    \makebox[1.0\textwidth][c]{%
        %{\myExampleLabelFormat}%
        \textbf{\upshape\hspace{3pt}\myExampleLabel\ \themyExampleT:}\quad\textit{#1}\hfill\textbf{#2\ }%
    }%
    \vskip 8pt% recupera gli 8pt di \rule[-8pt]
    \smallskip%
    %\noindent%
    \upshape% testo normale
}{%
    %###########################################################
    % ciò che viene eseguito alla fine dell'environment
    %###########################################################
    %\addvspace{3.2ex plus 0.8ex minus 0.2ex}%
    \end{myExampleT}
    %\par\vskip 2pt
    \noindent
    \smallskip
    % striscia colorata
    \makebox[0em][l]{%
        {\color{mylightblue!50}\rule%
                            [0pt]%[-9pt]%
                            {\textwidth}{0.6em}}% 1.8em
    }%
    % simbolo di chiusura
    \makebox[1.0\textwidth][c]{%
        %{\raisebox{7pt}{\myExampleEndMark}}
        {\raisebox{0.6pt}{\relsize{-2}\myExampleEndMark}}
    }%
    %\par\vskip-2pt%
    %\noindent
    \medskip
}%

%.................................................. Matlab Tip
% nuovo theorem-like environment
\newtheorem{myMatlabTipT}{}[chapter]

% myExample label & format
\newcommand\myMatlabTipLabel{Matlab {\itshape tip}}
\newcommand\myMatlabTipLabelFormat{%
    \textbf{\upshape\hspace{3pt}\myMatlabTipLabel\ \themyMatlabTipT}%
}

\newcommand\myMatlabTipKeyboardMouse{%
    \raisebox{-2pt}[0pt][0pt]{\relsize{4}\Keyboard\hspace{2pt}\ComputerMouse}%
}

\newcommand\myMatlabTipEndMark{%
    {\ding{118}}% \ding{111}%
}

% environment per gli Esempi
%\makeatletter
\newenvironment{myMatlabTip}[1][\myMatlabTipKeyboardMouse]{%  \ding{46}
    %###########################################################
    % ciò che viene eseguito all'inizio dell'environment
    %###########################################################
    \begin{myMatlabTipT}%
    \adjustbox{set height=1.1\baselineskip,set depth=0.5\baselineskip,valign=m,%
        center=\linewidth,bgcolor=mylightblue!50}{%
        \adjustbox{left=0.6\linewidth}{%
            \myMatlabTipLabelFormat%
        }%
        \adjustbox{right=0.4\linewidth}{%
            #1\ %
        }%
    }%
%
    \par\upshape% testo normale
    \smallskip%
}{%
    %###########################################################
    % ciò che viene eseguito alla fine dell'environment
    %###########################################################
    %\addvspace{3.2ex plus 0.8ex minus 0.2ex}%
    \end{myMatlabTipT}
%    %\par\vskip 2pt
    \par\noindent%\smallskip
    \adjustbox{set height=0.55\baselineskip,set depth=0.11\baselineskip,valign=m,%
        center=\linewidth,bgcolor=mylightblue!50}{\relsize{-2}\myExampleEndMark}
    \medskip
}%


% environment per gli Esercizi esteso (X)
%\makeatletter

% nuovo theorem-like environment
\newtheorem{myExerciseT}{}[chapter]

\newcommand\myExerciseEndMark{%
    {\ding{118}}% \ding{111}
}

% myExercise label & format
\newcommand\myExerciseLabel{Esercizio}
\newcommand\myExerciseLabelFormat{%
    %\textbf{\bf\ding{46}\ \myExampleLabel\ \themyExampleT}
    \textbf{\upshape\hspace{3pt}\myExampleLabel\ \themyExampleT}%
    \hfill\textbf{\bf\ding{46}\ \raisebox{-2pt}[0pt][0pt]{\relsize{4}\Keyboard\hspace{2pt}\ComputerMouse}\ }%
}

\newenvironment{myExerciseX}[2]{% accetta due argomenti: #1 titolo, #2 simbolo nell'header
    %###########################################################
    % ciò che viene eseguito all'inizio dell'environment
    %###########################################################
    \begin{myExerciseT}
    %\stepcounter{myexamplecounter}% incrementa il contatore
    \par\vskip 8pt% a capo e skip verticale
    \noindent
    % striscia colorata
    \makebox[0em][l]{%
        {\color{mylightblue!50}\rule[-8pt]{\textwidth}{1.8em}}%
    }%
    % scritta
    \makebox[1.0\textwidth][c]{%
        %{\myExampleLabelFormat}%
        \textbf{\upshape\hspace{3pt}\myExerciseLabel\ \themyExerciseT:}\quad\textit{#1}\hfill\textbf{#2\ }%
    }%
    \vskip 8pt% recupera gli 8pt di \rule[-8pt]
    \smallskip%
    %\noindent%
    \upshape% testo normale
}{%
    %###########################################################
    % ciò che viene eseguito alla fine dell'environment
    %###########################################################
    %\addvspace{3.2ex plus 0.8ex minus 0.2ex}%
    \end{myExerciseT}
    %\par\vskip 2pt
    \noindent
    \smallskip
    % striscia colorata
    \makebox[0em][l]{%
        {\color{mylightblue!50}\rule%
                            [0pt]%[-9pt]%
                            {\textwidth}{0.6em}}% 1.8em
    }%
    % simbolo di chiusura
    \makebox[1.0\textwidth][c]{%
        %{\raisebox{7pt}{\myExampleEndMark}}
        {\raisebox{0.6pt}{\relsize{-2}\myExerciseEndMark}}
    }%
    %\par\vskip-2pt%
    %\noindent
    \medskip
}%
%\makeatother

%%%\usepackage{amsthm}
%%%
%%% http://www.guit.sssup.it/phpbb/viewtopic.php?t=3523&highlight=ntheorem
%%% http://www.tug.org/tutorials/tugindia/chap17-prn.pdf

%%%\makeatletter
%%%\def\@thm#1#2#3{%
%%%  \ifhmode\unskip\unskip\par\fi
%%%  \normalfont
%%%  \trivlist
%%%  \let\thmheadnl\relax
%%%  \let\thm@swap\@gobble
%%%  \thm@notefont{\fontseries\bfdefault\upshape}%
%%%  \thm@headpunct{.}% add period after heading
%%%  \thm@headsep 5\p@ plus\p@ minus\p@\relax
%%%  \thm@space@setup
%%%  #1% style overrides
%%%  \@topsep \thm@preskip               % used by thm head
%%%  \@topsepadd \thm@postskip           % used by \@endparenv
%%%  \def\@tempa{#2}\ifx\@empty\@tempa
%%%    \def\@tempa{\@oparg{\@begintheorem{#3}{}}[]}%
%%%  \else
%%%    \refstepcounter{#2}%
%%%    \def\@tempa{\@oparg{\@begintheorem{#3}{\csname the#2\endcsname}}[]}%
%%%  \fi
%%%  \@tempa
%%%}
%%%\makeatother
%%%
%%%\newtheorem{myExampleSimple}{Esempio}[chapter]

\usepackage[hyphens]{url}

%------------------------------------------------------------------------------------------
% Management of floating material

\usepackage{float}
   % I got this somewhere on the net:
   \renewcommand{\textfraction}{0.15}
   \renewcommand{\topfraction}{0.85}
   \renewcommand{\bottomfraction}{0.65}
   \renewcommand{\floatpagefraction}{0.60}

% \usepackage[maxfloats=19]{morefloats}[2012/01/28]% v1.0f
\usepackage{morefloats}
   
\usepackage{placeins}

\usepackage{caption}
\captionsetup[figure]{
    %format=hang,%
    labelsep=space,%
    font=small,labelfont={bf,sf,small},%
    %justification=raggedright,%
    singlelinecheck=true%,%
    %width=1.1\textwidth%%
}
\captionsetup[table]{
    %format=hang,%
    labelsep=space,%
    font=small,labelfont={bf,sf,small},%bf,%
    justification=raggedright,%
    singlelinecheck=true,%
    %width=0.8\textwidth,%
    belowskip=0.5em%
}
\captionsetup[lstlisting]{
    %format=hang,%
    labelsep=space,%
    font=small,labelfont={bf,sf,small},%
    %justification=raggedright,%
    singlelinecheck=true%,%
    %width=1.1\textwidth%%
}

\captionsetup[ctable]{
    %format=hang,%
    labelsep=space,%
    font=small,labelfont={sf,bf,small},% bf
    %justification=raggedright,%
    singlelinecheck=true,%
    width=0.8\textwidth,%
    belowskip=0.5em%
}

%% see here: http://tex.stackexchange.com/questions/13625/subcaption-vs-subfig
%\usepackage{subfig}
\usepackage{subcaption}% multiple figures in one floating object 
% subcaption CANNOT BE USED IN COOPOEARTION WITH SUBFIG
\usepackage[innercaption,wide]{sidecap}

%------------------------------------------------------------------------------------------
% Graphics and related stuff
\usepackage{graphicx}
\graphicspath{{images/}}
\usepackage[export]{adjustbox}
\usepackage{tikz}
\usepackage{pgfplots}
\pgfplotsset{compat=1.10}
\usepackage{wrapfig}

%\usepackage{ifluatex}
%\ifluatex
%\usepackage{pdftexcmds}
%\makeatletter
%\let\pdfstrcmp\pdf@strcmp
%\let\pdffilemoddate\pdf@filemoddate
%\makeatother
%\fi

%% svg might be incompatible with subcaption
%\usepackage{svg}
%\setsvg{inkscape={"C:/Program Files (x86)/inkscape-0.48.5/inkscape.exe"= -z -C}}
%\setsvg{convert={"C:/Program Files/ImageMagick-6.9.0-Q16/convert.exe" -density 300}}
%\setsvg{pdftops={"C:/Program Files/xpdfbin-win-3.04/bin64/pdftops.exe" -eps}}

%\newcommand{\executeiffilenewer}[3]{%
%	\ifnum\pdfstrcmp%
%	{\pdffilemoddate{#1}}%
%	{\pdffilemoddate{#2}}%
%	>0%
%	{\immediate\write18{#3}}%
%	\fi%
%}
%\newcommand{\includesvg}[2][]{%
%	\executeiffilenewer{#1#2.svg}{#1#2.pdf}%
%	{inkscape.exe -z -D --file=#1#2.svg --export-pdf=#1#2.pdf --export-latex}%
%	\subimport{#1}{#2.pdf_tex}%
%}

%\usepackage{transparent}
\newlength\figureheight
\newlength\figurewidth
\usepackage{relsize}
\tikzset{
	master/.style={
		execute at end picture={
			\coordinate (lower right) at (current bounding box.south east);
			\coordinate (upper left) at (current bounding box.north west);
		}
	},
	slave/.style={
		execute at end picture={
			\pgfresetboundingbox
			\path (upper left) rectangle (lower right);
		}
	}
}

\usepackage{xcolor}
\usepackage{color}

\definecolor{black}{rgb}{0, 0, 0}
\definecolor{pblue}{rgb}{0.13,0.13,1}
\definecolor{pgreen}{rgb}{0,0.5,0}
\definecolor{pred}{rgb}{0.9,0,0}
\definecolor{pgrey}{rgb}{0.46,0.45,0.48}
\definecolor{mydarkgreen}{rgb}{0.03,0.47,0.03}
\definecolor{mydarkblue}{rgb}{0.07,0.08,0.4}
\definecolor{mylightblue}{rgb}{.8, .8, 1}
\definecolor{mylightgray}{rgb}{0.95,0.95,0.95}
\definecolor{mymidgray}{rgb}{0.85,0.85,0.85}
\definecolor{mydarkgray}{rgb}{0.35,0.35,0.35}
\definecolor{myblue}{rgb}{.4,.4,1}
\definecolor{colKeys}{rgb}{0,0,1}
\definecolor{colIdentifier}{rgb}{0,0,0}
\definecolor{colComments}{rgb}{0,0.5,1}
\definecolor{colString}{rgb}{0.6,0.1,0.1}

%\newcommand{\executeiffilenewer}[3]{%
%	\ifnum\pdfstrcmp%
%	{\pdffilemoddate{#1}}%
%	{\pdffilemoddate{#2}}%
%	>0%
%	{\immediate\write18{#3}}%
%	\fi%
%}
%
%\newcommand{\includesvg}[2][]{%
%	\executeiffilenewer{#1#2.svg}{#1#2.pdf}%
%	{inkscape-0.48.5.exe -z -D --file=#1#2.svg --export-pdf=#1#2.pdf --export-latex}%
%	\subimport{#1}{#2.pdf_tex}%
%}

%------------------------------------------------------------------------------------------
% Units of measure according to the SI standard

\usepackage{siunitx}
\sisetup{
  fixdp,dp=3,
  load=derived,
  unitsep=thin,
  valuesep=thin,
  decimalsymbol=comma,
  % digitsep=thin,
  % group-separator={},% ver. 2
  % sepfour=false,
  % round-mode = places,
  % round-precision = 3
}
%******************************************************************************************
%
% AUTHOR:           Agostino De Marco
% DESCRIPTION:      This is "_units.tex", an auxiliary LaTeX source file.
%                   Here we collect all customizations related to package siunitx
%                   (see _preamble.tex).
%
%******************************************************************************************

%------------------------------------------------------------------------------------------
% Meta-commands for the TeXworks editor
%
% !TeX root = ./Libro_MS.tex
% !TEX encoding = UTF-8
% !TEX program = pdflatex
%------------------------------------------------------------------------------------------

%-----------------------------------------------------------------------------------------------------
% siunitx ver 2.5 (July 2013 and later)
\usepackage%
  % [version-1-compatibility]%
  %[load-configurations = version-2]%
  {siunitx}
\sisetup{
  %load=derived,
  mode=math,
  inter-unit-product = {\,}, % unitsep=thin,
  number-unit-product = {\,}, % valuesep=thin,
  output-decimal-marker = {,}, % decimalsymbol=comma,% for European convention
  round-mode = places,
  exponent-product = {\cdot}, % expproduct = cdot,
  % digitsep=thin,
  group-separator = {},% ver. 2
  group-four-digits = false % sepfour=false
}

\DeclareSIUnit\degree{\ensuremath{^\circ}}

\DeclareSIUnit\deg{deg}

\DeclareSIUnit\inch{in}
\DeclareSIUnit\foot{ft}
\DeclareSIUnit\feet{ft}
\DeclareSIUnit\lb{lb}
\DeclareSIUnit\lbm{lb\ensuremath{_\mathrm{m}}}
\DeclareSIUnit\lbf{lb\ensuremath{_\mathrm{f}}}
\DeclareSIUnit\slug{slug}
\DeclareSIUnit\rankine{\ensuremath{^\circ}R}
\DeclareSIUnit\fahrenheit{\ensuremath{^\circ}F}
\DeclareSIUnit\kilopond{kg\ensuremath{_\mathrm{f}}}
\DeclareSIUnit\kgf{kg\ensuremath{_\mathrm{f}}}
\DeclareSIUnit\knots{kts}

% load settings from ./_units.tex

%------------------------------------------------------------------------------------------
% Dummy text, provisional stuff

\usepackage{lipsum}
\usepackage{blindtext}

\usepackage[%
   backgroundcolor=orange!20,bordercolor=orange,
   shadow,
   textsize=footnotesize,
   colorinlistoftodos
   % ,disable % use this to disable the notes
   ]{todonotes}

\newcounter{todocounter}

\newcommand{\TODO}[2][]{%
   % initials of the author (optional) + note in the margin
   \refstepcounter{todocounter}%
   {%
      \setstretch{0.85}% line spacing
      \todo[backgroundcolor={orange!20},bordercolor=orange,size=\relsize{-1}]{%
         \sffamily%
         \thetodocounter.~\textbf{[\uppercase{#1}]:} #2%
      }%
   }}

\newcommand{\TODOInline}[2][]{%
   % initials of the author (optional) + note in the margin
   \refstepcounter{todocounter}%
   {%
      \setstretch{0.85}% line spacing
      \todo[inline,backgroundcolor={orange!20},bordercolor=orange,size=\relsize{-1}]{%
         \sffamily%
         \thetodocounter.~\textbf{[\uppercase{#1}]:}~#2%
      }%
   }}

%%% Examples of usage:
%%% \TODO[ADM]{Continue this section taking material from Stevend \& Lewis book.}
%%% \TODOInline[JSB]{Remove this sentence.}

\newcommand{\TODOInsertRef}[1]{\todo[color=green!40]{#1}}
\newcommand{\TODOExplainInDetail}[1]{\todo[color=red!40]{#1}}
\newcommand{\TODORewrite}[1]{\todo[color=yellow!40]{#1}}

%------------------------------------------------------------------------------------------
% Miscellaneous stuff

\usepackage{xspace}
\usepackage{calc}
\usepackage{ragged2e}
\usepackage[defblank]{paralist}
\usepackage{algorithmic}
\usepackage{multicol}
\usepackage{appendix}

% smart cross-referencing
\usepackage[italian]{varioref}
% Si veda "Breve guida ai pacchetti più usati" di Enrico Gregorio
% http://profs.sci.univr.it/~gregorio/breveguida.pdf
\makeatletter
\vref@addto\extrasitalian{\def\reftextfaraway#1{a pagina~\pageref{#1}}}
\makeatother


\usepackage{dcolumn}

\usepackage{ctable}
\usepackage{booktabs,colortbl}
\usepackage{tabularx}
\usepackage{longtable}

%------------------------------------------------------------------------------------------
% Table of contents per chapter

\usepackage[english,tight]{minitoc}
\AtBeginDocument{%
    \addto\captionsitalian{\mtcselectlanguage{italian}}
}
%\mtcsetpagenumbers{secttoc}{off}
%\nostcpagenumbers

%------------------------------------------------------------------------------------------
% Advanced header and footer management

\usepackage[
            %sf,
            %bf,
            %compact,
            %topmarks,
            calcwidth,%
            pagestyles% loads titleps 
            ]{titlesec}

%\renewpagestyle{empty}[\small\sffamily]{
\renewpagestyle{empty}[\small]{
  \sethead{}{}{}
  \setfoot{}{}{}
}

%\renewpagestyle{plain}[\small\sffamily]{
\renewpagestyle{plain}[\small]{
  %\footrule
  \setfoot{}{\usepage}{}}

% needs \usepackage{calc} above
\widenhead*{0pt}{\marginparsep + \marginparwidth} % symmetrically
\renewpagestyle{plain}{}
%::::::::::::::::::::::::::::::::::::::::::::::::::::::::::::::::::::::::::::::::::::::::::
\newpagestyle{myBookPageStyle}[\bfseries\sffamily\footnotesize]{
    \headrule
    %\myColoredHeadrule% TO DO: got to define a colored line
    \sethead%
        [{\makebox[\marginparwidth+\marginparsep][l]{\thepage}%
            \ifnum\value{chapter}>0%
                {\chaptername}\ \thechapter\ \ {\sffamily\mdseries\chaptertitle}%
            \else {\sffamily\mdseries\chaptertitle}%
            \fi%
        }][][]
        {}{}{%
            \raggedleft%
            \ifnum\value{chapter}<1% e.g. in TOC chapter=0
                {\sffamily\mdseries\chaptertitle}%
            \else%
                \thesection\ \ {\sffamily\mdseries\sectiontitle}%
            \fi%
            {\makebox[\marginparwidth+\marginparsep][r]{\thepage}}%
        }
    % NOTE: this footer might become empty
    \footrule
    \setfoot%
        [][][\color{mydarkblue}\sf\scriptsize \myFooterLeft]% left-hand footer
        {\color{mydarkblue}\sf\scriptsize \myFooterRight}{}{}% right-hand footer
}
%::::::::::::::::::::::::::::::::::::::::::::::::::::::::::::::::::::::::::::::::::::::::::
\newpagestyle{myIndexPageStyle}[\bfseries\sffamily\footnotesize]{
    \headrule
    \sethead%
        [{\makebox[\marginparwidth+\marginparsep][l]{\thepage}%
            {\sffamily\mdseries Index}%
        }][][]
        {}{}{\raggedleft\thesection\ \ {\sffamily\mdseries\sectiontitle}{\makebox[\marginparwidth+\marginparsep][r]{\thepage}}}
    % NOTE: this footer might become empty
    %\footrule
    \setfoot%
        [][][\color{mydarkblue}\sf\scriptsize \myFooterLeft]% left-hand footer
        {\color{mydarkblue}\sf\scriptsize \myFooterRight}{}{}% right-hand footer
}
%::::::::::::::::::::::::::::::::::::::::::::::::::::::::::::::::::::::::::::::::::::::::::
\newpagestyle{myAppendixPageStyle}[\bfseries\sffamily\footnotesize]{
    \headrule
    \sethead%
        [{\makebox[\marginparwidth+\marginparsep][l]{\thepage}%
                {\chaptername}\ \thechapter\ \ {\sffamily\mdseries\chaptertitle}%
        }][][]
        {}{}{\raggedleft\thesection\ \ {\sffamily\mdseries\sectiontitle}{\makebox[\marginparwidth+\marginparsep][r]{\thepage}}}
    % NOTE: this footer might become empty
    %\footrule
    \setfoot%
        [][][\color{mydarkblue}\sf\scriptsize \myFooterLeft]% left-hand footer
        {\color{mydarkblue}\sf\scriptsize \myFooterRight}{}{}% right-hand footer
}
%::::::::::::::::::::::::::::::::::::::::::::::::::::::::::::::::::::::::::::::::::::::::::
\newpagestyle{myBibliographyPageStyle}[\bfseries\sffamily\footnotesize]{
    \headrule
    \sethead%
        [{\makebox[\marginparwidth+\marginparsep][l]{\thepage}%
            {\sffamily\mdseries Bibliography}%
        }][][]
        {}{}{\raggedleft{\sffamily\mdseries Bibliography}{\makebox[\marginparwidth+\marginparsep][r]{\thepage}}}
    % NOTE: this footer might become empty
    %\footrule
    \setfoot%
        [][][\color{mydarkblue}\sf\scriptsize \myFooterLeft]% left-hand footer
        {\color{mydarkblue}\sf\scriptsize \myFooterRight}{}{}% right-hand footer
}
%::::::::::::::::::::::::::::::::::::::::::::::::::::::::::::::::::::::::::::::::::::::::::
\newpagestyle{myGlossaryPageStyle}[\bfseries\sffamily\footnotesize]{
    \headrule
    \sethead%
        [{\makebox[\marginparwidth+\marginparsep][l]{\thepage}%
            {\sffamily\mdseries Glossary}%
        }][][]
        {}{}{\raggedleft\sffamily\mdseries Glossary\makebox[\marginparwidth+\marginparsep][r]{\thepage}}
    % NOTE: this footer might become empty
    %\footrule
    \setfoot%
        [][][\color{mydarkblue}\sf\scriptsize \myFooterLeft]% left-hand footer
        {\color{mydarkblue}\sf\scriptsize \myFooterRight}{}{}% right-hand footer
}
%::::::::::::::::::::::::::::::::::::::::::::::::::::::::::::::::::::::::::::::::::::::::::
\newpagestyle{myAcronymsPageStyle}[\bfseries\sffamily\footnotesize]{
    \headrule
    \sethead%
        [{\makebox[\marginparwidth+\marginparsep][l]{\thepage}%
            {\sffamily\mdseries Acronyms}%
        }][][]
        {}{}{\raggedleft\sffamily\mdseries Acronyms\makebox[\marginparwidth+\marginparsep][r]{\thepage}}
    % NOTE: this footer might become empty
    %\footrule
    \setfoot%
        [][][\color{mydarkblue}\sf\scriptsize \myFooterLeft]% left-hand footer
        {\color{mydarkblue}\sf\scriptsize \myFooterRight}{}{}% right-hand footer
}
%::::::::::::::::::::::::::::::::::::::::::::::::::::::::::::::::::::::::::::::::::::::::::
\newpagestyle{myListOfSymbolsPageStyle}[\bfseries\sffamily\footnotesize]{
    \headrule
    \sethead%
        [{\makebox[\marginparwidth+\marginparsep][l]{\thepage}%
            {\sffamily\mdseries List of symbols}%
        }][][]
        {}{}{\raggedleft\sffamily\mdseries List of symbols\makebox[\marginparwidth+\marginparsep][r]{\thepage}}
    % NOTE: this footer might become empty
    %\footrule
    \setfoot%
        [][][\color{mydarkblue}\sf\scriptsize \myFooterLeft]% left-hand footer
        {\color{mydarkblue}\sf\scriptsize \myFooterRight}{}{}% right-hand footer
}
%::::::::::::::::::::::::::::::::::::::::::::::::::::::::::::::::::::::::::::::::::::::::::
%::::::::::::::::::::::::::::::::::::::::::::::::::::::::::::::::::::::::::::::::::::::::::
%
% Set the initial style
%
\pagestyle{myBookPageStyle}

%% chapter head style via titlesec
\newcommand{\myChapterHeadingColorMix}{%
  %blue!65!black
  blue!0!black%
}
\newcommand{\myChapterHeadingColor}{\color{\myChapterHeadingColorMix}}

\titleformat{\chapter}[display]
    {\bfseries\Large}
    {%
        \myChapterHeadingColor% \color{blue!65!black}% color
        \filleft%
        %\Huge\chaptertitlename\ \thechapter%
        \minsizebox{!}{24pt}{\chaptertitlename}% needs package adjustbox
        \lapbox[0pt]{\width}{%
            \minsizebox{!}{40pt}{%
                %\ \fbox{\thechapter}
                \ \colorbox{\myChapterHeadingColorMix}{\color{white}\thechapter}% needs xcolor
            }%
        }% needs package adjustbox
    }
    {4ex}
    {{\color{gray!65!gray}\titlerule}
        \huge\bfseries\scshape
        \vspace{2ex}%
        \filright}
    [\vspace{2ex}%
        {\color{gray!65!gray}\titlerule}]

%------------------------------------------------------------------------------------------
% Advanced page margin management

\usepackage[strict]{changepage}% as of 2009
\usepackage{fullwidth}

%------------------------------------------------------------------------------------------
% Advanced graphics package

\usepackage{tikz}
\usetikzlibrary{calc}
\usetikzlibrary{positioning,trees,arrows}
\usetikzlibrary{fpu}

\usepackage{pgfplotstable}

% splittable boxes
\usepackage%
  [framemethod=tikz]%
  {mdframed}% texdoc.net/pkg/mdframed

\newcounter{myInsight}[chapter]

\global\mdfdefinestyle{InsightStyleDefault}{%
% Variant I
%    outerl inewidth=3pt,% works with tikz method
%    topline=true,%
%    linecolor=mydarkblue,%blue!20,%
%    innermargin=-13pt,
%    outermargin=-13pt,
%        %% WARNING: multicol is not supported in mdframed, otherwise you'd take a full span
%        % \dimexpr 2pt+3pt-13pt-\marginparsep-\marginparwidth\relax,
%    innerrightmargin=13pt,innerleftmargin=13pt,%
%    frametitlerule=true,frametitlerulecolor=mydarkblue,%blue!20,%
%    frametitlebackgroundcolor=mydarkblue,%blue!20,%
%    frametitlerulewidth=2pt%
%
% Variant II
    outerlinewidth=0pt,% 3pt,% works with tikz method
    middlelinewidth=0em,middlelinecolor=gray!60,
    hidealllines=true,
    topline=false,% true,%
    innermargin=-13pt,
    outermargin=-13pt,
        %% WARNING: multicol is not supported in mdframed, otherwise you'd take a full span
        % \dimexpr 2pt+3pt-13pt-\marginparsep-\marginparwidth\relax,
    innerrightmargin=13pt,innerleftmargin=13pt,%
    frametitlerule=true,%
    frametitlerulecolor=mydarkblue,%blue!20,%
    frametitlebackgroundcolor=mydarkblue,%blue!20,%
    frametitlerulewidth=0pt,% 2pt%
%%
   backgroundcolor=gray!15,
   middlelinecolor=gray!15,
   roundcorner=5pt,
   singleextra={%
      \fill[gray!60,rounded corners=2pt,] 
         ($(P)+(0,-2.05)$) -- ($(P)+(0.14,-0.55)$) --  ($(P)+(0.18,-0.25)$)
            [sharp corners] --  ($(P)+(0,-0.25)$) -- cycle ;
%
      \path let \p1=(P), \p2=(O) in ({(\x1-\x2)/2},\y2) coordinate (M) ;
      \shade[left color=gray!50,right color=gray!50,middle color=black!55,rounded corners] 
         ($(M)+(-0.51\linewidth,0)$) --
         ($(M)+(-0.51\linewidth,-0.135)$) -- ($(M)+(0,-0.055)$) 
            -- ($(M)+(0.51\linewidth,-0.135)$) -- ($(M)+(0.51\linewidth,0)$) ;
            % -- cycle ;
   },
   firstextra={%
      \fill[gray!60,rounded corners=2pt,] 
         ($(P)+(0,-2.05)$) -- ($(P)+(0.14,-0.55)$) --  ($(P)+(0.18,-0.25)$)
            [sharp corners] --  ($(P)+(0,-0.25)$) -- cycle ;
%
      \path let \p1=(P), \p2=(O) in ({(\x1-\x2)/2},\y2) coordinate (M) ;
      \fill[gray!15] ($(M)+(-0.5\linewidth-13pt,0)$)  --  ($(M)+(0.5\linewidth+13pt,0)$)
         [rounded corners=3pt] -- ($(M)+(0.5\linewidth+13pt,-0.2)$) -- ($(M)+(-0.5\linewidth-13pt,-0.2)$) 
         [sharp corners] -- cycle ;
%
      \shade[left color=gray!50,right color=gray!50,middle color=black!55,rounded corners] 
         ($(M)+(-0.51\linewidth,0-0.2)$) --
         ($(M)+(-0.51\linewidth,-0.135-0.2)$) -- ($(M)+(0,-0.055-0.2)$) 
            -- ($(M)+(0.51\linewidth,-0.135-0.2)$) -- ($(M)+(0.51\linewidth,0-0.2)$) ;
            % -- cycle ;
   },%
   roundcorner=5pt,
   secondextra={%
      \fill[gray!15]
         ($(P)+(-1.0\linewidth-13pt-13pt,0)$) -- ($(P)+(0.0\linewidth,0)$)
         [rounded corners=3pt] 
         -- ($(P)+(0.0\linewidth,+0.25)$) 
         -- ($(P)+(-0.5\linewidth-13pt-13pt,+0.22)$)
         -- ($(P)+(-1.0\linewidth-13pt-13pt,+0.25)$)
         [sharp corners] -- cycle ;
%
      \fill[gray!60,rounded corners=2pt,] 
         ($(P)+(0,-2.05)$) -- ($(P)+(0.14,-0.55)$) --  ($(P)+(0.18,-0.25)$)
            [sharp corners] --  ($(P)+(0,-0.25)$) -- cycle ;
%
      \path let \p1=(P), \p2=(O) in ({(\x1-\x2)/2},\y2) coordinate (M) ;
      \shade[left color=gray!50,right color=gray!50,middle color=black!55,rounded corners] 
         ($(M)+(-0.51\linewidth,0)$) --
         ($(M)+(-0.51\linewidth,-0.135)$) -- ($(M)+(0,-0.055)$) 
            -- ($(M)+(0.51\linewidth,-0.135)$) -- ($(M)+(0.51\linewidth,0)$) ;
            % -- cycle ;
   },%
   middleextra={%
      \fill[gray!15]
         ($(P)+(-1.0\linewidth-13pt-13pt,0)$) -- ($(P)+(0.0\linewidth,0)$)
         [rounded corners=3pt] 
         -- ($(P)+(0.0\linewidth,+0.25)$) 
         -- ($(P)+(-0.5\linewidth-13pt-13pt,+0.22)$)
         -- ($(P)+(-1.0\linewidth-13pt-13pt,+0.25)$)
         [sharp corners] -- cycle ;
%
      \fill[gray!60,rounded corners=2pt,] 
         ($(P)+(0,-2.05)$) -- ($(P)+(0.14,-0.55)$) --  ($(P)+(0.18,-0.25)$)
            [sharp corners] --  ($(P)+(0,-0.25)$) -- cycle ;
%
      \path let \p1=(P), \p2=(O) in ({(\x1-\x2)/2},\y2) coordinate (M) ;
      \fill[gray!15] ($(M)+(-0.5\linewidth-13pt,0)$)  --  ($(M)+(0.5\linewidth+13pt,0)$)
         [rounded corners=3pt] -- ($(M)+(0.5\linewidth+13pt,-0.2)$) -- ($(M)+(-0.5\linewidth-13pt,-0.2)$) 
         [sharp corners] -- cycle ;
%
      \shade[left color=gray!50,right color=gray!50,middle color=black!55,rounded corners] 
         ($(M)+(-0.51\linewidth,0-0.2)$) --
         ($(M)+(-0.51\linewidth,-0.135-0.2)$) -- ($(M)+(0,-0.055-0.2)$) 
            -- ($(M)+(0.51\linewidth,-0.135-0.2)$) -- ($(M)+(0.51\linewidth,0-0.2)$) ;
            % -- cycle ;
   },%
}

\newenvironment{myInsight}[1][]{%
    \refstepcounter{myInsight}%
    \mdfsetup{skipabove=\topskip,skipbelow=\topskip}%
    \ifstrempty{#1}%
      {% empty argument
        \mdfsetup{%
          frametitle={%
                    \color{white}
                    Approfondimento~\thechapter.\themyInsight
            }%
        }%
      }%
      {% non-empty argument
        \mdfsetup{%
          frametitle={%
                    \color{white}
                    Approfondimento~\thechapter.\themyInsight:~#1
            }%
        }%
      }%
      %\mdfsetup{innertopmargin=10pt,linecolor=blue!20,%
      %  linewidth=2pt,topline=true,
      %  innermargin=-10pt,outermargin=-10pt,
      %  innerrightmargin=10pt,innerleftmargin=10pt,
      %}
      \begin{mdframed}[style=InsightStyleDefault]\relax%      
  }%
  {%
   \end{mdframed}
  }


%\usepackage{kantlipsum}
%\input{mdframedaddon}

%---------------------------------------------------------------------------------
%%%%%%%%%%%%%%%%%% Buon fine pagina; Attenzione: non funziona sempre bene!
%% Il numero opzionale serve per indicare di quante righe prima della fine pagina
%% si può eseguire un salto pagina. Predefinite 4 righe

\newcommand*\goodpagebreak[1][4]{%
   \ifdim\dimexpr\pagegoal-\pagetotal<#1\baselineskip\newpage\fi}


%------------------------------------------------------------------------------------------
% Draft and Copyright

%................................ "ESO-PIC"
\usepackage%
           %[pscoord]
           {eso-pic}

%\usepackage{everyshi}% needed by eso-pic

\newcommand{\DraftText}{%
%         \sf\bfseries\Large%
%         \mbox{%\color[gray]{.65}%
%         DRAFT\ {\color{magenta}\fbox{\footnotesize\texttt{\myDocVersion}}}%
%            \ {\footnotesize Copyright\hspace{2pt}\copyright\ \myAuthorListAbbreviated}%
%         }%
}

\newcommand\BackgroundText{%
%    \checkoddpage% needs chngpage/changepage, I guess it needs two latex passes
%    \ifoddpage% needs changepage, RIGHT-HAND PAGE
%        \put(\LenToUnit{20.9cm},% horizontal offset (>0 from left to right)
%            \LenToUnit{7.5cm}%   vertical offset (>0 from bottom to top)
%        )%
%        {%
%            \makebox(0,0)[l]{%
%                \resizebox{!}{!}{%
%                    \rotatebox{90}{\textsf{\textbf{\color{mylightblue}\DraftText}}}%
%                }%
%            }%
%        }%
%    \else% LEFT-HAND PAGE
%        \put(\LenToUnit{0.2cm}, % horizontal offset (>0 from left to right)
%             \LenToUnit{7.5cm}%   vertical offset (>0 from bottom to top)
%        )%
%        {%
%            \makebox(0,0)[l]{%
%                \resizebox{!}{!}{%
%                    \rotatebox{90}{\textsf{\textbf{\color{mylightblue}\DraftText}}}%
%                }%
%            }%
%        }%
%    \fi
%    %\fi
}
%% show the text <see main.tex>
\AddToShipoutPicture{\BackgroundText}

%------------------------------------------------------------------------------------------
% Other page decorations
% see: http://tex.stackexchange.com/questions/48641/chapter-title-in-rotated-vertical-box-at-the-margin#48647

\usepackage{background}% by C. Medina, http://texdoc.net/pkg/background
% background common settings
\SetBgScale{1}
\SetBgAngle{0}
\SetBgOpacity{1}
\SetBgContents{}

% auxiliary counter
\newcounter{chapshift}
\addtocounter{chapshift}{-1}

% the list of colors to be used (add more if needed)
\newcommand\BoxColor{%
  \ifcase\thechapshift mydarkblue\or mydarkblue!80\or mydarkblue!60\or mydarkblue!40\else mydarkblue!20\fi}
\newcommand\BoxTextColor{%
  \ifcase\thechapshift white\or white\or white\or mydarkblue\else mydarkblue\fi}

% the main command; the mandatory argument sets the color of the vertical box
\makeatletter
\newcommand\ChapFrame[1]{%
\AddEverypageHook{%
\ifthenelse{\isodd{\thepage}}
{\SetBgContents{%
  \begin{tikzpicture}[overlay,remember picture]
  \node[fill=\BoxColor,inner sep=0pt,rectangle,
    text width=0.70cm,text height=6cm,
    align=center,anchor=north east] 
  % at ($ (current page.north east) + (-0cm,-2*\thechapshift cm) $) 
  at ($ (current page.north east) + (-0cm,\thechapshift\topmargin) $) 
  {%
      \rotatebox{90}{%
          \hspace*{.3cm}%
          %\parbox[c][0.73cm][t]{5.4cm}{%  \parbox[position][height][inner-pos]{width}{text}
          % example: \raggedright\textcolor{black}{\scshape\leftmark}
          \adjustbox{center=5.5cm}{%
              \maxsizebox{5.4cm}{0.67cm}{%
                \color{\BoxTextColor}%
                #1
              }% end-of-minsizebox
          }% end-of-adjustbox
          %}% end-of-parbox
      }% end-of-rotatebox
  };
  \end{tikzpicture}}%
}
{\SetBgContents{%
  \begin{tikzpicture}[overlay,remember picture]
  \node[fill=\BoxColor,inner sep=0pt,rectangle,
    text width=0.70cm,text height=6cm,
    align=center,anchor=north west] 
  % at ($ (current page.north west) + (-0cm,-2*\thechapshift cm) $) 
  at ($ (current page.north west) + (-0cm,\thechapshift\topmargin) $) 
  {%
      \rotatebox{90}{%
          \hspace*{.3cm}%
          %\parbox[c][0.73cm][t]{5.4cm}{%  \parbox[position][height][inner-pos]{width}{text}
          % example: \raggedright\textcolor{black}{\scshape\leftmark}
          \adjustbox{center=5.5cm}{%
              \maxsizebox{5.4cm}{0.67cm}{%
                \color{\BoxTextColor}%
                #1
              }% end-of-minsizebox
          }% end-of-adjustbox
          %}% end-of-parbox
      }% end-of-rotatebox
  };
  \end{tikzpicture}}
}
\bg@material}%
  \stepcounter{chapshift}
}
\makeatother



%------------------------------------------------------------------------------------------
% Document layout and related stuff

\usepackage{setspace}
%\onehalfspace
%\renewcommand{\baselinestretch}{1.1}
%\setstretch{1.0}
\def\mynormalstretch{1.15}% or 1.1
\setstretch{\mynormalstretch}% with this you fine-tune the interline spacing

\usepackage{mparhack} % marginpar hack, for two-side docs

\usepackage{lscape}
\usepackage{afterpage}
\usepackage{fancyvrb}

%*************************************************************************************
% Boxes mimicking Anderson's book

% see package tcolorbox

%------------------------------------------------------------------------------------------
% Listings and related stuff

\usepackage{fancyvrb}

\usepackage[]{listings}
\lstset{language=Java,
	showspaces=false,
	keepspaces,
	showtabs=false,
	tabsize=2,
	breaklines=true,
	showstringspaces=false,
	breakatwhitespace=true,
	commentstyle=\color{pgreen},
	keywordstyle=\color{pblue},
	stringstyle=\color{pred},
	basicstyle=\footnotesize\ttfamily,
	frame=tb,
	rulecolor=\color{black},
	framesep=1.2mm,
%	rulesep=1mm,
	rulesepcolor=\color{black},
	xleftmargin=8mm,
	framexleftmargin=8mm,
	fillcolor=\color{mymidgray},
	backgroundcolor=\color{white},
	numbers=left,
	numberstyle=\normalfont\footnotesize\color{mydarkgray}
%	numbersep=8pt
}
%	moredelim=[il][\textcolor{black}]{$$},
%	moredelim=[is][\textcolor{pgrey}]{\%\%}{\%\%}

\lstdefinelanguage{XML}
{
	morestring=[b]",
	morestring=[s]{>}{<},
	morecomment=[s]{<?}{?>},
	stringstyle=\color{black},
	identifierstyle=\color{mydarkblue},
	keywordstyle=\color{mydarkgray},
	morekeywords={xmlns,version,type,id,unit}% list your attributes here
}

\renewcommand{\lstlistingname}{Listing}
%\newcommand{\lstInCaption}[1]{\lstinline[basicstyle=\ttfamily\footnotesize]|#1|}

%%******************************************************************************************
%
% AUTHOR:           Agostino De Marco
% DESCRIPTION:      This is "_lst_reset.tex", an auxiliary LaTeX source file.
%                   Reset all settings related to package listings.
%                   The XML language is selected.
%
%******************************************************************************************

%------------------------------------------------------------------------------------------
% Meta-commands for the TeXworks editor
%
% !TeX root = ./Libro_MS.tex
% !TEX encoding = UTF-8
% !TEX program = pdflatex
%------------------------------------------------------------------------------------------

% language --> REVERT TO XML
% morekeywords, emph, etc --> VOID

\lstset{%
    language=xml,%
    breaklines=true,%
    tabsize=3,%
    showstringspaces=false,%
    aboveskip=3pt,%
    belowskip=3pt,%
    inputencoding=utf8,
    extendedchars=true,
    literate={à}{{\`a}}1 {è}{{\`e}}1 {é}{{\'e}}1 {ì}{{\`\i}}1 {ò}{{\`o}}1 {ù}{{\`u}}1,
    frame=none,% single,   
    %===========================================================
    %framesep=3pt,% expand outward.
    %framerule=0.4pt,% expand outward.
    %xleftmargin=3.4pt,% make the frame fits in the text area. 
    %xrightmargin=3.4pt,% make the frame fits in the text area.
    %=========================================================== 
    rulecolor=\color{red}%
}
\lstset{
   morekeywords={%
      %
      % TO DO: put more keywords here
      %
   },%
   % see http://tex.stackexchange.com/questions/27074/how-to-emphasize-with-in-a-listing-an-identifier-containing-a-digit
   otherkeywords={%
      %
      % TO DO: put keywords with digits here
      %
   },%
   morekeywords=[2]{%
      %
      % TO DO: put more keywords 2 here
      %
   },%
   keywordstyle=[2]{\color{mydarkgreen}\bfseries},%
   morekeywords=[3]{%
      %
      % TO DO: put more keywords 3 here
      %
   },%
   keywordstyle=[3]{\color{red}\bfseries},%
   %
   emph={%
      %
      % TO DO: put more emphasized words here
      %
   },
   emphstyle={\color{mydarkblue}\bfseries},%
   emph=[2]{%
      %
      % TO DO: put more emphasized words 2 here
      %
   },
   emphstyle=[2]{\color{mydarkblue}\bfseries}% \sffamily ??
}

% loads settings from ./_lst_reset.tex
%%******************************************************************************************
%
% AUTHOR:           Agostino De Marco
% DESCRIPTION:      This is "_lst_reset.tex", an auxiliary LaTeX source file.
%                   Reset all settings related to package listings.
%                   The XML language is selected.
%
%******************************************************************************************

%------------------------------------------------------------------------------------------
% Meta-commands for the TeXworks editor
%
% !TeX root = ./Libro_MS.tex
% !TEX encoding = UTF-8
% !TEX program = pdflatex
%------------------------------------------------------------------------------------------

% New definitions here

\lstdefinestyle{CommonCppStyle}
{
        breaklines=true,
        tabsize=3, 
        showstringspaces=false,
        aboveskip=0pt,
        belowskip=0pt, 
        extendedchars=\true,
        language=C++,
        frame=none,   
        %===========================================================
        framesep=3pt,%expand outward.
        framerule=0.4pt,%expand outward.
        xleftmargin=-3.4pt,%make the frame fits in the text area. 
        xrightmargin=3.4pt,%make the frame fits in the text area.
        %=========================================================== 
        rulecolor=\color{Red}%
}

\lstdefinestyle{ThemeCppA}
{
        style=CommonCppStyle,
        backgroundcolor=\color{Yellow!10},
        basicstyle=\scriptsize\color{Black}\ttfamily,
        keywordstyle=\color{Orange},
        identifierstyle=\color{Cyan},
        stringstyle=\color{Red}, 
        commentstyle=\color{Green} 
}

\newcommand{\IncludeCppCode}[2][style=ThemeCppA]
{
    \lstinputlisting[#1,caption={\href{#2}{#2}}]{#2}
}

\lstdefinelanguage{MyNoSpecificLanguage}{%
    emph={},%
    emph={[2]},%
    emph={[3]},%
    emph={[4]}%
}

% see: http://tex.stackexchange.com/questions/18532/listing-language-for-screen-session
\lstnewenvironment{myCommandLine}
    {\lstset{%
        basicstyle=\ttfamily\small,%
        backgroundcolor=\color{mylightgray},%
        frame=none,%
        columns=[l]fixed,%flexible, %
        prebreak=\makebox[1.4ex][r]{\color{blue}\raisebox{-0.9ex}[0ex][0ex]{\scriptsize\ensuremath{\hookleftarrow}}},%
        keepspaces=true,%
        escapechar=\»,%~,% use this for output
        morekeywords={ls,pwd,cd,cp,mv},%
        identifierstyle=\color{colIdentifier},%
        keywordstyle=\color{red!60!black}\bfseries,% \color{colKeys},%\fontseries{b}\selectfont, %
        stringstyle=\color{colString},%
        literate={\$}{{\textcolor{blue}{\$}}}{1}% prompt symbol
        }
    }
    {}
\lstnewenvironment{myCommandLineOutput}
    {\lstset{%
        language=MyNoSpecificLanguage,%
        basicstyle=\ttfamily\footnotesize,%
        backgroundcolor=\color{mylightgray},% \color{hellgelb}, %
        lineskip=-1pt,%
        frame=none,%
        columns=[l]fixed,%flexible, %
        prebreak=\makebox[1.4ex][r]{\color{blue}\raisebox{-0.9ex}[0ex][0ex]{\scriptsize\ensuremath{\hookleftarrow}}},
        keepspaces=true,%
        escapechar=~% use this for output
        }
    }
    {}
\lstnewenvironment{myCommandLineOutputSmaller}
    {\lstset{%
        language=MyNoSpecificLanguage,%
        basicstyle=\ttfamily\scriptsize,%
        backgroundcolor=\color{mylightgray},% \color{hellgelb}, %
        lineskip=-1pt,%
        frame=none,%
        columns=[l]fixed,%flexible, %
        prebreak=\makebox[1.4ex][r]{\color{blue}\raisebox{-0.9ex}[0ex][0ex]{\scriptsize\ensuremath{\hookleftarrow}}},
        keepspaces=true,%
        escapechar=~% use this for output
        }
    }
    {}

\lstloadlanguages{Matlab,C++,xml,MyNoSpecificLanguage}

% loads settings from ./_lst_defs.tex


%------------------------------------------------------------------------------------------
% Index management

\usepackage{imakeidx}

\makeatletter

%%% The argument to \indexnote will be set just before the start of 
%%% the index
\long\def\indexnote#1{\def\@indexnote{\noindent #1}}

%%%% The index is typeset in 2 columns, with automatic balance in
%%%% the last page; in order to save space it is set in small type

\let\imki@idxprologue\empty
\def\imki@columns{2}
\renewenvironment{theindex}{%
   \clearpage
   \csname phantomsection\endcsname
   %\chapter{\indexname}%
   \@makeschapterhead{\indexname}%
   %\@mkboth{\indexname}{\indexname}%
   \@indexnote\par\bigskip
   \parindent\z@
   \parskip\z@ \@plus .3\p@\relax
   \columnseprule \z@
   \columnsep 15\p@
   \raggedright
   \let\item\@idxitem
   \begin{multicols}{\imki@columns}[\imki@idxprologue]
           \addcontentsline{toc}{chapter}{\indexname}%
           \thispagestyle{plain}%
      \small
}{\end{multicols}\gdef\imki@idxprologue{}\clearpage}

\makeatother

\makeindex

%------------------------------------------------------------------------------------------
% Definitions for the book

\makeatletter

%%% We redefine \frontmatter and \mainmatter to give continuous 
%%% numbering from the first page (no roman numbers for the front 
%%% matter)

\renewcommand\frontmatter{%
  \@mainmatterfalse}
\renewcommand\mainmatter{%
  \@mainmattertrue}

%%% Modify \cleardoublepage to give a really blank page
%%% Use the emptypage package if it is in the TeX tree,
%%% otherwise use one of the many variations on the theme

%\IfFileExists{emptypage.sty}
%  {\usepackage{emptypage}}
%  {\usepackage{afterpage}
%   \g@laddto@macro\cleardoublepage
%  {\afterpage{\thispagestyle{empty}}}}

%\renewcommand*\cleardoublepage[1][empty]{%
%\clearpage
%\ifodd\c@page\else
%  \thispagestyle{#1}
%  \null\clearpage
%  \fi
%}

%%% some misc definitions

% \meta prints its argument as a metavariable
\protected\def\meta#1{$\langle$\textit{\rmfamily#1\,}$\rangle$}
\def\metaind#1{\index{#1=\meta{#1}}\meta{#1}}
%%% danger is for notes in small prints as in the TeXbook
\newenvironment{danger}
  {\par\addvspace{\medskipamount}\penalty\clubpenalty
   \leavevmode\small\llap{\ding{42}\ }\ignorespaces}
  {\par\addvspace{\medskipamount}}

\makeatother

% ABSTRACT defs
\newenvironment{abstract}%
{\clearpage%
  \thispagestyle{empty}%
  \null \vfill\begin{center}%
  \bfseries \abstractname \end{center}}%
{\vfill\null}

%------------------------------------------------------------------------------------------
% Bibliography

%\usepackage{natbib}

%% http://tex.stackexchange.com/questions/5091/what-to-do-to-switch-to-biblatex
\usepackage[%
    backend=biber,
    citestyle=numeric-comp, % authoryear
    bibstyle=numeric-comp, % authoryear,
    doi=false,url=false,isbn=false,
%   sorting=nyt,% none
    hyperref=true,
%   %style=apa,
    natbib=true]{biblatex}

%% see \bibliographystyle command in main file Tesi.tex
%%
%% iteresting links: 
%%    http://www.economics.utoronto.ca/osborne/latex/BIBTEX.HTM

%------------------------------------------------------------------------------------------
% One of the last package to be loaded must be hyperref

\usepackage[%
            %dvipdfmx,%dvips,%
            %pdfborder = 0 0 1,
            baseurl= http://,
            colorlinks=true,%
            linkcolor=mydarkblue,% black
            citecolor=mydarkblue% black
            ]{hyperref}
\usepackage{cleveref}
%\usepackage{createspace}


%\addbibresource%
%%   [datatype=bibtex]%
%   {Tesi.bib}% extension required

%*************************************************************************************
% BIBLATEX SETTINGS POST-HYPERREF

\bibliography{Tesi_Bibliography}% with biblatex

\defbibheading{myBibliography}[\bibname]{\chapter*{\centering#1}
   %\markboth{#1}{#1}
   \markboth{}{}
}

\DeclareCiteCommand{\citetitle}{}{\printfield{title}}{;}{}

% http://tex.stackexchange.com/questions/83440/inputenc-error-unicode-char-u8-not-set-up-for-use-with-latex
%\DeclareUnicodeCharacter{00A0}{ }
%\DeclareUnicodeCharacter{00A0}{~}

%*************************************************************************************

%%% Finishing touches
\setcounter{secnumdepth}{2}% 1
\setcounter{tocdepth}{2}% 1

%%% With the Fourier fonts, the em is too small for 3 digit page 
%%% numbers in the table of contents, so we adjust \@pnumwidth
\makeatletter
\AtBeginDocument{%
  \edef\@pnumwidth{\the\dimexpr\fontcharwd\font`\1*3\relax}%
}
\makeatother

\let\_\UnDeFiNeD
\DeclareRobustCommand\_{%
  \,\vrule height.2pt depth.2pt width.5em\,}

%% EOF% all preamble settings in ./_preamble.tex

%------------------------------------------------------------------------------------------
% LOCAL USER-DEFINED MACROS & SETTINGS

\input{_local_macros}

%------------------------------------------------------------------------------------------
% Stuff related to package glossaries
% loaded here after _local_macros.tex to make all custom definition
% known by the glossary code

%******************************************************************************************
%
% AUTHOR:           Agostino De Marco
% DESCRIPTION:      This is "_glossaty.tex", an auxiliary LaTeX source file.
%                   Here we collect all customizations related to package glossaries
%                   (Glossaries, Nomenclature, Lists of Symbols and Acronyms).
%
% See: http://www.latex-community.org/index.php?option=com_content&view=article&id=263:glossaries-nomenclature-lists-of-symbols-and-acronyms&catid=55:latex-general&Itemid=114
%
%
% FROM THE MANUAL: http://texdoc.net/texmf-dist/doc/latex/glossaries/glossariesbegin.pdf
%
% Note that the glossaries package must be loaded after the hyperref package
% (contrary to the general advice that hyperref should be loaded last).
% The glossaries package should also be loaded after html, inputenc, babel and ngerman.
%******************************************************************************************

%------------------------------------------------------------------------------------------
% Meta-commands for the TeXworks editor
%
% !TeX root = ./Tesi.tex
% !TEX encoding = UTF-8
% !TEX program = pdflatex
%------------------------------------------------------------------------------------------

\usepackage[%
            acronym,%        i nclude a list of acronyms
            %style=tree,%
            translate=false,%
            sort=standard,% def,% standard, use
            nonumberlist,%  do not display page numbers in nomenclatures
            toc% print glossaries/nomenclatures in the table of contents
            ]{glossaries}

%% http://en.wikibooks.org/wiki/LaTeX/Glossary
\usepackage{xparse}
\DeclareDocumentCommand{\newdualentry}{ O{} O{} m m m m } {
  \newglossaryentry{gls-#3}{name={#5},text={#5\glsadd{#3}},
    description={#6},#1
  }
  \newacronym[see={[Glossary:]{gls-#3}},#2]{#3}{#4}{#5\glsadd{gls-#3}}
}
%
% Use as follows:
%
%\newdualentry{OWD} % label
%  {OWD}            % abbreviation
%  {One-Way Delay}  % long form
%  {The time a packet uses through a network from one host to another} % description


\newcommand{\myGlossaryItemColorMix}{%
  red!30!black%
}
\newcommand{\myGlossaryItemColor}{\color{\myGlossaryItemColorMix}}

\renewcommand*{\glstextformat}[1]{\textcolor{\myGlossaryItemColorMix}{#1}}

%\usepackage{glossary-longragged}

\addto\captionsitalian{%
   \renewcommand*{\glossaryname}{Glossary}%
   \renewcommand*{\acronymname}{Acronyms}%
   \renewcommand*{\entryname}{Notation}%
   \renewcommand*{\descriptionname}{Description}%
   \renewcommand*{\symbolname}{Symbol}%
   \renewcommand*{\pagelistname}{Page List}% Page List
   \renewcommand*{\glssymbolsgroupname}{Symbols}%
   \renewcommand*{\glsnumbersgroupname}{Numbers}%
}

\newglossary{symbols}{sym}{sbl}{List of symbols}

\makeglossaries % This needs to come after any occurrence of \newglossary

\loadglsentries{Backmatter/Glossary}

%%% HOW TO update glossaries and nomenclature:
%%%
%%%   1.   define a new entry in file Tesi_Glossary.tex
%%%         e.g. one labelled "x:Body"
%%%
%%%   2.   use the entry in the document, 
%%%         e.g. "... here we have \gls{x:Body} pointing towards the front ..."
%%%
%%%         [this is not strictly necessary, as I've given the command \glsaddall
%%%          that adds all the entries to the glossaries without referencing each 
%%%          one explicitly]
%%%
%%%   3.   compile the document, via TeXworks or from command line: 
%%%         > pdflatex Tesi
%%%
%%%   4.   create the necessary files using the perl script makeglossaries from command line:
%%%         > makeglossaries Tesi
%%%
%%%   5.   re-compile the document, via TeXworks or from command line: 
%%%         > pdflatex Tesi


%------------------------------------------------------------------------------------------
%                                                              B E G I N    D O C U M E N T
%
\begin{document}
\selectlanguage{english}

%------------------------------------------------------------------------------------------
%                                                                   F R O N T E S P I Z I O
\begin{frontespizio}
\begin{Preambolo*}
\usepackage{newtxtext}
\renewcommand{\frontlogosep}{30pt}
%\renewcommand{\frontinstitutionfont}{\fontsize{15}{10}\bfseries }
%\renewcommand{\frontdivisionfont}{\fontsize{15}{18}\mdseries }
\renewcommand{\fronttitlefont}{\fontsize{23}{24}\bfseries }
%\renewcommand{\frontfootfont}{\fontsize{15}{17}\bfseries }
%\renewcommand{\frontfixednamesfont}{\fontsize{15}{20}\mdseries }
%\renewcommand{\frontnamesfont}{\fontsize{15}{20}\bfseries }
%\renewcommand{\frontsmallfont}{\fontsize{15}{15}\bfseries }
\renewcommand{\frontpretitlefont}{\fontsize{14}{16}\mdseries }
\end{Preambolo*}
\Margini {3cm}{2.5cm}{3cm}{2cm}
\Istituzione {Universit\`a degli Studi di Napoli Federico II\\[0.5\baselineskip ] \textsc {Scuola Politecnica e delle Scienze di Base} }
\Logo [3.5cm]{logounina.pdf}
\Dipartimento {Ingegneria Industriale}
\Corso [Laurea]{Ingegneria Aerospaziale}
\Titoletto {Tesi di Laurea Magistrale\\ in\\ Ingegneria Aerospaziale ed Astronautica}
\Titolo {Development of a Java Application for Parametric Aircraft Design}
\Sottotitolo {\mbox{}\\[5pt]}
\NCandidato {Candidato}
\Candidato [M53/334]{Lorenzo~Attanasio} % \rule{0pt}{1.3em}
\NRelatore {Relatori}{}
\Relatore {\begin{tabular}{@{}l@{}}Prof.~Fabrizio~Nicolosi\\ Prof.~Agostino~De~Marco\end{tabular}}
\Piede {Anno Accademico 2013/2014}
\end{frontespizio}
%------------------------------------------------------------------------------------------------------------------------------------------

% -----------------------------------------------------------------------------------------
%                                                                             S U M M A R Y
%
\newpage\null\thispagestyle{empty}\newpage\thispagestyle{empty}

\newgeometry{left=3cm,right=3cm}
\vspace*{2cm}
\begin{center}
\textit{A mia madre, per aver sempre posto la nostra famiglia prima di ogni altra cosa}

\medskip
\textit{A mio padre, per avermi insegnato a non arrendermi}
\end{center}
\vspace*{\fill}

\restoregeometry
\clearpage

\selectlanguage{english}
\begin{abstract}
The thesis deals with the development of \theApplicationName{} (\theApplicationNameFull{}), a Java-based application conceived as a fast, reliable and user friendly computational aid for aircraft designers in the conceptual/preliminary design phases of a transport aircraft. The ultimate goal of such a software framework is to perform a multi-disciplinary analysis of an aircraft and then search for an optimized configuration. The search domain boundaries are usually defined by the user through a set of specified parameters. 

Currently, the software is able to estimate the aircraft weight breakdown, the center of gravity location, the main aerodynamic parameters, and some stability derivatives. All these types of estimates can be usually performed using several interchangeable analysis methods. The wing aerodynamic load has been in particular estimated with a numerical method taken from \citet{NASA:Blackwell}. An extensive work has been carried out to validate all the results returned by the application.

ADOpT can be used from the command line or with a dedicated graphical user interface (GUI). The GUI allows the user to have an immediate feedback about the aircraft features when changing the input parameters, to manage multiple aircraft simultaneously and compare them side by side, and to view a 3D CAD model of the aircraft. The CAD model has been generated using the well known \OpenCascadeNameFull{} libraries, and can be eventually exported to file for external usage (e.g., in CAD/CAE suites).

Regarding the Input/Output capability of \theApplicationName{}, the application accepts configuration files in XML (eXtensible Markup Language) format and exports the results on file in two possible formats: XML and Microsoft Excel (XLS). The XML input files are also used in interactive work sessions to import a predefined aircraft and populate the GUI controls accordingly. The application has been developed making extensive use of the latest Java features introduced in 2014, i.\,e.~the Java~8 release by Oracle. This release includes the JavaFX platform, which has been used to set up the 3D view.

\end{abstract}

\selectlanguage{italian}
\begin{abstract}
La tesi descrive lo sviluppo di \theApplicationName{} (\theApplicationNameFull{}), un'applicazione scritta in linguaggio Java concepita per essere uno strumento veloce, affidabile e di semplice utilizzo per le fasi di sviluppo concettuale e preliminare di un velivolo da trasporto. Lo scopo finale del programma è effettuare un'analisi multidisciplinare di una configurazione definita dall'utente e successivamente alterarla in modo da ottenerne una ottimizzata. Il dominio di ricerca di tale configurazione è definito dall'utente tramite un apposito insieme di parametri.

Allo stato attuale il programma effettua la stima dei pesi dei principali componenti di un velivolo, valuta la posizione del baricentro, i principali parametri aerodinamici e alcune derivate di stabilità. La stima di ciascun parametro può essere generalmente effettuata tramite diversi metodi tra loro intercambiabili. Il carico aerodinamico sull'ala, in particolare, è stato stimato tramite un metodo numerico tratto da \citet{NASA:Blackwell}. Molta attenzione è stata in generale dedicata alla validazione dei risultati forniti dall'applicazione.

ADOpT può essere usato sia da riga di comando sia tramite un'apposita interfaccia grafica (GUI). Quest'ultima fornisce un riscontro immediato riguardo il cambiamento delle prestazioni del velivolo nel momento in cui l'utente modifica uno o più parametri di input; inoltre permette di gestire più velivoli (o più configurazioni dello stesso velivolo) simultaneamente, di confrontarne le caratteristiche e di visualizzarne il modello CAD. Questo è generato tramite la libreria \OpenCascadeName{} e può essere salvato su file in modo da poterlo usare in altri applicativi.

Per quanto riguarda i files di input e di output, l'applicazione accetta files in formato XML (eXtensible Markup Language) contenenti la configurazione del velivolo e può esportare i risulati sia in formato XML sia in formato XLS. I file XML di uscita possono esssere successivamente re-importati e quindi modificati tramite l'interfaccia grafica. L'applicazione è stata sviluppata facendo largo uso delle ultime caratteristiche del linguaggio Java introdotte da Oracle nel 2014 con la versione 8; questa include, tra l'altro, la piattaforma JavaFX che è stata usata per costruire il visualizzatore del modello CAD.
\end{abstract}
\selectlanguage{english}

%\clearpage
%%******************************************************************************************
%
% AUTHOR:           Agostino De Marco, Jon S. Berndt, and the JSBSim Development Team
% DESCRIPTION:      This is "CopyrightPage.tex" file, included by the master source file
%                   "Libro_MS.tex".
%
%******************************************************************************************
%
%------------------------------------------------------------------------------------------
% Meta-commands for the TeXworks editor
%
% !TeX root = ./Tesi.tex
% !TEX encoding = UTF-8
% !TEX program = pdflatex


\thispagestyle{empty}
\vspace*{\fill}
{\relsize{-1}
\begin{center}
{\bf Copyright Declaration}
\end{center}

\centering
\begin{minipage}{0.85\textwidth}
No part of this publication may be reproduced, stored in a retrieval system or transmitted in any form or by any means electronic, mechanical, photocopying, recording or otherwise without the prior written permission of the author.
%\begin{asparaitem}[$\circ$]% needs paralist
%\item
%Non sono consentite la riproduzione e la circolazione in formato cartaceo o elettronico.
%\item
%Non è consentito l'impiego a scopi commerciali se non previo accordo.
%\item
%\`E gradita la segnalazione di errori o refusi.
%\end{asparaitem}
\end{minipage}

\bigskip

\centering Copyright © Vittorio Trifari\\
%Università  degli Studi di Napoli ``Federico II''
\textsc{Università degli Studi di Napoli Federico II}

\medskip
\hfil (Italian Copyright law 22.04.1941 n. 633)
}% end of relsize

\frontmatter
%------------------------------------------------------------------------------------------
%                                                                       T I T L E   P A G E
%
%\input{CoverPage}
%%******************************************************************************************
%
% AUTHOR:           Agostino De Marco, Jon S. Berndt, and the JSBSim Development Team
% DESCRIPTION:      This is "TitlePage.tex" file, included by the master source file
%                   "Libro_MS.tex".
%
%******************************************************************************************
%
%------------------------------------------------------------------------------------------
% Meta-commands for the TeXworks editor
%
% !TeX root = ./Tesi.tex
% !TEX encoding = UTF-8
% !TEX program = pdflatex

\begin{titlepage}
\raggedright\leftskip=.1\textwidth
\large
Domenico~P.~Coiro\\
Agostino~De~Marco\\
Fabrizio~Nicolosi

\vspace*{\fill}

{\fontsize{46}{50}\selectfont% allowed by Type 1 fonts
   \bfseries
   Elementi di\\[10pt] Meccanica del volo
}%
\\[0.5\baselineskip]
{\color{blue!65!black}\rule{\linewidth}{0.4pt}}\mbox{}\\[0.7\baselineskip]
{\fontsize{36}{40}\selectfont%\Large
   \bfseries
   Manovre e stabilità statica
}%
\\[0.5\baselineskip]
{\color{blue!65!black}\rule{\linewidth}{0.4pt}}\mbox{}\\[0.7\baselineskip]
{\fontsize{36}{40}\selectfont%\Large
   \bfseries
%   %% QUADERNO 1
%   {\color{blue!65!black}Quaderno 1}\\[10pt]
%   \adjustbox{width=0.9\textwidth}{Definizioni di base e notazioni}
   %% QUADERNO 2
   {\color{blue!65!black}Quaderno 2}\\[10pt]
   \adjustbox{width=0.9\textwidth}{Richiami di Aerodinamica}
}
\vspace*{\fill}

\vspace*{\stretch{3}}
\normalsize\normalfont
{\color{magenta}\fbox{\texttt{\myDocVersion}}}\\[3pt]
%%%
%%%\myDocDate% hard-coded in _init.tex
%%%This document was created on: \today\ at \currenttime.% needs package datetime
%%%
\monthname[\the\month]\ \the\year% needs package datetime
\end{titlepage}
%
%% EOF

%
%------------------------------------------------------------------------------------------
%                                                               C O P Y R I G H T   P A G E
%
%
\cleardoublepage
%
%
% -----------------------------------------------------------------------------------------
%                                                         T A B L E   O F   C O N T E N T S
%
\tableofcontents
%
% -----------------------------------------------------------------------------------------
%                                                 I N T R O D U C T O R Y   M A T E R I A L
%
%******************************************************************************************
%
% AUTHOR:           Agostino De Marco
% DESCRIPTION:      This is "Acknowledgements.tex".
%                   Goes in document after table of contents.
%
%******************************************************************************************
%
%------------------------------------------------------------------------------------------
% Meta-commands for the TeXworks editor
%
% !TeX root = ./Tesi.tex
% !TEX encoding = UTF-8
% !TEX program = pdflatex
%------------------------------------------------------------------------------------------
%
\chapter{Acknowledgements}

La mia carriera universitaria non è certo iniziata in modo brillante: ai primi due esami sono stato bocciato e al terzo ho preso venti per cui decisi di rifarlo. Un grazie di cuore va per questo al Professor Armando d'Anna, per aver creduto in me quando io stesso non ci credevo più.

\bigskip
\noindent
Un ringraziamento sincero va al Professor De Marco e al Professor Nicolosi per la disponibilità, la cordialità, i consigli e le conoscenze trasferitemi in questi ultimi mesi.

\bigskip
\noindent
La studente universitario è sì uno studente ma lo studio è ormai duro come un lavoro. Questo però non gli rende (economicamente) nulla, proprio nell'età in cui il desiderio di indipendenza si fa forte. Ringrazio pertanto mia nonna Carmela, che nonostante la differenza generazionale ha sempre capito e assecondato le mie richieste.

\bigskip
\noindent
Grazie a mio fratello Angelo, per insegnarmi cose che da solo non scoprirei mai, e a mia sorella Eleonora: nessuna persona incarna l'adoloscenza e la voglia di vivere meglio di lei.

\bigskip
\noindent
Grazie a mio zio Carlo, per l'ottimismo con cui mi ha sempre incoraggiato in vista di un esame, e a mio zio Peppe, per ricordarmi di non prendere mai nessuno troppo sul serio.

\bigskip
\noindent
Se vuoi viaggiare veloce viaggia da solo, ma se vuoi andare lontano viaggia in compagnia, recita un proverbio africano. Grazie, grazie, grazie a Roberta, mio indispensabile specchio critico e sincero, per averlo smentito, permettendomi di viaggiare veloce ed in compagnia.

\bigskip
\noindent
Grazie, infine, a chi non c'è più: se oggi affrontiamo il mare mosso senza troppo timore è perchè qualcuno, tempo fa, ha costruito per noi una solida nave.

\clearpage


%%******************************************************************************************
%
% AUTHOR:           Agostino De Marco
% DESCRIPTION:      This is "Preface.tex".
%                   Goes in document after table of contents.
%
%******************************************************************************************
%
%------------------------------------------------------------------------------------------
% Meta-commands for the TeXworks editor
%
% !TeX root = ./Libro_MS.tex
% !TEX encoding = UTF-8
% !TEX program = pdflatex
%------------------------------------------------------------------------------------------
%
\chapter{Preface}

JSBSim was conceived in 1996 as a lightweight, data-driven, non-linear, six-degree-of-freedom (6DoF), batch simulation application aimed at modeling flight dynamics and control for aircraft. Since the earliest versions, JSBSim has benefited from the open source development environment it has grown within and from the wide variety of users that have contributed ideas for its continued improvement.

This document is split up into several parts. This is because JSBSim can be viewed from several different perspectives: from that of a flight vehicle model developer, from that of an integrator who will incorporate JSBSim into a full flight simulation architecture with visuals, and from that of a software developer who wants to adapt or enhance JSBSim with additional capabilities.
There is a QuickStart section that explains how to get started with JSBSim quickly. That is followed by Section One, which is a User’s Manual. The User’s Manual explains how to use JSBSim to make simulation runs, to create aircraft models, to write scripts, and how to perform various other tasks that do not involve changes to program code in JSBSim itself. Section Two is a Programmer’s Manual. The Programmer’s Manual explains the architecture of JSBSim – how the code is organized and how it works. Section Three is the Formulation Manual which contains a description of the math model and algorithms present in JSBSim. Section Four is a collection of some examples and case studies showing how JSBSim has been used.
What this document is and what it is not
This document is not an exhaustive reference on the derivation of the equations of motion and flight dynamics. For a text on that, see (Stevens \& Lewis, 2003), and (Zipfel, 2007). This document is meant to be the authoritative document about JSBSim.
Conventions used
When XML definitions are given, items in brackets (``\texttt{[]}`) are optional.

%\begin{danger}
%\lipsum[3-4]
%\end{danger}
%
\bigskip

\begin{flushright}
%League City, Texas (USA)---Napoli (Italy)\\
May 2012
\end{flushright}

\vfill
%\hrule
\medskip

{\footnotesize\noindent Readers may visit the software website

\smallskip
\texttt{http://www.jsbsim.org}

\smallskip\noindent
Here we put a disclaimer for the code included in the text.\par}


\markboth{}{}
\cleardoublepage

% -----------------------------------------------------------------------------------------
%                                                     O N E   M O R E   T I T L E   P A G E
%\thispagestyle{empty}
%\vspace*{\stretch{1}}
%\begin{flushright}
%\Huge\itshape Elementi di Meccanica del volo\\
%\LARGE Manovre e stabilità statica
%\\[35pt]
%%Definizioni di base e notazioni
%Richiami di Aerodinamica
%\end{flushright}
%\vspace*{\stretch{3}}

% -----------------------------------------------------------------------------------------
%                                                                M A I N    C O N T E N T S
\mainmatter
\pagestyle{myBookPageStyle}
%
% Here we include the core material of the book
%
%------------------------------------------------------------------------------------------
% Meta-commands for the TeXworks editor
%
% !TeX root = ./Tesi.tex
% !TEX encoding = UTF-8
% !TEX program = pdflatex
%
%\setcounter{chapter}{6}

\part{\theApplicationName{} application overview}
\input{contents/IntroductionUser}
\clearpage

\input{contents/InterfacesUser}
\clearpage

\input{contents/InputOutputUser}
\clearpage

%\input{contents/CADmodelUser}
%\clearpage

%\input{contents/RunningTheApplicationUser}
%\clearpage

\part{\theApplicationName{} technical overview}
\input{contents/IntroductionDeveloper}
\clearpage

%\input{contents/MainPackage}
%\clearpage

\input{contents/CorePackage}
\clearpage

\input{contents/CadPackage}
\clearpage

\input{contents/UtilitiesPackage}
\clearpage

%\input{contents/GuiPackage}
%\clearpage

\part{\theApplicationName{} theoretical background}

\input{contents/Weights}
\clearpage

%\input{contents/Balance}
%\clearpage

\input{contents/Aerodynamics}
\clearpage

%\part{Case Studies}
\input{contents/ATR-72}
\clearpage

\chapter*{Conclusions}
An aircraft design and optimization desktop application written in Java, and its functionality, has been introduced. The adoption of established software engineering practices, the use of advanced development tools, and concurrent development enabled the developer team to build a feature-rich application in a relatively short period of time. As of its design, the application is easily maintainable and extensible. The software is still growing and the choice of Java language was really helpful. Infact being Java a pure object oriented programming language, it greatly encourages and simplifies modularization. Each module (package) can be programmed quite independently so that it is relatively easy to divide the work among several programmers working simultaneously or one after the other.\\
The application, moreover, can be easily integrated into a comprehensive aircraft optimization cycle. As all analysis modules inside the JPADCore package will be completed and tested, the final purpose of the code will be to allow users to define a certain numbers of macroscopical geometrical parameters, along with a given objective function, and to receive as output the best set of the previous parameters which suits the wanted objective. These future targets will make the software able to carry out an analysis of an aircraft during its preliminary design phase in a fast and flexible way.
\clearpage


%% reset the material handled by package background
\makeatletter
\renewcommand{\bg@material}{}%
\makeatother

% -----------------------------------------------------------------------------------------
%                                                                       A P P E N D I C E S
% some configuration commands by package appendix
\cleardoublepage \phantomsection
\renewcommand{\chaptername}{Appendix}
\renewcommand{\appendixtocname}{Appendices}
\renewcommand{\appendixpagename}{Appendices}
\appendixpage
\noappendicestocpagenum
\addappheadtotoc
\pagestyle{myAppendixPageStyle}
%
% Here we include the appendix material of the book
%
\begin{appendices}% needs package appendix
%
%------------------------------------------------------------------------------------------
% Meta-commands for the TeXworks editor
%
% !TeX root = ./Libro_MS.tex
% !TEX encoding = UTF-8
% !TEX program = pdflatex

%%-------------------------------------------------------- APPENDIX 1
%
%%------------------------------------------------------------------------------------------
% Meta-commands for the TeXworks editor
%
% !TeX root = ../Tesi.tex
% !TEX encoding = UTF-8
% !TEX program = pdflatex
%
\chapter{Compiling ADOpT}
\label{ch:Compiling:ADOpT}
%\ChapFrame{{\bfseries\scshape Definizioni e Notazioni}}% NOTE: needs _three_ LaTeX passes
%
%
% motto...
%
{\smaller
\hfill\textit{Put here the quote}

\hfill -- And here the author}
%%
%%
%\setcounter{minitocdepth}{1}% 1: chapter level; 2: section level
%\minitoc %\mtcskip \minilof \minilot
%%
%
\section{Introduction}\label{sec:Intro:Compiling:ADOpT}

\lipsum

\pagestyle{pippo}

\chapter{HDF dataset and database reader creation}
\label{ch:hdigitalizer}
\markboth{HDF dataset and database reader creation}{}
In a tool for preliminary design phase of an aircraft, it's very important to have aviable database. It's possible to create database starting from graphics using external software. In this appendix will be explained the step required in order to digitalize the graphics, create an HDF dataset and set up the database-reader class in Jpad.

\section{Chart Digitization}
The first step required for create a dataset is to digitalize a chart. Often data is found presented in reports and references as functional X-Y type scatter or line plots. In order to use this data, it must somehow be digitized. This is made with an external software, such as {\itshape Plot Digitizer}. Plot Digitizer is a Java program used to digitize scanned plots of functional data. This program will allow you to take a scanned image of a plot (in GIF, JPEG, or PNG format) and quickly digitize values off the plot just by clicking the mouse on each data point.\cite{plotdigitizer}

\begin{figure}[H]
\centering
{\includegraphics[height=7.9cm]{Immagini/digitize.png}} 
\caption{Chart digitization using Plot Digitizer.}
\label{angles}
\end{figure} 


In order to digitize a chart, first of all it's necessary to calibrate the axis. Plot Digitizer works with both linear and logarithmic axis scales. After it's possible to digitize a curve simply click o it. The values obtained can then be saved to a tex file or .csv file.


\section{Creation of an HDF file with Matlab}

Obtained the .csv file from digitization is necessary to create the HDF file. First of all it's necessary to import the file with couple of coordinates as matrix. After saving the imported files as .mat file, Matlab code comes in play to manage these data and to generate the digitalized curves and the HDF dataset.  The code interpolates curves points with cubic splines in order to have more points to plot for each curve.

\bigskip
\lstset{language=Matlab}
\begin{lstlisting}[frame=rbl,caption={{\footnotesize MATLAB script for creating the HDF Database}},label= [style=\bfseries]{Listing}]
clc; close all; clear all;

%% Import data
DeltaAlphaCLmax_vs_LambdaLE_dy1p2 = importdata('DeltaAlphaCLmax_vs_LambdaLE_dy1p2.mat');
DeltaAlphaCLmax_vs_LambdaLE_dy2p0 = importdata('DeltaAlphaCLmax_vs_LambdaLE_dy2p0.mat');
DeltaAlphaCLmax_vs_LambdaLE_dy3p0 = importdata('DeltaAlphaCLmax_vs_LambdaLE_dy3p0.mat');
DeltaAlphaCLmax_vs_LambdaLE_dy4p0 = importdata('DeltaAlphaCLmax_vs_LambdaLE_dy4p0.mat');

nPoints = 30;
lambdaLEVector_deg = transpose(linspace(0, 40, nPoints));

%% dy/c = 1.2
smoothingParameter = 0.999999;
DAlphaVsLambdaLESplineStatic_Dy1p2 = csaps( ...
    DeltaAlphaCLmax_vs_LambdaLE_dy1p2(:,1), ...
    DeltaAlphaCLmax_vs_LambdaLE_dy1p2(:,2), ...
    smoothingParameter ...
    );

DAlphaVsLambdaLEStatic_Dy1p2 = ppval( ...
    DAlphaVsLambdaLESplineStatic_Dy1p2, ...
    lambdaLEVector_deg ...
    );

%% dy/c = 2.0

smoothingParameter = 0.999999; 
DAlphaVsLambdaLESplineStatic_Dy2p0 = csaps( ...
    DeltaAlphaCLmax_vs_LambdaLE_dy2p0(:,1), ...
    DeltaAlphaCLmax_vs_LambdaLE_dy2p0(:,2), ...
    smoothingParameter ...
    );

DAlphaVsLambdaLEStatic_Dy2p0 = ppval( ...
    DAlphaVsLambdaLESplineStatic_Dy2p0, ...
    lambdaLEVector_deg ...
    );

%% dy/c = 3.0

smoothingParameter =0.999999;
DAlphaVsLambdaLESplineStatic_Dy3p0 = csaps( ...
    DeltaAlphaCLmax_vs_LambdaLE_dy3p0(:,1), ...
    DeltaAlphaCLmax_vs_LambdaLE_dy3p0(:,2), ...
    smoothingParameter ...
    );

DAlphaVsLambdaLEStatic_Dy3p0 = ppval( ...
    DAlphaVsLambdaLESplineStatic_Dy3p0, ...
    lambdaLEVector_deg ...
    );

%% dy/c = 4.0

smoothingParameter = 0.999999; 
DAlphaVsLambdaLESplineStatic_Dy4p0 = csaps( ...
    DeltaAlphaCLmax_vs_LambdaLE_dy4p0(:,1), ...
    DeltaAlphaCLmax_vs_LambdaLE_dy4p0(:,2), ...
    smoothingParameter ...
    );

DAlphaVsLambdaLEStatic_Dy4p0 = ppval( ...
    DAlphaVsLambdaLESplineStatic_Dy4p0, ...
    lambdaLEVector_deg ...
    );


%% Plots
figure(1)
plot ( ...
    lambdaLEVector_deg, DAlphaVsLambdaLEStatic_Dy1p2, '-*b' ... , ...
 );
 hold on
 
 plot ( ...
    lambdaLEVector_deg, DAlphaVsLambdaLEStatic_Dy2p0, '-b' ... , ...
 );

hold on

plot ( ...
    lambdaLEVector_deg, DAlphaVsLambdaLEStatic_Dy3p0, '*b' ... , ...
 );

hold on

plot ( ...
    lambdaLEVector_deg, DAlphaVsLambdaLEStatic_Dy4p0, 'b' ... , ...
 );

 xlabel('\Lambda_{le} (deg)'); ylabel('\Delta\alpha_{C_{L,max}}');
 title('Angle of attack increment for wing maximum lift in subsonic flight');
  legend('\Delta y/c = 1.2', '\Delta y/c = 2.0', '\Delta y/c = 3.0','\Delta y/c = 4.0');
 axis([0 50 0 9]);
 grid on;
 

 
%% preparing output to HDF

% dy/c
dyVector = [ ...
    1.2;2.0;3.0;4.0 ...
    ];

%columns --> curves
myData = [ ...
    DAlphaVsLambdaLEStatic_Dy1p2,...
        DAlphaVsLambdaLEStatic_Dy2p0, ... % -> 2
        DAlphaVsLambdaLEStatic_Dy3p0, ... % -> 3
        DAlphaVsLambdaLEStatic_Dy4p0];    % -> 4

hdfFileName = 'DAlphaVsLambdaLEVsDy.h5';

if ( exist(hdfFileName, 'file') )
    fprintf('file %s exists, deleting and creating a new one\n', hdfFileName);
    delete(hdfFileName)
else
    fprintf('Creating new file %s\n', hdfFileName);
end

% Dataset: data
h5create(hdfFileName, '/DAlphaVsLambdaLEVsDy/data', size(myData'));
h5write(hdfFileName, '/DAlphaVsLambdaLEVsDy/data', myData');

% Dataset: var_0
h5create(hdfFileName, '/DAlphaVsLambdaLEVsDy/var_0', size(dyVector'));
h5write(hdfFileName, '/DAlphaVsLambdaLEVsDy/var_0', dyVector');

% Dataset: var_1
h5create(hdfFileName, '/DAlphaVsLambdaLEVsDy/var_1', size(lambdaLEVector_deg'));
h5write(hdfFileName, '/DAlphaVsLambdaLEVsDy/var_1', lambdaLEVector_deg');
\end{lstlisting}

\noindent \\ \\ 
This script plot the graph after digitization. In this way it's possible to compare the initial graph and the digitized one.


 \begin{figure}[H]
\centering
{\includegraphics[height=6cm]{Immagini/digitize2.png}} 
\label{angles}
\end{figure} 




%
%%-------------------------------------------------------- APPENDIX 2
%\input{appendices/NativeFDMProperties}

%\input{appendices/RichiamiAerodinamica_ProfiliAlari_AddOn}


%
\end{appendices}
%
%
% -----------------------------------------------------------------------------------------
%                                                                     B I B L I O G R P H Y
\cleardoublepage \phantomsection
\addcontentsline{toc}{chapter}{Bibliography}
\pagestyle{myBibliographyPageStyle}
\nocite{*}
%\printbibheading[title={Bibliography}]
\printbibliography[heading=myBibliography]% needs biblatex, see myBibliography in _preamble.tex

% -----------------------------------------------------------------------------------------
%                                       P R I N T   G L O S S A R Y   and   A C R O N Y M S
%
% add all the entries to the glossaries without referencing each one explicitly
\glsaddall
%
\glossarystyle{list}%    can choose other style for displying glossaries
% styles -->   list, listdotted, long, longragged, super, tree
%
%\printglossaries
%%%
%%%  other possible uses:
%%%
%
\markboth{}{}
\cleardoublepage
\pagestyle{myGlossaryPageStyle}
\printglossary[type=main]% the default glossary is printed
%
\markboth{}{}
\cleardoublepage
\pagestyle{myAcronymsPageStyle}
\printglossary[type=\acronymtype]
%
\markboth{}{}
\cleardoublepage
\pagestyle{myListOfSymbolsPageStyle}
\printglossary[type=symbols]

\newpage
\thispagestyle{empty}

% -----------------------------------------------------------------------------------------
%                                                                                 I N D E X
%\markboth{}{}
%\cleardoublepage
%\pagestyle{myIndexPageStyle}
%
%\indexnote{%
%In the index we have many voices, of various types. The page number on the right
%reveals where those terms appear in the book.%{\parfillskip=0pt\par}
%
%\medskip\noindent
%\blindtext% this is just some dummy text
%
%\medskip
%{\color{red}
%The index---and index entries---should be the very last thing to work at when writing a book!
%So, do not worry about what you see below. These items are going to change many times before
%the final version of the document is released.}
%}% end-of-indexnote
%
%\phantomsection
%\printindex
%
%\markboth{}{}
%\cleardoublepage
%
% -----------------------------------------------------------------------------------------
%                                                                       T O   D O   L I S T
%
%\listoftodos[List of TO-DOs and comments]
%

\end{document}
%---------------------------------------------------------------------------------------------------
%                                                                                                                                 E N D    D O C U M E N T
%
