%------------------------------------------------------------------------------------------
% Meta-commands for the TeXworks editor
%
% !TeX root = ./Tesi.tex
% !TEX encoding = UTF-8
% !TEX program = pdflatex
%

%--------------------------------------------------------------------------------
%                                                G L O S S A R Y    E N T R I E S

%%% ---------------------------------------------------------------- Main glossary

\newdualentry{AF} % label
  {A.F.}            % abbreviation
  {Autonomy Factor}  % long form
  {Is the combination of three main efficiency: the propulsive efficiciency represented by SFC, the propeller efficiency $\eta_{p}$ or the jet efficiency represented by V and, finally, the aerodynamic efficinecy $\frac{L}{D}$} % description

\newdualentry{JPAD} % label
  {JPAD}            % abbreviation
  {Java Program toolchain for Aircraft Design}  % long form
  {Collection of libraries and classes with the aim of providing complete aircraft preliminary design analyses through the use of several semi-empirical formulas tested against experimental data} % description

\newdualentry{ACRF} % label
  {ACRF}            % abbreviation
  {Aicraft Construction Reference Frame}  % long form
  {The reference frame which has its origin in the fuselage forwardmost point, the x-axis pointing from the nose to the tail, the y-axis from fuselage plane of symmetry to the right wing (from the pilot's point of view) and the z-axis from pilot's feet to pilot's head} % description

\newdualentry{DOC}%
  {D.O.C.}            % abbreviation
  {Direct Operative Cost}  % long form
  {The totality of aircraft costs directly connected to the aircraft flight. It can be seen as the amount of money necessary to carry 1 ton of payload upon 1 km}

\newdualentry{HDF}%
  {HDF}            % abbreviation
  {Hierarchical Data Format}  % long form
  {A set of file formats (HDF4, HDF5) designed to store and organize large amounts of data}
  
\newdualentry{FAR}%
 {FAR}            % abbreviation
 {Federal Aviation Regulations}  % long form
 {The Federal Aviation Regulations, or FARs, are rules prescribed by the Federal Aviation Administration (FAA) governing all aviation activities in the United States}
  
\newdualentry{TAS}%
  {TAS}            % abbreviation
  {True AirSpeed}  % long form
  {Is the speed of the aircraft relative to the airmass in which it is flying; TAS is the true measure of aircraft performance in cruise, thus it is the speed listed in aircraft specifications, manuals, performance comparisons, pilot reports, and every situation when cruise or endurance performance needs to be measured. It is the speed normally listed on the flight plan, also used in flight planning, before considering the effects of wind}
  
\newdualentry{PNG} % label
  {png}            % abbreviation
  {Portable Network Graphics}  % long form
  {A raster graphics file format that supports lossless data compression. PNG was created as an improved, non-patented replacement for Graphics Interchange Format (GIF), and is the most used lossless image compression format on the Internet} % description
  
\newglossaryentry{TiKZ}{ % 
  name=TikZ,      
  description={Is a set of higher-level macros that use PGF}
}

\newdualentry{KBE}% label
  {KBE}            % abbreviation
  {Knowledge-Based Engineering}  % long form
  {KBE is essentially engineering on the basis of knowledge models. A knowledge model uses knowledge representation to represent the artifacts of the design process (as well as the process itself) rather than, or in addition to, conventional programming and database techniques. KBE can have a wide scope that covers the full range of activities related to Product Lifecycle Management and Multidisciplinary Design Optimization. KBE's scope includes design, analysis, manufacturing, and support} % description

\newdualentry{IDE}% label
  {IDE}            % abbreviation
  {Integrated Development Environment}  % long form
  {An integrated development environment (IDE) is a software application that provides comprehensive facilities to computer programmers for software development. An IDE normally consists of a source code editor, build automation tools and a debugger. Most modern IDEs have an intelligent code completion. Some IDEs contain a compiler, interpreter, or both, such as NetBeans and Eclipse; others do not, such as SharpDevelop and Lazarus. The boundary between an integrated development environment and other parts of the broader software development environment is not well-defined. Sometimes a version control system, or various tools to simplify the construction of a Graphical User Interface (GUI), are integrated. Many modern IDEs also have a class browser, an object browser, and a class hierarchy diagram, for use in object-oriented software development} % description

\newdualentry{MDO}% label
  {MDO}            % abbreviation
  {Multi-disciplinary Design Optimization}  % long form
  {Multi-disciplinary design optimization (MDO) is a field of engineering that uses optimization methods to solve design problems incorporating a number of disciplines.MDO allows designers to incorporate all relevant disciplines simultaneously. The optimum of the simultaneous problem is superior to the design found by optimizing each discipline sequentially, since it can exploit the interactions between the disciplines. However, including all disciplines simultaneously significantly increases the complexity of the problem} % description


\newglossaryentry{List}{%
	name=List,
	description={The java.util.List interface is a subtype of the java.util.Collection interface. It represents an ordered list of objects, meaning you can access the elements of a List in a specific order, and by an index too. You can also add the same element more than once to a List}
}

\newglossaryentry{Enum}{%
	name=Enumeration,
	description={The java.util.Enumeration interface represents a special data type that enables for a variable to be a set of predefined constants. The variable must be equal to one of the values that have been predefined for it. Because they are constants, the names of an enum type's fields are in uppercase letters.}
}

\newglossaryentry{Static}{%
	name=static method,
	description={The term static means that the method is available at the Class level, and so does not require that an object is instantiated before it's called}
}

\newglossaryentry{Map}{%
	name=Map,
	description={The java.util.Map interface represents a mapping between a key and a value. The Map interface is not a subtype of the Collection interface. Therefore it behaves a bit different from the rest of the collection types}
}

\newglossaryentry{Interface}{%
	name=Interface,
	description={In the Java programming language, an interface is a reference type, similar to a class, that can contain only constants, method signatures, default methods, static methods, and nested types. Method bodies exist only for default methods and static methods. Interfaces cannot be instantiated—they can only be implemented by classes or extended by other interfaces.}
}

\newglossaryentry{DATCOM}{%
	name=DATCOM,
	description={Digital Datcom is a computer program which calculates static stability, high lift and control, and dynamic derivative characteristics using the methods contained in the USAF Stability and Control Datcom (Data Compendium). Configuration geometry, attitude, and Mach range capabilities are consistent with those accommodated by the Datcom. The program contains a trim option that computes control deflections and aerodynamic increments for vehicle trim at subsonic Mach numbers}
}

\newglossaryentry{parsing}{%
	name=parsing,
	description={Parsing or syntactic analysis is the process of analysing a string of symbols, either in natural language or in computer languages, conforming to the rules of a formal grammar}
}

\newglossaryentry{User:Developer}{%
   name=user developer,
   description={%
      The term refers to the developer which will use a method without being interested in how the method performs the required action. This is the case of a utility method: the developer is the one who writes the method, while the user developer is who uses that method to accomplish some action which requires the functionality provided by the utility method. It has to be noticed that the user developer and the developer can be the same person}
}

%%% --------------------------------------------------------------- List of symbols
%%% see _local_macros.tex

\newglossaryentry{vec:g}{%
   type=symbols,
   name={\ensuremath{\vec{g}}},
   sort=g,
   description={gravitational acceleration}
}

\newglossaryentry{weight}{%
   type=symbols,
   name={\ensuremath{W}},
   sort=W,
   description={weight, in N or lbf}
}

\newglossaryentry{n}{%
   type=symbols,
   name={\ensuremath{n}},
   sort=n,
   description={load factor}
}

\newglossaryentry{span}{%
   type=symbols,
   name={\ensuremath{b}},
   sort=b,
   description={span}
}

\newglossaryentry{velocity}{%
   type=symbols,
   name={\ensuremath{V}},
   sort=V,
   description={scalar velocity}
}

\newglossaryentry{surface}{%
   type=symbols,
   name={\ensuremath{S}},
   sort=S,
   description={surface}
}

\newglossaryentry{ar}{%
   type=symbols,
   name={\ensuremath{\AR}},
   sort=A,
   description={aspect ratio}
}

\newglossaryentry{taperRatio}{%
   type=symbols,
   name={\ensuremath{\lambda}},
   sort=m,
   description={taper ratio}
}

\newglossaryentry{sweep}{%
   type=symbols,
   name={\ensuremath{\Lambda}},
   sort=m,
   description={sweep}
}

\newglossaryentry{thickness}{%
   type=symbols,
   name={\ensuremath{t}},
   sort=m,
   description={thickness}
}

\newglossaryentry{chord}{%
   type=symbols,
   name={\ensuremath{c}},
   sort=m,
   description={chord}
}

\newglossaryentry{rho}{%
   type=symbols,
   name={\ensuremath{\rho}},
   sort=zz:rho,
   description={air density}
}

\newglossaryentry{alpha:Wing}{%
   type=symbols,
   name={\ensuremath{\alpha_\Wing}},
   sort=zz:alpha:W,
   description={angle of attack referred to wing root chord}
}

\newglossaryentry{Drag}{%
   type=symbols,
   name={\ensuremath{D}},
   sort=D,
   description={aerodynamic drag}
}

\newglossaryentry{DragCoeff}{%
   type=symbols,
   name={\ensuremath{C\textsubscript{D}}},
   sort=C,
   description={aerodynamic drag coefficient}
}

\newglossaryentry{ParasiteDragCoeff}{%
   type=symbols,
   name={\ensuremath{C\textsubscript{D0}}},
   sort=C,
   description={aerodynamic parasite drag coefficient}
}

\newglossaryentry{Lift}{%
   type=symbols,
   name={\ensuremath{L}},
   sort=L,
   description={aerodynamic lift}
}

\newglossaryentry{LiftCoeff}{%
   type=symbols,
   name={\ensuremath{C\textsubscript{L}}},
   sort=C,
   description={aerodynamic lift coefficient}
}

\newglossaryentry{Thrust}{%
   type=symbols,
   name={\ensuremath{T}},
   sort=T,
   description={thrust}
}

\newglossaryentry{Range}{%
   type=symbols,
   name={\ensuremath{R}},
   sort=R,
   description={range in nmi or km}
}

\newglossaryentry{etaProp}{%
   type=symbols,
   name={\ensuremath{\eta_p}},
   sort=aaE,
   description={propeller efficiency}
}


\newglossaryentry{Mach}{%
   type=symbols,
   name={\ensuremath{\Mach}},
   sort=M,
   description={Mach number}
}

\newglossaryentry{i:Wing}{%
   type=symbols,
   name={\ensuremath{i_\Wing}},
   sort=iW,
   description={the angle between the wing root chord and the ACRF x-axis}
}

\newglossaryentry{sub:cruise}{%
   type=symbols,
   name={\ensuremath{(\;)_{cruise}}},
   sort=aaC,
   description={quantity related to cruise condition}
}

\newglossaryentry{sub:LG}{%
   type=symbols,
   name={\ensuremath{(\;)_\text{LG}}},
   sort=aaL,
   description={quantity related to the landing gear}
}

\newglossaryentry{sub:flap}{%
   type=symbols,
   name={\ensuremath{(\;)_\text{f}}},
   sort=aaF,
   description={quantity related to flaps}
}

\newglossaryentry{Mff}{%
   type=symbols,
   name={\ensuremath{M_\text{ff}}},
   sort=M,
   description={Fuel fraction over entire mission}
}

\newglossaryentry{WTO}{%
   type=symbols,
   name={\ensuremath{W\textsubscript{TO}}},
   sort=W,
   description={Take Off Weight}
}

\newglossaryentry{WOE}{%
   type=symbols,
   name={\ensuremath{W\textsubscript{OE}}},
   sort=W,
   description={Operating Empty Weight}
}

\newglossaryentry{Wpayload}{%
   type=symbols,
   name={\ensuremath{W\textsubscript{Payload}}},
   sort=W,
   description={Payload Weight}
}

\newglossaryentry{Wfuel}{%
   type=symbols,
   name={\ensuremath{W\textsubscript{fuel}}},
   sort=W,
   description={Fuel Weight}
}

%%% -------------------------------------------------------------------- Acronyms

\newacronym{acr:MZFW}{MZFW}{%
   Maximum Zero Fuel Weight
}

\newacronym{acr:SFC}{SFC}{%
   Specific Fuel Consumption
}

\newacronym{acr:AOE}{AOE}{%
   All Operative Engines
}

\newacronym{acr:CAE}{CAE}{%
   Computer-Aided Engineering
}

\newacronym{acr:OEI}{OEI}{%
   One Inoperative Engine
}

\newacronym{acr:SFCJ}{SFCJ}{%
   Jet Specific Fuel Consumption
}

\newacronym{acr:ODE}{ODE}{%
   Ordinary Differential Equations
}

\newacronym{acr:IVP}{IVP}{%
   Initial Value Problem
}

\newacronym{acr:UAV}{UAV}{%
   Unmanned Aerial Vehicles
}

\newacronym{acr:LER}{LER}{%
  Leading Edge Radius 
}

\newacronym{acr:AIAA}{AIAA}{%
American Institute of Aeronautics and Astronautics
}

\newacronym{acr:GUI}{GUI}{%
Graphical User Interface
}